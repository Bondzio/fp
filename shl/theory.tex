\section{Theory}
\subsection{Short introduction to nuclear physics}
\paragraph{In order to understand}
the notions of radioactie decay and half life, we
give a short introduction into nuclear physics following \cite{martin2006nuclear}
In the following, let $Z$ be the atomic number, $N$ the number of neutrons and $A$ the number of nuclei, also
called nucleon number. Thus, $A = Z + N$. The electric charge of the nucleus in the ground state is $+Z e$.
It is common to describe the so called \emph{nuclides} with $_{Z}^{A}\textrm{Y}$.\\
The forces binding the nuclei contribute to the total mass of an atom in terms of the 
binding energy $\Delta E = \Delta M c^2$:
\begin{equation}
\Delta M = M_{tot.} - Z(M_p + M_e) - N M_n \, .
\end{equation}
We used the proton, electron and neutron masses $M_p$, $M_e$ and $M_n$ 
(see figure~\ref{fig:bindingenergy}).
\begin{figure}[htpb]
    \centering
    \includegraphics[width=0.8\linewidth]{figures/bindingenergy}
    \caption{Binding energies for nuclei that are stable or long-lived \cite{Hooshyar}.}
    \label{fig:bindingenergy}
\end{figure}
If we look at the distribution of stable nuclei (see figure~\ref{fig:nuclidmap}) we notice
that they occur only in a very narrow band. All other nuclei are unstable and decay spontaneously.
The decay is characerized by a \emph{decay constant} $\lambda$, which is related to the activity $\mathcal{A}$
by 
\begin{equation}\label{eq:decay}
    \mathcal{A} = -\frac{\partial N}{\partial t} = \lambda N 
\end{equation}
where the activity $\mathcal{A}$ is typically given in Bequerel: $1 \textrm{Bq}= \textrm{decay}\cdot s^{-1}$.
A solution to the differential equation \eqref{eq:decay} is
\begin{equation}
    \mathcal{A}(t) = \lambda N_0 \exp(-\lambda t) \, ,
\end{equation}
with the initial condition $N_0 := N(t=t_0)$. 
The probability of an atom to decay within time $t$ is thus
\begin{equation}
    \mathcal{P}(t) = \int_{0}^{t}\lambda \exp(-\lambda t')\mathrm{d}t' = 1 -  \exp(-\lambda t)  \quad
    \textrm{with} \quad \int_{0}^{\infty}\lambda \exp(-\lambda t')\mathrm{d}t' = 1 \, .
\end{equation}
The probability density is accordingly $f(t) = \lambda \exp(-\lambda t)$. 
Hence, we can calculate the expectation value of a random variable $t$ within the probability space
$\Omega$ with the measure  $\mathcal{P}$:
\begin{equation}
    \tau := E[t] = \int_{0}^{\infty} t \lambda \exp(-\lambda t) \mathrm{d}t 
    =\lim_{t \rightarrow \infty}\left[ \frac{exp(-\lambda t) (1-\lambda t) - 1 }{\lambda} \right] 
= \frac{1}{\lambda}
\end{equation}
The half-life $t_{1/2}$ is obviously connected by $t_{1/2}= -\mathrm{log}(2)/ \lambda = - \tau \mathrm{log}(2)$.
\begin{figure}[htpb]
    \centering
    \includegraphics[width=0.9\linewidth]{analysis/figures/halflife}
    \caption{Probability $\mathcal{P}(t) = 1 -  \exp(-\lambda t)$ of an atom to decay within time $t$.
        Here we chose as the decay of
        $^{57}\textrm{Cs}\rightarrow ^{57}\textrm{Fe}$ with $t_{1/2}=270$ over a range
    of 4 years as an example.}
    \label{fig:decay}
\end{figure}
\clearpage

\subsubsection{Semi-Empirical Mass Formula: The liquid drop model}
\begin{figure}[htpb]
    \centering
    \includegraphics[width=0.6\linewidth]{figures/nuclidmap}
    \caption{Nuclid map of stable nuclei \cite{Hooshyar}: The squares are the long-lived nuclei
    occuring in nature; other known nuclei lie within the jagged lines and are unstable.}
    \label{fig:nuclidmap}
\end{figure}
\label{ssub:Semi-Empirical Mass Formula: The Liquid Drop Model}
\paragraph{Going along with}  
\cite{Hooshyar}, we approach a theoretical model containing constants which
have to be fitted with experiments. For this reason, the model is called semi-empirical and 
is often referred to as \emph{SEMF} (semi-empirical mass formula).
It was first established by Weizsäcker and yields a 
good approximation for the atomic masses. The assumptions underlying are:
\begin{enumerate}
    \item All nuclei have approximately the same interio mass density.
        \label{it1}
    \item Their total binding energies are approximately proportional to their masses.
        \label{it2}
\end{enumerate}
Both turnout to be valid in the regime we are looking at.
The assumptions show why the model is called \emph{liquid drop model}: 
(\ref{it1}) is the analogue of regarding the nuclei as incompressible fluids, 
(\ref{it2}) is related to the proportionality of latent heats of vaporization to the masses of a drop. 
We construct the energy in units of masses with different contributions:
\begin{equation}
    M_{tot} = \sum_{i=0}^{5} f_i 
\end{equation}
\begin{itemize}
    \item \textbf{Mass term:} 
First we take into account the atomic masses 
consisting of the mass of the nucleons and electrons:
\begin{equation}
    f_0 = Z(M_p + m_e) + (A-Z)M_n
\end{equation}
    \item \textbf{Volume term:} This term will estimate the effect of strong 
        nuclear forces proportional to the volume:
        \begin{equation}
            f_1 = -a_1 A\, ,
        \end{equation}
        where $a_1$ has to be found depending on the volume. 
    \item \textbf{Surface term:} 
        This term corrects the volume energy by a surface term
        \begin{equation}
            f_2 = a_2 A^{\frac{2}{3}} 
        \end{equation}
        taking into account that in binding energy is lowered as nuclei at the surface 
        have fewer interaction partners. The analogue is the surface tension energy 
        in the classical liquid drop model.
    \item \textbf{Coulomb term:} The protons of each nucleus repell each other due to the 
        electrostatic Coulomb force.If we assume a uniform charge distribution of 
        radius proportional to $A^{\frac{1}{3}}$, then
        \begin{equation}
            f_3 = a_3 \frac{Z(Z-1)}{A^{\frac{1}{3}}} \approx a_3 \frac{Z^2}{A^{\frac{1}{3}}} 
            \qquad (\text{for} \quad Z \gg 1)
        \end{equation}
    \item \textbf{Assymetric term:} Due to the Pauli principle the nuclei tend to broaden their distribution
        on energies, with leads to a positive energy correction for more assymetric numbers $A$ and $Z$:
        \begin{equation}
            f_4 = a_4 \frac{(Z- \frac{A}{2})^2}{A}
        \end{equation}
    \item \textbf{Pairing term:} This correction accounts for the tendency of proton pairs and neutron pairs
        to occur, where an even number of particles is more stable than an odd number:
      \begin{equation}
          f_5   =
          \begin{cases}
              -f(A) & \text{if $Z$ even and $A-Z=N$ even}\\
              0     & \text{if $Z$ even but $A-Z=N$ odd or $Z$ odd but $A-Z=N$ even}\\
              f(A)  & \text{if $Z$ odd and $A-Z=N$ odd}
          \end{cases}
          \label{eq:pair}
      \end{equation}
      The function $f(A)$ should be estimated by fitting the data, 
      often $f(A) = a_5 A^{-\frac{1}{2}}$ is used.
\end{itemize}

\subsubsection{Unstable States}
\label{ssub:Unstable States}
\paragraph{If we want investigate}
the form of the energy distribution of a decay, we have to 
introduce the \emph{natural decay width}, given by
\begin{equation}
    \Gamma_f = \frac{\hbar}{\tau_f} 
\end{equation}
which can be defined for each channel $f$, such that in total we get
\begin{equation}
    \Gamma = \sum_{f} \Gamma_f \, .
\end{equation}
We can define the \emph{branching ratio} for channel $f$ by
\begin{equation}
    B_f = \frac{\Gamma_f}{\Gamma} \, .
\end{equation}
The energy distribution of the unstable state to a final state $f$ resolves into a 
\emph{Breit-Wigner-distribution}: 
\begin{equation}
    N_f(M_f) = \frac{\Gamma_f}{(M_f-M_i)^2 c^4 + \Gamma_f^2/4}
\end{equation}
with the mass of the decaying state $M_i$ and the invariant mass of decay products $M_f$. \\\\
\paragraph{At this point one}
might ask why it is necessary to make use of such a sophisticated setup (see next chapter
for the technical devices we are using). It would be much easier to measure a very high resolution of the
energy peak of the process and fit the peak with the a Breit-Wigner distribution in order to get $\Gamma$. 
In the interest of investigating this question, we will do some approximations regarding to the
half life period we are interested in, whereas we have $E = 14.4$ eV and $t_{1/2} = 98$ns:
\begin{equation}
    \Gamma = \frac{\hbar}{\tau} = \frac{\log(2)\cdot \hbar}{t_{1/2}} \approx 4.66 \cdot 10^{-9} \mathrm{eV} 
\end{equation}
This is beyond the limits of a meaningful calculation or measurement, which indicates that the peak should
be totally sharp with regards to the Breit-Wigner shape. The broad width of our curves is not due to the
resonance but due to a gaussian process related to thermodynamical influences. Hence it is not possible
in this case\footnote{In other cases it is a good possibility, but only when the width $\Gamma$ is comparable
to the resolution of the experimental setup of the spectroscope. This is for instance the case for extremely
short half lifes as possible in high energy physics.}
to obtain the half life period with the help of the energy distribution.
\clearpage

\subsection{Modes of radioactive decay}
\paragraph{Radioactivity has been}
experimentally observed since the end of the 19th century, 
starting with the experiments of Henri Becquerel (1896), when the atom was still seen 
either within the 'plum pudding model', and later, after the experiments performed by 
Rutherford et.~al. in 1911, a 'planetary model'. Rutherford coined the terms still in 
use today to describe various types of radioactive decay, using the greek letters 
$\alpha$, $\beta$ and $\gamma$. A more refined but not complete overview is 
given by the following.~ \cite{martin2006nuclear}
The mass and atomic numbers of the daughter nuclei ($A'$ and $Z'$, respectively) 
is given in terms of $A$ and $Z$ corresponding to the parent nucleus. 
\begin{description}
    \item [Decay involving the emission of nuclei:] Changing both $A$ and $Z$.
        \begin{itemize}
            \item
                \textbf{$\alpha$-decay:} a $~_2^4$He particle is emitted, the daughter nucleus 
                has a mass number of $A' = A - 4$ and atomic number $Z' = Z - 2$. Example:
                \begin{equation}
                    \mathrm{~^{238}_{92}U}\rightarrow\mathrm{~^{234}_{90}Th} + \mathrm{~^{4}_{2}He} 
                \end{equation}
            \item
                \textbf{Spontaneous fission:} parent nucleus breaks up into two daughter nuclei \emph{without}
                external action. This only occurs in very heavy nuclei. $\mathrm{~^{238}_{92}U}$ can be
                used as an example in this case, as well:
                \begin{equation}
                    \mathrm{~^{238}_{92}U}\rightarrow\mathrm{~^{145}_{57}La} + \mathrm{~^{90}_{35}Br} + 3n  \, .
                \end{equation}
                Instead of breaking into two parts of equal size, the most probable cases yield daughter 
                nuclei differing by approximately 45 in mass number -- a question that remains to be solved 
                until today.
        \end{itemize}
    \item [$\beta$ decay:]
        This form of radioactive decay is observed in atoms of lower mass also. 
        It conserves mass number $A$. The decay chain follows characteristic mass parabolas which 
        can be predicted by the SEMF and have a minimum corresponding to the stable end of a decay chain.
        \begin{itemize}
            \item
                \textbf{$\beta^-$ decay:} conversion of a neutron into a proton with emission of electron and 
                electron antineutrino, raising the atomic number $Z$ by one:
                \begin{equation}
                    {}^{1}_{0} \mathrm {n} \to {}^{1}_{1} \mathrm {p} + \mathrm{e}^{-} + \overline{\nu}_e 
                \end{equation}
            \item
                \textbf{$\beta^+$ decay:} conversion of a proton into a neutron with emission of positron and 
                electron neutrino, reducing the atomic number $Z$ by one:
                \begin{equation}
                    {}^{1}_{0} \mathrm {p} \to {}^{1}_{1} \mathrm {n} + \mathrm{e}^{+} + \nu_e 
                \end{equation}
            \item
                \textbf{Electron capture (EC):} an electron of one of the inner orbitals is 'captured', transforming a proton 
                into a neutron under emission of an electron neutrino
                \begin{equation}
                    {}^{1}_{0} \mathrm {p} + \mathrm{e}^{+} \to {}^{1}_{1} \mathrm {n} + \nu_e 
                \end{equation}
                In most cases, the electron stems from the K-orbital. The daugther nucleus is left in 
                an unstable excited state, like in the case of the $^{57}$Co used in the experiment. 
                Figuratively, the hole in the lower shell produced by the $e^-$-capture is successively 
                passed to the outer ones under emission of characteristic $\gamma$-rays. 
        \end{itemize}
    \item [Transition between states of the same nucleus:] 
        A nucleus in an excited state decays into a state of 
        lower energy without changing $A$ and $Z$.
        \begin{itemize}
            \item
                \textbf{Isomeric transition:} The energy is transfrerred to a photon. This is the process to be observed 
                in the experiment. 
            \item
                \textbf{Internal conversion:} Succession transfer of energy to an elctron of the outer orbital, 
                which is then ejected. The resulting excited state is lowered by emission of characteristic 
                x-rays or a so-called \emph{Auger electron}. 
        \end{itemize}
\end{description}

\subsection{Expected peaks}
\paragraph{Examining the two probes}
at hand, we expect various peaks in the energy spectrum. 
Depending on the kinds of decay and energies involved, there are five outstanding features:~\cite{ver}
\begin{description}
    \item [Photo peak:] 
        The entire energy $E_\gamma$ of the emitted photon is transfered to the scintillator 
        via the photoelectric effect, if all electrons and photons of the further relaxation are 
        absorbed by the scintillator.
    \item [Escape peak:] 
        If a x-ray quantum escapes from the scintillator, the entry is recorded in the 
        escape peak bucket. In our case of the NaI(Ti)-scintillator, the corresponding photoelectric 
        effect occurs preferably at the iodine atoms, for which the energy difference between K- and L-orbital 
        and therefor the energy of the characteristic x-rays is 28 keV. We thus expect an escape peak 
        at $E_\gamma - 28$ keV. 
    \item [Compton edge:] 
        The Compton effect allow a transfer in energy within a specific but continuous range. 
        The edge of this range, the maximal energy transfered by the involved elastic scattering, 
        is called the compton edge. However, since the Compton effect is dominant at a range of 
        $E_\gamma \in (200\, \mathrm{keV}, 5\, \mathrm{MeV})$ for $Z \approx 50$, we don't expect it to be of 
        measurable magnitude in this experiment. 
    \item [Backscattering peak:] 
        Photons scattered from the shielding transfer part of their energy in the process 
        and are thus recorded in the backscattering peak. 
    \item [X-ray fluorescence peak:] 
        Interaction of the $\gamma$-quanta and electrons in the shielding can produce 
        yet another peak, if the original photon is not but the resulting photons or 
        Auger electrons are recorded in the scintillator. 
\end{description}

\subsection{Decay of $^{57}$Co and $^{241}$Am}
\paragraph{The excited state to be}
examined is of the isotope $^{57}$Fe, one of four stable isotopes 
with an abundance of 2.1\%~\cite{nist}. We do not excite a sample of the isotope, though, but 
use a $^{57}$Co sample, which decays into a state energetically above the one considered. 
A simplified scheme of the decay is shown in figure \ref{fig:decay_scheme_Co}.
The $^{57}$Co atoms decay due to electron capture, such that the nucleon number is not changed. 
The half life for this decay is 270 days, so the sample is expected to have a somewhat 
moderate activity. We also note that the activity might be considerably low, 
such that the expected results may be of lower excactness compared to those obtained in 
the previous years.
This state lies at 136.5 keV above the ground state and has a half live of only 9 ns. 
It decays into the 14.4 keV state with a probability of 88\%. This latter state, which is the one 
we will examine, has a nominal half live of 98 ns and decays directly to the ground level. 
A photon, however, is only emitted in 19\% of the cases -- the bulk tranfers to the ground level 
by inner conversion and can thus not be registered by the detectors used in the experiment.%
~\cite{ver}
\begin{SCfigure}
    \begin{centering}
        \includegraphics[width=0.70\linewidth]{figures/decay_scheme_Co}
        \caption{Simplified decay scheme of $^{57}$Co, modified from~\cite{ver}}
        \label{fig:decay_scheme_Co}
    \end{centering}
\end{SCfigure}

\paragraph{The second sample under}
consideration is the $^{241}$Am sample, which by $\alpha$ decay turns into $^{237}$Np, the 
first transuranic element which has been synthesized only in 1940. The excited 
states expected to be observable in this experiment are shown in figure~\ref{fig:decay_scheme_Am}. 
After the $\alpha$ decay, the nuclues is found in the upper state with $59.5$ keV 
above the ground state and a half live of $67$ ns. It follows a decay either into another 
but very short lived ($T_{1/2} < 4$ ns) excited state at $33.2$ keV or into the 
ground state. The $33.2$ keV falls into the ground state, as well. All described conversions 
emit a photon and are thus expected to be visible. This sample will be used exclusively for the 
identification of the cited peaks in order to develope the relationship between channels and 
photon energy. 
\begin{SCfigure}
    \begin{centering}
        \includegraphics[width=0.70\linewidth]{figures/decay_scheme_Am}
        \caption{Simplified decay scheme of $^{57}$Am, taken from~\cite{ver}}
        \label{fig:decay_scheme_Am}
    \end{centering}
\end{SCfigure}


\subsection{Overview of technical instruments}
\paragraph{We will give a short}
overview over the technical instruments used. The exact role 
of the devices will be explained in section thereafter.
\label{sub:overview_of_technical_instruments}
\paragraph{Scintillator}
In order to detect the $\gamma$ - rays from a radioactive sample, we use a widely used detector called/
\emph{scintillator}. The name refers to a quality of the material used: The \emph{scintillation}, or production 
of flashes of light induced by the passing $\gamma$ photons. 
The energy of incoming particles
is adsorbed and released at a lower frequency (ideally in the visible or UV-spectrum). 
We can measure the energy as well as the time of detection. 
In our case we use the organic crystal \textbf{NaI(Tl)} as scintillator, which
is preferable because of its high yield~\cite{ver} in comparison to others. However, the decay time is longer
than for organic scintillatiors, which is not a problem in our case since the low activity of our probe will
not result in (a measurable quantity of) jammed photons.
In the detector, this material is coupled to a photomultiplier in order
to gain an measurable signal. 

\paragraph{The Photomultiplier}
will be used to amplify the signal from the scintillator. It is built upon
two fundamental physical phenomena: The \textit{photoelectric effect} and the \textit{secondary emission}. While the former
is well known due to its scientific father\footnote{Despite the effect that Heinrich Hertz discovered the 
photoeletric effect 1887, Albert Einstein published an explanation 1905 with which he shows the quantum character
of light. 1921 he was rewarded with the Nobel Price in physics.}, the latter is less popular: Particles (with
sufficient energy) induce the emission of secondary particles while going through a specific material. In
our case the primary particles will be the photons emitted by the scintillator, while the secondary induced
in the photomultiplier are electrons. These electrons are accelerated towards an electrode (which is called
\textit{dynode} in this special case) in order to induce the emission of further electrons. 
The process is repeated over and over such that even single photons can be detected clearly
(see figure~\ref{fig:photomultiplier} for a schematic diagram).
Depending on the experimental setup it is possible to amplify a signal by eight orders of magnitude. 
\begin{figure}[htpb]
    \centering
    \includegraphics[width=0.8\linewidth]{figures/photomultiplier}
    \caption{
        Schematic diagram of a photomultiplier tube. The incoming photons are translated into electrons, 
        which are accelerated towards the
        next dynode such that more and more electrons are induced to amplify the signal.
        Taken from~\cite{martin2006nuclear}. 
        }
    \label{fig:photomultiplier}
\end{figure}
\paragraph{The Single Channel Analyzer (SCA)} is the next instrument to process the signal detected. In general
it refers to a electronic component creating a outputsignal only when the amplitude (of the input) happens to be
between predefined constraints. Hence the SCA is a binary unit. We will use it amongst others for
applying an energy window to the spectrum in order to indentify certain peaks with events. 
\paragraph{The Time-to-amplitude Converter (TAC)} is the missing link between energies detected by the scintillator
and times. It is possible to convert the amount of time between a start and a stop signal into an amplitude,
which can be further analyzed statistically. The component was developed by a famous italian physicist%
\footnote{1942 Bruno Rossi invented the TAC in order to ascertain the half life period of mesons. The first
    experiment of this kind was done in similar fashion as ours: The time difference between the
    arrival of a meson in an absorber and its decay through an electron emission was measured with an
    electronic circuit producing an amplitude proportional to the time interval. Quoting Rossi:
    \textit{``How is it possible that results bearing on fundamental problems of
        elementary particle physics could be achieved by experiments of an almost childish simplicity,
        costing only a few thousand dollars and requiring only the help of one or two graduate students? ``}}
and is used especially because of the linear relationship betwen amplitude and real time, a quality we will also
show in a measurement.
\paragraph{Multi Channel Analyzer (MCA)} are used intensively for sorting a stream of voltage pulses
into a histogram, integrating over a specific amount of time. The stored number of events versus pulse-height
can be analyzed afterwards.
