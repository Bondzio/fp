\section{conclusion}
\paragraph{The fitting of the energy}
spectra supplied reasonable results as they showed to be consistent 
with each other. At this point, there is no comparison at hand to check these results, however. 
The calibration of the TAC showed a linear behaviour as expected, giving
a correspondance between channels and time. 
Unfortunatelly, however,we were not able to reproduce the literature value of 
$T_{1/2, \mathrm{lit}}=98$ns. Instead, our procedure yields the result
\begin{equation}
T_{1/2} =\left [11.9 \pm 0.5 \right ] ns .
\end{equation}
Since the data resamples in a exponential way and the fitting parameter seem to agree very good with the data,
we would like to emphasize the possibiliy of a technical problem which we were not able to resolve yet. 
We consider two possibilities with varying probability:
\begin{itemize}
    \item 
        We might have chosen the wrong peak and therfore measured a different half life period of a different 
        process. Since there were a few peaks, this is not impossible but rather unprobable. However, since there
        is this great disparty between literatur and measurement, we conclude that this is a possibility.
    \item 
        Between delay and channel there is an element which we have not considered yet:  
        As our measurements in itself seem to be quite consistent, we suspect an error in the 
        time-channel correspondance of approximately 4x -- 5x. 
        We tried to run the analysis with different factors and realizied that even small changes
        shift the energy values by a lot. 
        The direct source of the problem remains unclear, lying in an error done 
        while measuring, in the setup or even in the analysis (after trying to exclude this 
            possibility from many directions).
        However, this possibility is clearly more probable than the former. 
\end{itemize}
\paragraph{Howbeit,} we consider this experiment an overwhelming success since we gained deep 
insight into various concepts of nuclear physics and statistical interference. 
