\section{Introduction}
In this experiment, we will measure the half life of an excited state of 
$^{57}$Fe. The state lies 14.4 keV above the ground state, in literatur its half live is given 
by $T_{1/2} = 98$ ns~\cite{ver}. With the given setup, 
we are able to measure the time difference between to successive events within this 
time span. By identifying the state under consideration as a member of a chain of successive decays, 
we can thus measure the time between begin and end of a single state. 
Prior to measuring the half life, we identify the peaks on the energy spectrum 
with the help of another know sample of $^{241}$Am. 


