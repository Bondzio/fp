\section{theory}
\label{sec:theory}
\subsection{Ionization detector}
\label{subsec:detector}

\subsection{plateaus}
\label{subsec:plateaus}

\subsection{Potassium}
\label{subsec:potassium}
The emission of $\beta$-rays is continuos due to the adsorption when
passing through matter. Following \cite{ver} we will observe the following
counting rate
\begin{equation}
n(m) = f_B \frac{\Omega}{4 \pi} A_s \frac{F \rho }{\mu}A \left (1 - \exp \left ( - \frac{\mu}{F \rho}m \right ) \right )
\label{eq:potassium}
\end{equation}
Where we defined the following coefficients:
\begin{itemize}
\item[] $A_s = A/ m$... specific activity of the preperation  
\item[] $\mu$ ... extinction coefficient  of the $\beta$ decay 
\item[] $\Omega = 2 \pi$... solid angle
\item[] $f_B\approx 1.29$ ... backscattering coefficient 
\end{itemize}
We will use a least squares optimization in order to estimate the coefficients for the equation.

