\section{theory}
\label{sec:theory}
\subsection{Ionization detector}
\label{subsec:detector}

\subsection{plateaus}
\label{subsec:plateaus}

\subsection{Potassium}
\label{subsec:potassium}
The emission of $\beta$-rays is continuos due to the adsorption when
passing through matter. Following \cite{ver} we will observe the following
counting rate
\begin{equation}
n(m) = f_B \frac{\Omega}{4 \pi} A_s \frac{F \rho }{\mu}A \left (1 - \exp \left ( - \frac{\mu}{F \rho}m \right ) \right )
\label{eq:potassium}
\end{equation}
Where we defined the following coefficients:
\begin{itemize}
\item[] $A_s = A/ m$... specific activity of the preperation  
\item[] $\mu$ ... extinction coefficient  of the $\beta$ decay 
\item[] $\Omega = 2 \pi$... solid angle
\item[] $f_B\approx 1.29$ ... backscattering coefficient 
\end{itemize}
We will use a least squares optimization in order to estimate the coefficients for the equation:
\begin{equation}
n(m) = a ( 1 - e^{-bm})
\end{equation}
such that we can identify the parameters
\begin{align}
a &\equiv  f_B \frac{\Omega}{4 \pi} A_s \frac{F \rho }{\mu}A  \label{eq:a}\\ 
b &\equiv \frac{\mu}{F \rho} \label{eq:b}
\end{align}
For calculating the half life time, we use $T_{1/2} = \frac{\mathrm{log}(2)}{\lambda}$. Unfortunatelly,
we cannot use the $\lambda$ directly for the activity, since the activity does not resample the real activity, but
the due to electron capture reduced activity. We can use the following estimation \cite{ver}:
\begin{equation}
\lambda = \lambda_{\beta} + \lambda_{EC} = \lambda_{\beta} + \frac{0.1072}{0.8928}\lambda_{\beta} = 1.12 \lambda_{\beta}
\end{equation}
which let us arrive at the following result
\begin{align}
A = N \lambda = 1.12 \cdot N\lambda_\beta= N \tau ^{-1} \Leftrightarrow  T_{1/2} = \frac{N \mathrm{log}(2)}{1.12 \cdot A}.
\label{eq:As}
\end{align}
Additionally to that, we are using the specific Activity $A_s = A / m$ with the molar mass $m$. Since we are not using
pure $^{40}\textrm{K}$ cores but KCl, we have to take the relative occurence of those in our sample into consideration by
\begin{equation}
N = \frac{h_{rel} m N_A}{m_{\mathrm{KCl}}}
\end{equation}
with the Number of $^{40}\textrm{K}$ cores, the relative mass ratio of $^{40}\textrm{K}$ in natural kalium $h_{rel} = 1.18 \cdot 10^{-4}$,
the Avogadro constant $N_A$ and the molar mass $m_{\mathrm{KCl}}$ of KCl.
Plugging in equation \eqref{eq:a}, \eqref{eq:As} and in the end \eqref{eq:b} we get to the final result (where the factor $m$ chancels)
\begin{equation}
A_s = \frac{A}{m} =  \frac{4 \pi a \mu}{\Omega f_B F \rho} \Leftrightarrow T_{1/2} = \frac{N \Omega f_B F \rho \mathrm{log}(2)}{1.12 \cdot 4 \pi m a \mu} 
= \frac{N \Omega f_B \mathrm{log}(2)}{1.12 \cdot 4 \pi m a b} = \frac{N_A h_{rel}\Omega f_B \mathrm{log}(2)}{1.12 \cdot 4 \pi a b m_{\mathrm{KCl}}}
\end{equation}
At this point we use $f_B=1.29$, $\Omega = 2 \pi$ and $\eta = [3.80 \pm 0.06]\cdot 10^{17}$ counts/g (where estimated the errors of the constants to be of the 
same order of magnitude as the last digit) to arrive at
\begin{equation}
T_{1/2} = \frac{\eta}{a b} 
\label{eq:pot_final}
\end{equation}
\subsection{samarium}
We will use the detection of $\alpha$ rays for estimating the half life period of samarium. 
The emission depends to a great deal on the surface of the emitting object.
Following \cite{ver} the rate $n$ of samarium can be calculated with
\begin{equation}
n = A_v \frac{F}{4} R_{Sm_2O_3}
\label{eq:range2}
\end{equation}
where we introduced the Activity per Volume $A_v = A/V$, the range $R_{Sm_2O_3}$ and the Surface area $F$.
Since it is not always possible to conduct experiments for all possible incident particles and materials, we
can estimate the range with the \textbf{Bragg-Kleemann rule} \cite{knoll2000radiation}, if we know
the range for another material:
\begin{equation}
    \frac{R_1}{R_0} = \frac{\rho_0 \sqrt{m_{eff,1}}}{\rho_1 \sqrt{m_{eff,0}}} 
    \label{eq:range}
\end{equation}
where we have two materials and the respective radii $R$, densities $\rho$ and atomic weights $A$. 
The effective mass molar mass $m_A$ has to be concluded from the single weights of the atoms and the relative densities of these with
\footnote{%
    We have to raise strong doubts about the academic integrity the source \cite{ver}, since no source for equation~\ref{eq:masses} was given.
    Probably it was taken from \cite{staatsexamen}, but even there we cannot find any reference or source for the formula, which even more violates our
    understanding about scientific practice: This formula is neither generally valid nor trivial. 
    After some research we found it within the book \cite{knoll2000radiation}. The main argument is the following: A \textit{linear stopping power S} for
    charged particles in a given absorber is defined by the differential energy loss for that particle within the material, divided by the corresponding
    differential path length:
    \begin{equation}
    S = - \frac{\del E}{\del x} = \frac{4 \pi e^4 z^2}{m_e v^2} N B 
    \end{equation}
    where the last step is the famous, but classical derived Bethe formula with velocitiy $v$ and charge $z$ of the particle, Number density $N$ and atomic number $Z$
    of the adsorber atoms and the electron rest mass $m_e$. The paramter $B$ is an expression obtained from integration, dependent on $v$ and average excitation and
    ionization potential of the adsorber. Now we have to implied various assumptions:
    \begin{itemize}
    \item The stopping power per atom of compounds or mixtures is additive (known as \textbf{Bragg-Kleemann rule}).
    \item The particle should be heavy and charged, like alpha particles, in order to interact with matter primarily through coulomb forces
    between their charges
    \item Interactions with nuclei can be neglected, since they happen only rearely and they are not significant in the response of radiation detectors.
    \item The charged particle immediately interacts simultaneously with many electrons, depending on the proximity either to raise the respective electron to
    a higher lying shell within the adsorber (\textit{excitation}) or remove the electron from the atom (\textit{ionization}).
    \end{itemize}
    Going from here the already mentioned \textbf{Bragg-Kleemann rule} can be stated as
    \begin{equation}
    \frac{1}{N_c}\left (\frac{\del E}{\del x} \right )_c = \sum_i P_i  \frac{1}{N_i}\left (\frac{\del E}{\del x} \right )_i.
    \label{eq:masses2}
    \end{equation}
    with the atomic fraction $p_i$ of the ith component in the compound $c$. From equation \eqref{eq:masses2} is now possible with some
    intermediate steps to arrive at \eqref{eq:masses}.
} 
\begin{equation}
\sqrt{m_{eff}} = \sum_i p_i \sqrt{m_{A,i}} .
\label{eq:masses}
\end{equation}

With this formula it is now possible to eliminate the constant $C$, given that we can find the effective mass density for another
material. For instance, air approximately consists of nitrogen, oxygen and argon: 
\begin{equation}
\sqrt{m_{\mathrm{eff,air}}} =p_{\mathrm{N}}\sqrt{m_{\mathrm{N}}} +p_{\mathrm{O}} \sqrt{m_{\mathrm{O}}} +p_{\mathrm{Ar}} \sqrt{m_{\mathrm{Ar}}}
\end{equation}
with the probabilites \cite{ver} and the atomic masses \cite{haynes} in table~\ref{tab:air}.
\begin{table}
\caption{Atomic fractions $p$ and the respective atomic masses $m$ for air.}
\centering
\begin{tabular}{ll}
    \cellcolor{tabcolor} $p_{\mathrm{N}}$    &$0.75518 \pm 0.00001$ \\
    \cellcolor{tabcolor} $p_{\mathrm{O}} $   &$0.23135 \pm 0.00001$\\
    \cellcolor{tabcolor} $p_{\mathrm{Ar}}$   &$0.01288 \pm 0.00001$
\end{tabular}
\begin{tabular}{ll} 
    \cellcolor{tabcolor} $m_{\mathrm{N}}  $ & $[14.0067 \pm 0.0002] u $ \\
    \cellcolor{tabcolor} $m_{\mathrm{O}}  $ & $ [15.9994 \pm 0.0003] u$  \\
    \cellcolor{tabcolor} $m_{\mathrm{Ar}} $ & $ [39.948 \quad \pm 0.001] u $
\end{tabular}
\label{tab:air}
\end{table}
As result we arrive at the effective mass  $m_{\mathrm{eff,air}} = [14.6926\pm0.0007]u$. \\
we have to calculate the effective mass for samarium oxide $Sm_2O_3$ in order to conclude the range:
\begin{equation}
\sqrt{m_{\mathrm{eff,Sm_2O_3}}} =p_{\mathrm{Sm}}\sqrt{m_{\mathrm{Sm}}} +p_{\mathrm{O_3}} \sqrt{m_{\mathrm{O_3}}} 
\end{equation}
with the probabilites\footnote{%
We calculated the probabilites with the approximation\cite{staatsexamen}, 
that the relative probability is proportional
to the number of atoms times the molecular mass: $p_{\mathrm{Sm_2}} = \frac{2 m_{Sm}}{2m_{Sm} + 3m_{O}}$ and
$p_{\mathrm{O_3}} = \frac{3 m_{O_3}}{2m_{Sm} + 3m_{O}}$.
}
and masses \cite{haynes} in table~\ref{tab:sm2o3} we
arrive at $m_{\mathrm{eff},Sm_2O_3}[123.763\pm0.009]u$.
\begin{table}
\centering
\caption{Atomic fractions $p$ and the respective atomic masses $m$ for $Sm_2O_3$.}
\begin{tabular}{ll}
\begin{tabular}{ll}
    \cellcolor{tabcolor}  $p_{\mathrm{Sm_2}}$       &$0.86235\pm0.00001$ \\
     \cellcolor{tabcolor} $p_{\mathrm{O_3}} $   &$0.13764\pm0.00001$ \\
\end{tabular}
&
\begin{tabular}{ll} 
    \cellcolor{tabcolor} $m_{\mathrm{Sm}}  $ & $[150.36 \pm 0.01] u $ \\
    \cellcolor{tabcolor} $m_{\mathrm{O}}  $ & $ [16.000\quad \pm 0.001] u$  \\
\end{tabular}
\end{tabular}
\label{tab:sm2o3}
\end{table}
At this point we can use equation \eqref{eq:range}, by implying the densities\footnote{%
These rely very sensitive on the pressure and Temperature, we used $T = 293^{\circ}K$ and $P = 101kPa$ 
(standard conditions) in the further calculations}\footnote{%
    The density of samarium(III)oxide is different to the
    value given in wikipedia.org and wolframalpha.com, which is $\rho_{Sm_2O_3} = [7.6 \pm 0.1] \mathrm{g/cm}^3$,
    because we used \cite{haynes}, which we consider to be a more reliable source, since the other
    two sources rely to some extent upon an older version of \cite{haynes}.
}\\
$\rho_{air} = [0.001184 \pm 0.0001] \mathrm{g/cm}^3$ and
$\rho_{Sm_2O_3} = [7.6 \pm 0.1] \mathrm{g/cm}^3$ \\ and the range of $\alpha$ rays in air
$R_{\mathrm{air}} = [1.25\pm0.2]$cm (\footnote{%
We used this value from \cite{ver}, because it was not possible to determine the value in a better
fashion, because the range of $\alpha$ rays depends strongly on their energy, which ranges in the $\alpha$ decay
from 2 - 10 MeV \cite{konya2012nuclear} (in our case, the energy is given by $E_\alpha = [2.233 \pm 0.001]MeV$
\cite{staatsexamen}).
Let us assume
we want to determine the range in air, then the range depends on the interaction of the alpha particles with
the orbital electrons of the molecules. The range of the particles is closely approximated
by \cite{cember1996introduction}:
\begin{equation}
R_{\mathrm{air}} =
\begin{cases}
0.56 E_\alpha \qquad \text{ for } E_\alpha < 4 \mathrm{MeV} \\
1.24 E_\alpha - 2.62 \qquad \text{ for } E_\alpha < 4 \mathrm{MeV} 
\end{cases}
\end{equation}
As it can be seen the ranges vary greatly, depending on the energy. Plugging in $E_\alpha$ we get
\begin{equation}
R_{\mathrm{air}} = [1.250\pm0.022] \mathrm{cm}
\end{equation}
Which varies from the value given in \cite{staatsexamen}. Since there is no source given (again we
emphasize the fact that this does not correspond to our idea of scientific integrity) we tend
to trust our calculated value than the one given.
}):
\begin{equation}
    R_{Sm_2O_3} =R_{\mathrm{air}} \frac{\rho_{\mathrm{air}\sqrt{m_{\mathrm{eff,Sm_2O_3}}}}}
    {\rho_{\mathrm{Sm_2O_3}} \sqrt{m_{\mathrm{eff,air}}} } 
    = [5.7 \pm 0.5] \mathrm{\mu m}
\end{equation}
This range seems to be very small compared to the range in air, but one has to consider the much higher density
in $Sm_2O_3$ which contributes to a huge extent to the absorption. \\
The next step is to consider the activity per volume using \eqref{eq:range2}
\begin{equation}
A_v = A \cdot V = A \cdot F \cdot d = \frac{4\cdot n \cdot d}{R_{Sm_2O_3} }.
\end{equation}
with the sample width $d$ and the surface $F$. From here the half life period directly follows with
\begin{equation}
T_{1/2} = \frac{\mathrm{log}(2)N}{A} = \frac{log(2) N R_{Sm_2O_3}F}{4 n \cdot V}.
\label{eq:T12_sam}
\end{equation}

This expression only leaves to estimate the number of $^{147}Sm$ atoms $N$ with
\begin{equation}
    N = p_{\mathrm{^{147}Sm}} \cdot N_{\mathrm{Sm_2O_3}} 
    \qquad \text{ with \cite{haynes} } p_{\mathrm{^{147}Sm}}= [0.1499\pm0.0033]
\end{equation}
(since for the radiation we only
rely on the atoms which are decaying into neodym), hence
\begin{equation}
    N = 2 \cdot p_{\mathrm{^{147}Sm}} \cdot N_{\mathrm{Sm_2O_3}} 
    = \frac{2 \cdot p_{\mathrm{^{147}Sm}} \cdot N_A m}{m_{\mathrm{Sm_2O_3, mol}}}
    = \frac{2 \cdot p_{\mathrm{^{147}Sm}} \cdot N_A \rho_{\mathrm{Sm_2O_3}} V}{m_{\mathrm{Sm_2O_3, mol}}}
    \label{eq:N}
\end{equation}
with $m_{\mathrm{Sm_2O_3, mol}} = 348.72 \mathrm{g/mol} $ \cite{haynes}. 
Plugging \eqref{eq:N} into \eqref{eq:T12_sam} we arrive at
\begin{equation}
T_{1/2} = \frac{\mathrm{log}(2)p_{\mathrm{^{147}Sm}} 
 N_A \rho_{\mathrm{Sm_2O_3}}R_{Sm_2O_3}F}{2m_{\mathrm{Sm_2O_3, mol}}n } = \zeta \frac{F}{n} 
\label{eq:T12_sam}
\end{equation}
with $\zeta=[2.22\pm0.19]\cdot10^{22} \mathrm{counts \cdot cm^{-2}}$.

