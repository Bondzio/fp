\section{Conclusion}
\label{sec:conclusion}
For the half life period of samarium-147 we derived
\begin{align}
T_{1/2} &= [1.15\pm0.17]10^{11}\mathrm{a} \qquad \text{\textbf{measurement 2.2.2}}\\ 
T_{1/2} &= [1.12\pm0.17]10^{11}\mathrm{a} \qquad \text{\textbf{measurement 2.2.3}}\\
\bar{T}_{1/2} &= [1.14\pm0.16]10^{11}\mathrm{a} \qquad \text{\textbf{both}}
\end{align}
which is less than one sigma bigger than the literature value $T_{1,2} =[1.06\pm0.2]\cdot10^{11} \mathrm{a}$.
We consider the measurement hence to be reasonable as far as our result is agreeing very good with the theory.\\\\

Our derived value for the half life period of potassium is
\begin{equation*}
T_{1/2} =  [3.82\pm0.13]\cdot 10^{16} s = [1.21 \pm 0.04]\cdot 10^9 a
\end{equation*}
The literature value is $T_{1/2} = [1.248 \pm 0.003]\cdot10^9$a, which is within the range of our result. \\
We would therefore consider this experiment to be sucessfull within the limitations given by the experimental
setup. 
