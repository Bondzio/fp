\section{\textsc{Aufgabenstellung}}

\pagenumbering{arabic}
\subsection*{Vermessung der Signale mit dem Oszilloskop}
Zuerst wird eines der beiden Präparate ( $^{57}Co$ oder $^{241}Am$) vermessen. Hier bei wird eine Skizze der Signale von Hand angefertigt und die Anstiegszeiten (zwischen 10 \% und 90 \% des Maximalwertes), die Abfallzeiten (90\% $\rightarrow$ 10\%) und die Signal-Amplitude vermessen.
\subsection*{Aufnehmen der Energie-Spektren}
Mit dem Multi Channel Analyser (MCA) werden nun die Energie-Spektren von $^{57}Co$ und $^{241}Am$ aufgenommen. Dabei wird das Spektrum des $^{57}Co$ mit beiden Detektoren für beide möglichen Orientierungen der Quelle gemessen. Aus dem Ergebnis dieser Messung kann nun bestimmt werden mit welchem Detektor im weiteren Versuchsverlauf für den 14,4keV-Peak oder den 122keV-Peak gemessen wird.
In der Auswertung soll eine Energie-Kalibration für den gewählten Detektor mit den charakteristischen Peaks von beiden Quellen durchgeführt werden und die Spektren der beiden Präparate anhand der Ergebnisse interpretiert werden.
\subsection*{Setzen der Energiefenster}
Die Energiefenster werden an den beiden Single Channel Analyser (SCA) jeweils für ein Signal eines 14,4keV Photons bzw. eines 122keV Photons eingestellt.
\subsection*{Messung der verzögerten Koinzidenzen}
Nun wird das Spektrum der verzögerten Koinzidenzen vermessen. Dabei gibt das 14,4keV Photon das Startsignal und das verzögerte 122keV Photon das Stoppsignal.
Ausgewertet wird einmal durch einen Fit einer Exponentialfunktion an das linear aufgetragene Spektrum und einmal durch den Fit einer linearen Funktion an das logarithmische Spektrum. Davor muss das Spektrum der zufälligen Koinzidenzen abgezogen werden, welches man durch eine Messung bestimmt bei dem das Start- und Stoppsignal vertauscht wird.
\subsection*{Zeitkalibration des TAC}
Der Time to Amplitude Converter (TAC) wird kalibriert indem man die Pulse eines Single Channel Analysers (SCA) aufteilt und eines der Signal über die Delay-Boxen verzögert. Das unverzögerte Signal wird an den Start-Eingang angeschlossen und das verzögerte an den Stopp-Eingang. Man variiert nun die Delays und nimmt zum jeweiligen Delay den jeweiligen Kanal auf welcher angesprochen wird auf.