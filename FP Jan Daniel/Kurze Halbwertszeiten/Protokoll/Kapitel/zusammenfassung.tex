\clearpage
\section{Zusammenfassung}
In diesem Versuch sollte die Halbwertszeit des 14,4 keV-Zustandes von $^{57}Fe$ bestimmt werden. Hierfür vermaßen wir zunächst die Energiespektren von $^{57}Co$ und $^{241}Am$, um mithilfe der bekannten Zerfallsenergien dieser Stoffe eine Energiekalibrierung für den Multi Channel Analyzer durchführen zu können. Für ein anderes verwendetes Gerät, den Time-to-Amplitude Converter, wurde eine Zeitkalibrierung durchgeführt.\\
Um die Halbwertszeit zu bestimmen, benutzten wir die Messung der verzögerten Koinzidenzen, welche einen Untergrund durch zufällige Koinzidenzen hatte: dieser wurde bereinigt. Wir erlangten schließlich die Halbwertszeit, indem wir die erhaltenen Werte auf zwei verschiedene Arten fitteten: Bei der Verwendung einer linearen y-Achse verwendeten wir einen exponentiellen, bei einer logarithmischen einen linearen Fit. Die resultierenden mittleren Lebensdauern und Halbwertszeiten ergaben sich somit zu:\\
$\tau_{exp}=(60\pm10)ns$\\
$T_{(1/2)_{exp}}=(41\pm7)ns$\\
~\\
$\tau_{log}=(39,5\pm1,9)ns$\\
$T_{(1/2)_{log}}=(27,4\pm1,3)ns$\\
~\\
Die Literaturwerte liegen bei $\tau=(140,2\pm0,3)ns$ (Quelle:[REB]) beziehungsweise $T_{1/2}=98 ns$ (Quelle: [Ver]). Die von uns gestimmten Werte sind deutlich von diesen entfernt. Ein Erklärungsversuch hierfür wurde in dem Kapitel \glqq Auswertung\grqq ~ vorgenommen.

