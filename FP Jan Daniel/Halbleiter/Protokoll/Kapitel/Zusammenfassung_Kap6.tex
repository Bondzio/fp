\chapter{Summary}
\section{Measuring the Bandgap}
In experiment no. 1 we determined the bandgap energy of Si and Ge respectively. Our results are:\\
\begin{center}
$E_{Si}=(1,087\pm 0,018)$eV ; literature value: 1,12 eV (source: [Ver])\\
$E_{Ge}=(0,64\pm0,03)$eV; literature value: 0,67 eV (source: [Ver])\\
\end{center}
Both these values are, especially regarding the fact that the errors are certainly too small (see analysis for further information) good. \\
\section{Haynes-Shockley-experiment}
This experiment consisted of two different approaches: in the first part we modified the distance, in the second the voltage applied.\\
The results were the following:\\
First part:\\
\begin{center}
Mobility $\mu=(2420\pm 70)\frac{cm^2}{Vs}$\\
Mean lifetime $\tau=(2,4\pm 0,1)*10^-6 s$\\
Diffusion constant $D_{n}=(5\pm1)*10^-8 \frac{mm^2}{s}$
\end{center}
Second part:\\
\begin{center}
Mobility $\mu=(2600\pm 400) \frac{cm^2}{Vs}$\\
Mean lifetime $\tau=(2,0\pm 0,6)*10^-6 s$\\
Diffusion constant $D_{n}=(0,3\pm9,1)*10^-8 \frac{mm^2}{s}$
\end{center}
~\\
The literature values are the following:\\
\begin{center}
Mobility $\mu=3900 \frac{cm^2}{Vs}$\\
Mean lifetime $\tau=(45\pm 2)*10^-6 s$\\
Diffusion constant $D_{n}=101 \frac{cm^2}{s}$
\end{center}
~\\
The values for $\mu$ are in the same order of magnitude as the literature values. This does not hold for $\tau$ and especially not for the diffusion constant $D_{n}$. Further explanations and possible reasons can be found in the analysis of this part of the experiment.\\
~\\
\section{Semiconductor detectors}
In this section, our task was to measure the spectra of two radioactive sources (Am-241 and Co-57) with the semiconductor detectors Si and CdTe respectively.\\
The results for the ratio of the absorption probabilities of Si and CdTe for different energies were the following:\\
\begin{table}[htbp]
  \centering
  \caption{Results}
    \begin{tabular}{rrrr}
    Energy in keV & Ratio R & Error s\_R & Literature value \\
    59,5  & 0,031 & 0,002 & 0,014 \\
    122,06 & 0,00023 & 0,00006 & 0,0183 \\
    136,47 & 0,000013 & 0,000009 & 0,02 \\
    \end{tabular}%
  \label{tab:addlabel}%
\end{table}%\\
~\\
As already discussed, we might have had some kind of trouble with the setup. \\
~\\
Our second task was to calculate the energy resolutions of the semiconductor detectors:\\
~\\
  \caption{Energy resolutions}
      \begin{tabular}{rrrr}
      Detector & Energy in keV & RER[E]  & s\_RER[E] \\
      Si    & 59,5  & 0,121 & 0,003 \\
            & 122,06 & 0,068 & 0,007 \\
            & 136,47 & 0,05229 & 0,00005 \\
      CdTe  & 59,5  & 0,118  & 0,006 \\
            & 122,06 & 0,060 & 0,005 \\
            & 136,47 & 0,07  & 0,04 \\
      \end{tabular}%
    \label{tab:addlabela}%
~\\
  The expected general tendency could at least for Si be shown. For CdTe it's obvious that the value for 59,5 keV is - as expected - higher as those for the peaks of Co-57. Unfortunately it's difficult to distinguish between the peaks of Co-57.

\newpage
\chapter{Bibliography}
~\\
[Amr] Simon Amrein, Staatsexamensarbeit: "Halbleiter und Halbleiterdetektoren", 2008\\
[Ver] Versuchsanleitung, Fortgeschrittenen-Praktikum für Physiker,Uni-Freiburg\\
[Sclo] http://www.scilogs.de/wblogs/gallery/59/previews-med/720px-Isolator-metal.png\\