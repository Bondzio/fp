\section{Durchführung}
Zunächst wurde die Polarisatoreinstellung von $90^{\circ}$ gesucht. Mithilfe der so gefundenen Einstellungen stellten wir nun einen Winkel von $0^{\circ}$ ein. Dann wurden die kompensierenden Magnetfelder kalibriert (das Signal minimiert), was sich als schwierig erwies, da es immer wieder zu Schwankungen in der Intensität kam, was auf Störfelder im Raum schließen lässt. Die Peltier-Elemente waren die gesamte Zeit bis zum Ende der Kalibrierung in Betrieb und kühlten die Resonanzzelle auf $-11^{\circ}C$ herunter. \\
Nachdem die fertigen Einstellungen getroffen waren, wurden die Peltier-Elemente ausgeschaltet und eine Messreihe für Polarisatoreinstellungen von $0^{\circ}, 45^{\circ}$ und $90^{\circ}$ für den Bereich von $-10^{\circ}C$ bis $13^{\circ}C$ aufgenommen, wobei die Schrittweite $1^{\circ}C$ betrug, da die Anzeige des digitalen Thermometers keine feinere Aufteilung zuließ. \\
Anschließend wurden die Peltier-Elemente angeschaltet, wobei unterschiedliche Spannungen angelegt und ein Einpendeln der Temperatur abgewartet wurde, um sinnvolle Ergebnisse zu erhalten: Dann wurden Kurven für die Polarisatoreinstellungen von $0^{\circ}$ und $90^{\circ}$ aufgenommen.