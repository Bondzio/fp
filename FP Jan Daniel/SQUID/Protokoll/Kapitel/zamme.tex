\section{Zusammenfassung}
Die magnetischen Felder und Dipolmomente wurden für unterschiedliche Widerstände berechnet und mithilfe eines SQUID-Sensors vermessen. Die erhaltenen Ergebnisse sind folgende:
\begin{table}[htbp]
\caption{}
\begin{center}
\begin{tabular}{|l|p{3.5cm}|p{3.5cm}|p{3.5cm}|p{3.5cm}|}
\hline
Nr & $p_{m}$ in C*nm mit SQUID & $p_{m}$ in C*nm theoretisch & $B_{z}$ in nT mit SQUID & $B_{z}$ in nT theoretisch \\ \hline
R1 & $900\pm200$ & $590\pm30$ & $4,76\pm0,03$ & $3,3\pm0,9$ \\ 
R2 & $450\pm120$ & $305\pm16$ & $2,480\pm0,008$ & $1,7\pm0,5$ \\ 
R3 & $80\pm20$ & $106\pm6$ & $0,454\pm0,006$ & $0,59\pm0,16$ \\ 
R4 & $52\pm14$ & $63\pm3$ & $0,290\pm0,006$ & $0,4\pm0,1$ \\ 
R5 & $23\pm6$ & $32,6\pm1,7$ & $0,126\pm0,006$ & $0,18\pm0,05$ \\ \hline
\end{tabular}
\end{center}
\label{}
\end{table}
~\\
Es ist zu sehen, dass sich die berechneten und die mit dem SQUID-Sensor gemessenen Werte jeweils im Rahmen von ein bis zwei Standardabweichungen entsprechen. Eine detailliertere Diskussion ist dem Unterkapitel 'Vergleich' des Kapitels 'Auswertung' zu entnehmen. \\
~\\
Für die einzige vermessene Probe erhielten wir folgende Werte für B-Feld und Dipolmoment:
\begin{table}[htbp]
\caption{}
\begin{center}
\begin{tabular}{|l|l|l|l|}
\hline
Probe & $\Delta U$ in V & $p_{m}$ in C*nm & $B_{z}$ in nT \\ \hline
Magnetospan & $3,703\pm0,007$ & $52000\pm14000$ & $287,0\pm0,6$ \\ \hline
\end{tabular}
\end{center}
\label{}
\end{table}.