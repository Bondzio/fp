\clearpage
\section{Conclusion}
\subsection{Homogeneity measurement of the magnetic field}
The magnetic field was found to have a zone in which the magnetic field was homogeneous with $B=(419,0\pm0,8)mT$.
\subsection{Nuclear magnetic moment of $^{19}F$}
For the nuclear magnetic moment, we got a result of $\mu_{K}=(5,05\pm0,01)\cdot 10^{-27}\frac{J}{T}$, which equals within its limits of accuracy the literature value of $\mu_{K,Lit}=5,05\cdot10^{-27} \frac{J}{T}$.
\subsection{Determination of the gyromagnetic ratios}
\subsubsection{Hydrogen}
For hydrogen, the gyromagnetic ratio was determined to be $\gamma_{H}=(2,643\pm0,005)\cdot10^{8} ~s^{-1}\cdot T^{-1}$
\subsubsection{Glycol}
For glycol, the gyromagnetic ratio was determined to be $\gamma_{glycol}=(2,647\pm0,005)\cdot 10^{8} ~s^{-1}\cdot T^{-1}$.
\subsection{Determination of the Proton Resonance Frequency Using the Lock-in Method}
In this part of the experiment, the hydrogen sample was used once again.\\
With this method, the proton resonance frequency was determined to be $\nu=(17,623\pm0,007)MHz$. \\
Using this result, we got $\gamma=(2,643\pm0,005)\cdot 10^{8} ~s^{-1}\cdot T^{-1}$ for the gyromagnetic ratio.
\subsection{Discussion of the results}
The gyromagnetic ratios we determined for hydrogen with both methods and for glycol are equal within their limits of accuracy. This means, that our measurements are consistent. Still, they do not equal within their limits of accuracy the literature value $\gamma_{H,Lit}\approx\gamma_{Gl,Lit}=2,675\cdot10^{8} ~s^{-1}\cdot T^{-1}$ (source: [NIST]; theoretically, the values should be the same, but because of the different structure there is a minor difference, which is not of any interest in this experiment, as we cannot measure that accurately). Because of this, most probably there was a systematic error in this experiment. It is possible, that there were impurifications in both samples, which is unlikely as we used two different samples and get the same value within the limits of accuracy. Another possibility would be a weakening of the B-field which we doubt as we checked it between all parts of the measurement and didn't notice any change. As we touched the samples our fingerprints might have been the cause for a minor spin, causing a systematic error. This could be a possible reason as we touched both samples and the values we received are consistent. 