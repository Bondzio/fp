\section{Introduction}
The goal of this experiment is to research the nuclear magnetic resonance (NMR) at a proton. To do this, we had to check the homogeneity of a permanent magnetic field with a Hall sensor. Afterwards we were to determine the gyromagnetic ratio $\gamma$ of a proton in hydrogen and glycol respectively. Also, we had to determine the nuclear magneton and the magnetic moment of the $^{19}F-atom$ in Teflon.\\
\section{Theoretical Background}
As our task was to determine nuclear characteristics we put different samples (Teflon, hydrogen and glycol) into a high frequency radiation field located in a homogeneous magnetic field. The nuclear Zeeman-effect, creating energy differences between two spin stances because of the interaction of the nuclear magnetic moment with an external magnetic field, was utilized. When the radiaton field's frequency reaches the resonance frequency, the radiation field loses energy because of absorption.
\subsection{Nuclear Spin}
The spin is an intrinsic property of a quantum-mechanical particle. One can only determine its absolute value and the projection on a particular axis, just as in the case of the angular momentum. The following equations hold:\\
$\left\vert{\vec{S}}\right\vert=\hbar\cdot\sqrt{S\cdot(S+1)}$, where $S=0,1/2,1,...$ and $\hbar$ is the reduced Planck constant.\\
The projection on an axis leads to the following:\\
$S_{p}=m_{S}\cdot\hbar$,where $-S\leq m_{S} \leq S$\\
As the core is made up of such particles, it has a spin, which we refer to as:\\
$\left\vert{\vec{I}}\right\vert=\hbar\cdot\sqrt{I\cdot(I+1)}$, where $I=0,1/2,1,...$ and \\
$I_{p}=m_{I}\cdot\hbar$,where $-I\leq m_{I} \leq I$\\
This makes clear that the spin can have $2I+1$ possible stances.
\subsection{Magnetic Moment}
The spin of a quantum-mechanical particle causes a magnetic dipole moment interacting with external magnetic fields. It is\\
$\vec{\mu}=\gamma\cdot\vec{I}$, where $\gamma=\frac{g_{I}\cdot\mu_{K}}{\hbar}$ and $\mu_{K}=\frac{e\cdot\hbar}{2\cdot m_{p}}$.\\
The variables in the formulas stand for the following:\\
$\mu$: magnetic dipole moment\\
$\gamma$: gyromagnetic ratio\\
$g_{I}$: nuclear g-factor\\
$e$: elementar charge\\
$\mu_{K}$: nuclear magneton\\
$m_{p}$: proton mass\\
~\\
Because of the so-called 'Pauli exclusion principle', every orbital is occupied by two protons and neutrons with contrary spin respectively. Thus, the core does only have a spin if there is an odd number of either protons or neutrons or both.\\
ee: $I=0$\\
oo: $I=1$\\
eo,oe: $I=\frac{1}{2}$\\
The first letter stands for the number of protons, the second for the number of neutrons. The letter 'e' stands for an even number, 'o' for an odd number.\\
Oxygen and carbon are 'ee'-cores, which means that they don't have a spin. Because of this, in the experiment for both glycol ($C_{2}H_{6}O_{2}$) and water ($H_{2}O$) a resulting spin of $I=\frac{1}{2}$ for the hydrogen atom is expected. With 9 protons and 10 neutrons, $^{19}F$ has a resulting spin of $I=\frac{1}{2}$ as well.\\
\subsection{Interaction with Magnetic and Radiation Fields (Nuclear Magnetic Resonance)}
A magnetic dipole moment $\vec{\mu}$ in an external magnetic field has the energy \\
\[E=-\vec{\mu}\cdot\vec{B}\] which, if again projected on a particular axis, leads to \[E_{p}=-g_{I}\cdot\mu_{K}\cdot m_{I}\cdot B\]
This leads to the creation of several energy levels depending on $m_{I}$ (Zeeman-effect), where \[\Delta E=g_{I}\cdot\mu_{K}\cdot B\] is the energy difference of two neighboring energy levels.\\
This is the energy needed or set free when a spin-flip takes place. This energy can be provided by a radiation field (via photons) with resonance frequency 
\[\nu=\frac{\Delta E}{h}=\frac{g_{I}\cdot\mu_{K}\cdot B}{h}=\frac{\gamma\cdot B}{2\pi}\]
When in thermal equilibrium, the population of two neighboring levels (where $E_{high}> E_{low}$)  is Boltzmann-distributed:\\
\[\frac{n_{high}}{n_{low}}=e^{-\frac{E_{high}-E_{low}}{kT}}=e^{-\frac{\Delta E}{kT}}\] where k is the Boltzmann-constant and T is the temperature.\\
There are more particles on the lower level, as can be seen in this formula.
\subsection{Relaxation Processes}
One would expect the population to be equally distributed after some time. However, there are relaxation processes causing a radiationless fall-back to the lower level. This is the reason why the Boltzmann distribution is maintained. The processes are the following:\\
\subsubsection{Spin-Lattice Relaxation}
Excited cores provide the molecule lattice with thermal energy.
\subsubsection{Spin-Spin Relaxation}
Two particles' magnetic moments interact and thereby cause a line broadening and a small shift of the resonance frequency.
\subsection{Hall Effect and Hall Sensor}
An external magnetic field distracts moving charge in a conductor. The resulting force is given by \[\vec{F_{L}}=e\cdot(\vec{v}\times \vec{B})\] with $\vec{v}$ being the electron drift velocity. Due to this force a separation of charge is taking place, leading to the generation of an electric field. After some time, the force caused by the electric field equals the Lorentz force. As it is $v\perp B$, this leads to 
\[F_{L}=F_{E}\]
\[\leftrightarrow e\cdot v\cdot B=e\cdot E\]
\[\leftrightarrow v\cdot B=\frac{U_{H}}{d}\]
\[\leftrightarrow U_{H}=v\cdot B\cdot d=\frac{I}{n\cdot e}\]
where it is\\
$U_{H}$: Hall voltage\\
d: conductor width\\
n: charge carrier density\\
$I=n\cdot e\cdot v\cdot B\cdot d$ is the current flowing through the conductor.
For Hall sensors often semiconductors are used as those have a much smaller charge carrier density than conductors, making the Hall voltage much larger.\\
\subsection{Lock-In Method}
This method is used as it is capable of visualizing weak signals with a large background noise. The signal is multiplied with a reference signal in the synchron detector, which has a frequency close to the one of the expected signal. To improve our result for the resonance frequency, the magnetic field gets modulated with a saw tooth voltage superimposed by a sine voltage. This way, we get the differentiated absorption curve, in which - for us - the position of the zero is of interest, as that's the position where the absorption curve has its minimum.