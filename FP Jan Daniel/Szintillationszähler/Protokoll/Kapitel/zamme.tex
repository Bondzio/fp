\clearpage
\section{Zusammenfassung}
\subsection{Eichung}
Aus den Peaks der Messungen mit ${22}Na$, $^{60}Co$ und $^{152}Eu$ und den bekannten Energien für die Peaks konnte eine Eichung der Kanäle des Multi Channel Analyzers vorgenommen werden, mithilfe derer eine Umrechnung von Kanälen auf die Energie erfolgen kann: 
\[E(C)=(0,1719\pm0,0005)\frac{keV}{Kanal}\cdot C+(-21\pm3)keV.\]
Die gemessenen Werte werden sehr gut von dem linearen Fit beschrieben.
\subsection{Untergrundspektrum}
Im Untergrundspektrum war ein Peak der Energie $E=(1461\pm5)$ keV zu sehen, welcher im Rahmen der Messungenauigkeit dieselbe Energie mit dem als Nebeneffekt bei dem Zerfall von $^{40}K$ in $^{40}Ar$ emittierten $\gamma$-Quanten teilt, was die Vermutung nahelegt, dass es sich um diese handelt.
\subsection{Analyse des $^{228}Th$-Spektrums}
Nach der Vermessung des Thoriumspektrums wurden die Peaks gefittet und daraus die Energien der Peaks berechnet. Wir erhielten 10 Peaks  
und ordneten sie unterschiedlichen Zerfallsenergien aus der Zerfallskette von $^{228}Th$ zu. Manche Energien liegen abseits der Literaturwerte, andere liegen sehr nahe an den Literaturwerten. Für eine ausführliche Diskussion siehe die Auswertung.\\
Danach wurden die Intensitäten berechnet und mit der Photopeak-Effizienz des verwendeten NaJ-Kristalls korrigiert. Um einen Vergleich mit den Literaturwerten der zugeordneten Peaks zu haben, wurde der höchste Peak auf 100 normiert:
\begin{table}[htbp]
\begin{center}
\caption{}
\begin{tabular}{|l|l|r|}
\hline
 & $I_{norm}$ & \multicolumn{1}{l|}{Literaturwert} \\ \hline
Peak 1 & $100\pm8$ & 100 \\ \hline
Peak 2 & $41\pm3$ & \multicolumn{1}{l|}{10,7 bzw. 8,49} \\ \hline
Peak 3 & $64\pm16$ & \multicolumn{1}{l|}{10 bzw. 0,41} \\ \hline
Peak 4 & $94\pm13$ & 0,636 \\ \hline
Peak 5 & $64\pm3$ & 0,757 \\ \hline
Peak 6 & $41\pm4$ & 0,101 \\ \hline
Peak 7 & $8,7\pm0,5$ & 2,28 \\ \hline
Peak 8 & $8\pm5$ & 8,52 \\ \hline
Peak 9 & $15,0\pm0,8$ & 0,0002 \\ \hline
Peak 10 & $3,3\pm0,5$ & \multicolumn{1}{l|}{10 bzw. 0,26} \\ \hline
\end{tabular}
\end{center}
\label{}
\end{table}
~\\
Außer Peak 1 (per Konstruktion) und Peak 8 liegen alle Werte fernab des Literaturwerts. Mögliche Gründe hierfür sind im Kapitel 'Auswertung' beschrieben.
\subsection{Winkelkorrelation von Vernichtungsphotonen}
Für die Vermessung der Vernichtungsphotonen erhielten wir eine Gauss-Verteilung um $\Phi_{0}=(-1,4\pm0,2)^{\circ}$. Somit konnte im Rahmen der Messungenauigkeit nicht bestätigt werden, dass die Vernichtungsphotonen in einem Winkel von $180^{\circ}$ (Einstellung von $\Phi_{0}=0^{\circ}$) abgesondert werden. Mögliche Gründe hierfür werden in dem Kapitel 'Auswertung' diskutiert.

