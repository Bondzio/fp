\clearpage
\section{Zusammenfassung}
\subsection{Aufgabe 1}
Für die Gitterkonstante des Sinusgitters erhielten wir einen Wert von \[K=(1,01\pm0,03)\cdot 10^{-6}m.\]
\subsection{Aufgabe 2}
Die Gitterkonstanten der 5 verwendeten Gitter sind unten aufgelistet. Man sieht, dass der erhaltene Wert für das Gitter PHYWE 08540 im Rahmen der Messungenauigkeit mit der Herstellerangabe übereinstimmt:
\[phywe 08540: K=(100 \pm 37) \mu m ~~ (K_{Lit}=100 \mu m)\]
\begin{center}
\begin{tabular}{ll}
Gitter 1: & $K_1=(127 \pm 1)\mu m$\\
Gitter 2: & $K_2=(33.8 \pm 0.7) \mu m$\\
Gitter 3: & $K_3=(101.2 \pm 1.5) \mu m$\\
Gitter 4: & $K_4=(101.2 \pm 1.5) \mu m$\\
Gitter 5: & $K_5=(50.8 \pm 0.9) \mu m$.\\
\end{tabular}
\end{center}
Für das Auflösungsvermögen der Gitter erhielten wir durch Ausmessen des Laserstrahls und mithilfe der bekannten Gitterkonstanten:
\begin{center}
\begin{tabular}{ll}
Gitter 1: & $a_1=137 \pm 20$\\
Gitter 2: & $a_2=104 \pm 15$\\
Gitter 3: & $a_3=173 \pm 25$\\
Gitter 4: & $a_4=173 \pm 25$\\
Gitter 5: & $a_5=138 \pm 20$\\
\end{tabular}
\end{center}
\subsection{Aufgabe 3 und 4}
Mithilfe der gemessenen Amplituden der Beugungsmaxima wurde für Gitter 1 die Aperturfunktion aufgestellt und geplottet. Aus dieser ließ sich dann das Verhältnis der Spaltbreite zur Gitterkonstanten für dieses Gitter bestimmen: \[c=()\]
\subsection{Aufgabe 5}
Die von der Raman-Nath-Theorie vorhergesagte Proportionalität zwischen Intensitätsverlauf einer Ordnung und dem Quadrat der Bessel-Funktionen derselben Ordnung konnte für die ersten Werte der Messreihe bestätigt werden. Betrachtet man jedoch alle Werte, so ergeben sich Diskrepanzen, die im Teil $ Auswertung$ diskutiert werden.\\
Die Schallfrequenz in Isooktan berechnete sich nach unserer Messung zu \[f=(2,15\pm0,12) MHz.\] Dieser Wert stimmt im Rahmen der Messungenauigkeit mit der eingestellten Frequenz von $f=2,1686 MHz$ überein.