\clearpage
\section{Zusammenfassung}
\subsection{Pockels-Effekt}
In diesem Teil des Versuchs bestimmten wir den elektrooptischen Koeffizienten $r_{41}$ von ADP mithilfe des Pockels-Effekts. Alle hierfür benötigten Werte sind in der Versuchsanleitung gegeben, nur die Halbwellenspannung $U_{\lambda/2}$ musste noch bestimmt werden. Wir taten dies mit zwei unterschiedlichen Methoden und erhielten folgende Werte:\\
\begin{center}
$r_{41_{Sägezahn}}=(23\pm3)\frac{pm}{V}$\\
$r_{41_{Sinus}}=(22,41\pm0,04)\frac{pm}{V}$\\
\end{center}
Der Literaturwert beträgt:\\
$r_{41_{Lit}}=23,4\frac{pm}{V}$ (Quelle: [Ver])
Der mithilfe der Sägezahnmethode bestimmte Wert stimmt im Rahmen einer Standardabweichung mit dem Literaturwert überein. Der mithilfe der Sinusmethode bestimmte Wert ist zwar in der Größenordnung recht nahe am Literaturwert, allerdings ist der Fehler sehr klein. Eine detaillierte Diskussion wurde im Unterkapitel "Pockels-Effekt/Linearer elektrooptischer Effekt" der Auswertung vorgenommen.\\
\subsection{Faraday-Effekt}
Wir nutzten den Faraday-Effekt, um die Verdet-Konstante von Schwerflintglas zu bestimmen. Dabei erhielten wir:\\
\begin{center}
$V=(-0,04848\pm0,00008)\frac{Min}{Oe*cm}$\\
\end{center}
Die Herstellerangabe lautet $V=-0,05 \frac{Min}{Oe*cm}$. Wie man sieht, ist in der Größenordnung der Wert nahe an dieser, allerdings ist der Fehler sehr klein. Eine Diskussion dieses Sachverhalts ist im Unterkapitel "Faraday-Effekt/Magnetooptischer Effekt" der Auswertung zu finden.\\
Für den Winkel zwischen den Polarisationsebenen erhielten wir $2\epsilon=(164,77\pm0,05)\,^\circ $