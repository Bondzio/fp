\newpage
\section{Ziel des Versuchs}
Bei bestimmten Materialen kann man beim Anlegen eines elektrischen oder magnetischen Feldes Doppelbrechung beobachten. Im Falle des elektrischen Feldes spricht man von elektrooptischen Effekt und unterscheidet hier zwischen einer linearen (Pockels) und einer quadratischen (Kerr) Abhängigkeit. Im zweiten Fall spricht man vom magnetooptischen oder Faraday-Effekt. Ziel des Versuches ist es das Tensorelements, welches die Brechung in Ammoniumdihydrogenphosphat (ADP) beschreibt beim Pockelseffekt und die Verdetkonstante von Schwerflintglas mit dem Faraday-Effekt zu bestimmen.
~\\
~\\
~\\
~\\
~\\
~\\
\section{Aufgabenstellung}
\begin{itemize}
\item Mit der Hilfe des Pockels-Effekts wird der elektrooptische Koeffizient $r_{41}$ von ADP bestimmt. Die zur Berechnung benötigte Halbwellenspannung wird hierfür auf zwei verschieden Arten bestimmt:
\begin{itemize}
\item Sägezahnmethode
\item Modulierte Gleichspannung
\end{itemize}
\item Bestimmung der Verdetkonstante von Schwerflintglas bei Licht einer Wellenlänge von $\lambda = 589 nm $ mithilfe des Faraday-Effekts, indem der Ablenkwinkel bei unterschiedlicher Magnetfeldstärke, von linear polarisiertem Licht, gemessen wird.
\end{itemize}