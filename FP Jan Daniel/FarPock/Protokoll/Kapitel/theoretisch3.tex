\subsection{Magnetooptischer Effekt/ Faraday-Effekt}
Passiert linear polarisiertes Licht längs eines homogenen Magnetfeldes ein Medium, so kann man beobachten, dass sich die Polarisierungsebene verdreht. Der Drehwinkel ist proportional zur magnetischen Feldstärke $H$ und zur Länge $l$ welche das Licht im Material zurücklegt. Wir erhalten somit:
\[ \alpha = V \cdot H \cdot l \]
Der Proportionalitätsfaktor $V$ ist nicht von $H$ oder $l$ abhängig. $V$ wird Verdetkonstante genannt und ist abhängig vom Material des Mediums und der Wellenlänge des Lichts.
\subsection{Halbschattenpolarimeter}
Ein Halbschattenpolarimeter ist ein Aufbau aus eine Polarisator, einem Analysator und einem Prisma. Nach dem Polarisator rotiert ein Prisma einen Teilbereich des einfallenden Lichtes, so dass zwei Lichstrahlen mit einem Polarisationsunterschied von einem Winkel $\epsilon$ entsteht. Zwischen Prisma und Analysator wird eine Probe platziert. Am Analysator kann man nun zwei verschieden helle Flächen sehen. Um nun herauszufinden wie die Probe die Polarisationsebene gedreht hat muss man den Analysator so drehen das nur noch eine gleich helle Fläche zu sehen ist.