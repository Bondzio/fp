\section{Conclusion}
We summarize this report by reviewing the conduction of the experiment and restating the central results.

The most difficult part while experimenting was the calibration: Due to low activity of the sample, it was especially 
challenging to set the delays for coincidence optimally. The low rates further demanded long measurement times which in 
turn exposed us strongly to the effects of unstable electronics. For a large part of our measurements the measured rates 
where in the same order of magnitude as the background and random coincidences. 

The calibration showed consistency with the expected literature values even though the Compton edges were hardly visible 
and fraught with errors. 

Fits on the peaks of photon and electron energies yielded acceptable results only for angles larger than $45^\circ$. 
For lower angles, we assessed the position of the maximum by taking the maximum of filtered data and estimating the error. 
Using these results, we were able to confirm the conservation of energy within the given uncertainties. Moreover, 
the distributions of scattered wavelength and electron energies followed the expected behavior. 

The more complicated analysis of the differential cross sections theoretically described by the Klein-Nishina formula was 
subject to a large number of uncertainties. However, after correcting the measured values with the different factors, 
the result  was in good agreement with the predictions.  

Despite all difficulties, we consider the experiment a success. We got a good overview over the significance of the Compton
effect in the development of the theory. Experimentally, we learned a considerable amount about analog signal processing 
and the setup of a coincidence circuits. Finally, we are positively surprised by the good agreement between experiment and 
theory. 
