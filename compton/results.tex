\clearpage
\section{Analysis and results}
\label{sec:analysis}
\subsection{Techniques used for the evaluation}
All calculations in this section are done with scripts written in 
the \textit{python} programming language~\cite{python}, relaying in several 
packages:
\begin{itemize}
    \item
        \textit{matplotlib}~\cite{Hunter2007} for plotting,
    \item
        \textit{scipy}~\cite{scipy} for fitting, and 
    \item
        \textit{uncertainties}~\cite{uc} for error propagation.
\end{itemize}
The latter applies Gaussian error propagation for correlated and uncorrelated variables. 
We will thus not explicitly write down the formulas for the error propagation 
for each quantity calculated but instead state the numerical result, only. 
We will, however make a quick remark on the use of covariance matrices in 
error propagation: Contrary to measured data, which in our case is usually 
expected to be uncorrelated, all fitted data yields variables that in general correlate. 
The propagation is then done as follows:
Let's assume we have random
variables $x_0,...,x_N$ which are correlated through the $N\times N$ Matrix $cov(x_i,x_j)$.
For a scalar function $f(x_0,...,x_N) \rightarrow \mathbb{R}$, the variance is estimated (linearly) by:
\begin{equation}
Var[f] = \sigma^2 = \sum_{i,j} \frac{\partial f}{\partial x_i} \frac{\partial f}{\partial x_j} cov(x_i,x_j) \,.
\end{equation} 
If instead, $\mathbf{f}$ is a vector field in $m$ dimensions, namely 
$\mathbf{f}(x_0,...,x_N) \rightarrow \mathbb{R}^m$, then the components of $\mathbf{f}$ 
are further correlated. We can write down the relation between the covariance matrices $V$ and $U$ of 
$\mathbf{x}$ and $\mathbf{f}$, respectively, in matrix relations:
\begin{equation}
    U = A V A^T
\end{equation}
where $A$ is the matrix defined by 
\begin{equation}
    A_{ij} = \left[ \frac{\partial f_i}{\partial x_j}\right]_{\mathbf{x} = \mu}
\end{equation}
with expectation value $E[\mathbf{x}] = \mu$.~\cite{cowan1998statistical}
In order to facilitate notation, the covariance matrices will in general be notated without 
specifying the units. If not specified explicitly, the units will correspond to those of the
variables: If $x_i, x_j$ have the units $[x_i], [x_j]$, respectively, 
then the entry of the covariance matrix has the unit $[x_i] \cdot [x_j]$. 

\clearpage


\subsection{Calibration of PVC scintillator}
\label{sub:calibration}
The purpose of this section is to calibrate the channels to known energies, in our case
\begin{itemize}
\item $^{22}$Na Compton edge 1: 341 keV
\item $^{22}$Na Compton edge 2: 1064 keV
\item $^{137}$Cs Compton edge: 477 keV
\end{itemize}
First we fit the two Compton edges of the $^{22}$Na sample. The result is
channel $108 \pm 2$ and $414 \pm 4$ with a correlation of about 50\%, see
figure~\ref{fig:calib_ps_na}.

\begin{figure}[htpb]
    \centering
    \includegraphics[width=0.9\linewidth]{./analysis/figures/calib_ps_na}
    \caption{Calibration of the PVC scintillator with $^{22}$Na sample (measurement time
    16.5 h) with refined background (measurement time 15.6h). Notice that the rate is 
    only fitted for the section in which the background is smaller than the sample. We
    subtracted the background rate at each channel for the sample in order not to fit the 
    behavior of the noise. The error of the two Compton edges is large (see text) coming
    from the fit and their high correlation coefficient.}
\label{fig:calib_ps_na}
\end{figure}
For the $^{137}$Cs sample we found the Compton edge at channel $178.9 \pm 0.3$ 
(notice the much smaller error compared to the $^{22}$Na sample), 
see figure~\ref{fig:calib_ps_cs} for the data and the fit.
\begin{figure}[htpb]
    \centering
    \includegraphics[width=0.9\linewidth]{./analysis/figures/calib_ps_cs}
    \caption{Calibration of the PVC scintillator with $^{137}$Cs sample. Measurement time
    6 h. We subtracted the background rate at each channel. }
\label{fig:calib_ps_cs}
\end{figure}

\begin{figure}[htpb]
    \centering
    \includegraphics[width=0.9\linewidth]{./analysis/figures/calibration_ps_linear_fit}
    \caption{This is the result of the three Compton edges. We will use this calibration
    for all the following calculations. Note that statements about the error and linearity of the MCA 
    are not possible due to the low number of data points}
\label{fig:calibration_ps_linear_fit}
\end{figure}
The final result of the linear calibration can be seen in
figure~\ref{fig:calibration_ps_linear_fit}.

\subsection{Calibration of NaI scintillator}

\label{sub:calibration_of_na_scintillator}
As in the last section we calculate a calibration, this time for the NaI scintillator. The
first figure~\ref{fig:histo_na_137cs} shows the spectrum of the $^{136}$Cs sample 
whereas the data for
$^{22}Na$ can be seen in figure~\ref{fig:histo_na_22na} and figure~\ref{fig:histo_na_22na2}.
The results of the peak fits as well as literature values are displayed in the tables~\ref{tab:peaks_cs_ps}
and \ref{tab:peaks_na_ps} for each of the two samples. The literature values are fitted over peak positions, see 
figure~\ref{fig:calibration_na_linear_fit}, in order to yield a calibration. 
\begin{table}[htpb]
    \centering
    \caption{Peaks and fitting results of $^{137}$Cs.}
\label{tab:peaks_cs_ps}
    \begin{tabular}{lll}
        \rowcolor{LightCyan} Name &Energy & Channel \\ 
        Photo peak & 662 keV & $8040.59 \pm 0.03$\\ 
        Compton edge & 477 keV & $5720 \pm 4$\\  
        Escape peak & 183 keV & $2510 \pm 12$
    \end{tabular}
\end{table}

\begin{figure}[htpb]
    \centering
    \includegraphics[width=0.9\linewidth]{./analysis/figures/histo_na_137cs}
    \caption{Calibration of the NaI scintillator using $^{137}$Cs sample (measurement
    time 2.7h) with refined background (measurement time 1h). }
\label{fig:histo_na_137cs}
\end{figure}

\begin{figure}[htpb]
    \centering
    \includegraphics[width=0.9\linewidth]{./analysis/figures/histo_na_22na}
    \caption{Calibration of the NaI scintillator using $^{22}$Na sample (measurement
        time about 1h) with refined background (same as in figure~\ref{fig:histo_na_137cs},
        measurement time 1h). }
\label{fig:histo_na_22na}
\end{figure}

\begin{figure}[htpb]
    \centering
    \includegraphics[width=0.9\linewidth]{./analysis/figures/histo_na_22na2}
    \caption{Enlargement of the critical area of the Compton edge of
    figure~\ref{fig:histo_na_22na}.     
    The red line is a Savitzky-Golay filter 
    with a width of 1001 points and a fourth
    degree polynomial applied to the data (see~\cite{scipy} for reference) in order
    to see the slope of the curve. The least-squares fit uses the real data, though.
}
\label{fig:histo_na_22na2}
\end{figure}

\begin{table}[htpb]
    \centering
    \caption{Peaks and fitting results of $^{22}$Na. As it is also visible in the figure
    of the leasts-squares fit~\ref{fig:calibration_na_linear_fit} the errors on the
    Compton edges are much higher than those of the Photo peaks. This is due to the high
    degree of uncertainty of the particular edge resulting from the fit. However, the 
    uncertainty of the linear fit is much less in the end.}
\label{tab:peaks_na_ps}
\begin{tabular}{lll}
    \rowcolor{LightCyan} Name &Energy & Channel \\ 
       1. Photo peak& 511 keV & $6347 \pm 3$ \\ 
       2. Photo peak& 1277 keV & $14180 \pm 20 $\\
       1. Compton edge& 341 keV& $4000 \pm 2000$\\
       2. Compton edge& 1064 keV & $12000 \pm 4000$
    \end{tabular}
\end{table}

\begin{figure}[htpb]
    \centering
    \includegraphics[width=0.9\linewidth]{./analysis/figures/calibration_na_linear_fit}
    \caption{All peaks measured by the NaI scintillator with the respective linear fit. The errors on 
        most of the peaks are too small to be shown, only the Compton peaks of $^{22}$Na show
        an exorbitant error. The second photo peak ($^{22}$Na sample) seems to be quite apart from
    the result of the other values; this might be due to a general systematic error.}
\label{fig:calibration_na_linear_fit}
\end{figure}
\clearpage
\subsection{Energy conservation}
\label{sub:energy_conservation}
In this section we work out the dependence of the energy with respect to the scattered
angle. Figure~\ref{fig:coin_ps_background} shows the background (with
coincident setup but without sample) and random coincidences.
We subtract the background from our data before pursuing the analysis.
This will include evaluating a number of different measurements; we will
only show two example figures for each detector 
(figures~\ref{fig:coin_ps_90} and~\ref{fig:coin_ps_15} for the PVC scintillator and 
~\ref{fig:coin_na_90} and~\ref{fig:coin_na_30} for the NaI scintillator). The main
result is summarized in figure~\ref{fig:coin_na_90}. It shows the expectation values of
the energy, dependent on different scattering angles $\theta$.


\begin{figure}[htpb]
    \centering
    \includegraphics[width=0.9\linewidth]{./analysis/figures/coin_background_random}
    \caption{Background and random coincidences of the PVC scintillator. 
        The background is measured in coincident setup for 13.4h, the random coincidences are 
    measured with the $^{137}$Cs sample inserted (measurement time 1h).}
\label{fig:coin_ps_background}
\end{figure}

\begin{figure}[htpb]
    \centering
    \includegraphics[width=0.9\linewidth]{./analysis/figures/coin_ps_90}
    \caption{\textbf{Energy of electrons:}
    Rate of coincident events of PVC scintillator at angle of $\theta = 90^\circ$ 
        (measurement time 1h). The distribution 
    was approximated with a Gaussian distribution. }
\label{fig:coin_ps_90}
\end{figure}

\begin{figure}[htpb]
    \centering
    \includegraphics[width=0.9\linewidth]{./analysis/figures/coin_ps_15_filter_}
    \caption{\textbf{Energy of electrons:}
        Rate of coincident events of 
        PVC scintillator at angle of $\theta = 15^\circ$.
        In this figure we used, unlike at higher angles, applying the Savitzky-Golay filter
        with a width of 81 points and a fourth
        degree polynomial (see~\cite{scipy} for reference). This was done for 
        angles $\theta = 15^\circ$ up to $\theta = 60^\circ$.}
\label{fig:coin_ps_15}
\end{figure}
\begin{figure}[htpb]
    \centering
    \includegraphics[width=0.9\linewidth]{./analysis/figures/coin_na_background}
    \caption{Background in coincident setup for the NaI scintillator, measured for 62h. Note that we can the clear shape because of the long measurement. } \label{fig:coin_na_background}
\end{figure}
\begin{figure}[htpb]
    \centering
    \includegraphics[width=0.9\linewidth]{./analysis/figures/coin_na_30}
    \caption{\textbf{Energy of photons:} Rate of coincident events of 
        NaI scintillator at angle of $\theta = 15^\circ$. The photo peak was approximated
    with a Gaussian distribution. The fit is a scaled Gaussian distribution plus an exponential $c \exp(-b E)$, which
    exhibits the noisy peak from 200 up to 600 keV.}
\label{fig:coin_na_30}
\end{figure}




\begin{figure}[htpb]
    \centering
    \includegraphics[width=0.9\linewidth]{./analysis/figures/coin_na_90}
\caption{\textbf{Energy of photons:} Rate of coincident events of 
        NaI scintillator at angle of $\theta = 90^\circ$. The distribution 
    was approximated with a Gaussian distribution plus exponential, see figure~\ref{fig:coin_na_30}. }
\label{fig:coin_na_90}
\end{figure}


\begin{figure}[htpb]
    \centering
    \includegraphics[width=0.9\linewidth]{./analysis/figures/energy_conservation}
    \caption{Summary of the analysis of the foregoing chapter. We arrive at a small tendency of a negative slope, which
    would disagree with energy conservation. However, this tendency is small, it is
    $[-1.1 \pm 0.3]$ keV/$^\circ$ where the height is $[750 \pm 20]$keV.}
\label{fig:energy_conservation}
\end{figure}
\clearpage
\subsection{Differential cross section}
\label{sub:cross_section}
This last section addresses the differential cross section elaborated in section~\ref{sec:cross_section}, which can
be seen as the rate of a photon being detected under an angle of $\theta$. 
For the calculations
we use the fits of the last section, yielding the intensity $I(\theta)$ of coincident photons. Those are compared to the total
intensity $I_\mathrm{total}$ of photons measured in a separate setup. Furthermore we have to take into account the density of
electrons in the material, which is in our case $n_e= 3.4\cdot10^{23}\mathrm{cm}^{-3}$, and the width $d = [1.4 \pm 0.1]$cm. 
The cross section (unit in barn, i.e. 1 barn = $10^{24}\mathrm{cm}^{2}$) then reads
\begin{equation}
    \label{eq:cross}
    \frac{\mathrm{d}\sigma}{\mathrm{d} \Omega} = \frac{I_\mathrm{total}}{I(\theta) n_e \cdot d \cdot \delta \omega}
\end{equation}
where we already have implied the dependence on the solid angle $\Delta \Omega$. The solid angle can be seen as the size of the
detector in radial units, which can be calculated via the radius of the NaI scintillator $u=[4.8\pm0.1]$cm and the Distance
between NaI opening and PVC scintillator $U=[11.5\pm0.5]$cm
\begin{equation}
    \Delta \Omega = \frac{\pi u^2}{U^2}  = [0.17 \pm 0.02] \pi.
\end{equation}
Up to now we did not talk about microscopical constraints of the detector, but now we have to consider both the absorption
of the material $\sim \exp(-\mu \cdot d)$ and the efficiency of detection $\epsilon$. Both of this quantities are dependent on the
energy the detector is operating. Correcting equation \eqref{eq:cross} we arrive at
\begin{equation}
\label{eq:diff_cross}
    \frac{\mathrm{d} \sigma}{\mathrm{d} \Omega} = \frac{I_\mathrm{total}}{I(\theta) n_e \cdot d \cdot \delta \omega} \
    \frac{\epsilon(E(\theta = 0 ))}{\epsilon(E(\theta))} \left[ e^{-\mu(E_{\mathrm{total} }) d/2}
    \left( 1 - e^{-\mu(E_{\mathrm{total} }) d/2}  \right)\right]^{-1}.
\end{equation}
The last equation can be understood as follows: We approximate the photon to travel half of the PVC width $d$ with its
primary energy, therefore we need the probability of not being absorbed (hence the factor 1 - exp, if the absorption happen 
in this regime, the scattering will not occur). The other probability accounts for being absorbed in the other half of the width,
leading to an scattering which we can detect. However, this correction is very small and does not change the overall asymptotic
of the curve. The result for our intensities is shown in figure~\ref{fig:na_cross_section}. It is clear that the result is only
correct with respect to the order of magnitude; the dependency with respect to the angle $\theta$ is not fulfilled at all, 
compared to the theoretical prediction of the Klein-Nishina formula calculated before. 

\begin{SCtable}[1.5][tbd]

\caption{Summary of values used for the calculation of the differential cross
  section. Efficiency $\epsilon$ taken from \cite{fluegge} and absorption coefficients $\mu$
  taken from \cite{ver}. The equation for the differential cross section~\eqref{eq:diff_cross} shows that only the 
  coefficient $\epsilon(E(\theta = 0) ) / \epsilon(E(\theta))$ is important for the behavior, for $\mu$ it is slightly
different. }
  \begin{tabular}{lll}
      \rowcolor{LightCyan}  $\theta$ / $^\circ$ & $\mu$ / $\mathrm{cm}^{-1}$ & $\epsilon$ \\ 
      \cellcolor{LightCyan}  0 &    0.089 & 0.40 \\ 
 \cellcolor{LightCyan}  15 &   0.091 & 0.41  \\
 \cellcolor{LightCyan}  30 &   0.091 & 0.45  \\
 \cellcolor{LightCyan}  45 &   0.098 & 0.51  \\
 \cellcolor{LightCyan}  60 &   0.108 & 0.55  \\
 \cellcolor{LightCyan}  75 &   0.120 & 0.63  \\
 \cellcolor{LightCyan}  90 &   0.120 & 0.65  \\
 \cellcolor{LightCyan}  105 &  0.120 & 0.70  \\
 \cellcolor{LightCyan}  120 &  0.136 & 0.80  
  \end{tabular}
    \label{tab:cross}
\end{SCtable}


\begin{figure}[htpb]
    \centering
    \includegraphics[width=0.9\linewidth]{./analysis/figures/na_cross_section}
    \caption{Our result of the differential cross section with equation~\eqref{eq:diff_cross}. As the figure shows, the
    experimental result does \textbf{not resemble the theoretical prediction}, for small angle not even with respect to error. 
We think 
that the reason for his huge disagreement is hidden in the calculated intensities (see figure~\ref{fig:coin_na_30}, the
peak next to Gaussian peak, which we approximated by an exponential decay seems to absorb most of the intensity. This changes
for higher angles, but this is not what we would expect. In order for the cross section to satisfy the theoretical prediction,
one expect lower intensities for smaller angles, but due to the \emph{absorption} of the peak next to the coincident peak
we get the opposite behavior: Lower intensities for smaller angles. Our explanation is as follows: Since the measurement
of the random coincidences failed, we could include them in the analysis here, but they seem to recover the nonlinear behavior
seen for low angles. Since random coincidences are a relict of the electronic mechanism of the detector, this disagreement
is of no physical origin. These noise seems to interfere with the intensities of the coincident peaks in such a way that the
intensity dependence with respect to the angles is lost.}
\label{fig:na_cross_section}
\end{figure}
\newpage
