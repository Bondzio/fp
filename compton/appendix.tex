\section{Appendix}
We will show here the figures which where omitted in the main report.
\subsection{Energy of electrons}
In this section we will show the Rate of coincident events of PVC scintillator
at various angles, each one two times: The first approach uses a least-squares fit
with a Gaussian to approximate
the shape of the data, the second approach uses the Savitzky-Golay filter 
with a width of 81 points in the channel and a fourth
degree polynomial applied to the data (see~\cite{scipy} for reference). As it turns out,
the best solution is to take a compromise between these two methods. For small angles
15-60 degrees we take the peak of the filtered data, for the rest we use the Gaussian.


\foreach \n in {15,30,45,60,75,90,105,120}{
\begin{figure}[htpb]
    \centering
    \includegraphics[width=0.9\linewidth]{./analysis/figures/coin_ps_\n}
    \caption{
    Gaussian least squares fit at angle of $\theta = \n^\circ$}
\end{figure}
}

\foreach \n in {15,30,45,60,75,90,105,120}{
\begin{figure}[htpb]
    \centering
    \includegraphics[width=0.9\linewidth]{./analysis/figures/coin_ps_\n_filter_}
    \caption{Savitzky-Golay filter at angle of $\theta = \n^\circ$}
\end{figure}
}
\clearpage
\subsection{Energy of photons}
In this section we show the rate of coincident events of NaI scintillator at
various angles. Those will be fitted against 
\begin{equation*}
    n(E) =  \frac{A}{\sqrt{2\pi\sigma^2}} \exp \left( 
    \frac{-(E-E_0)^2}{2\sigma^2}\right)  + B\exp(-b E).
\end{equation*}
The peak which is referred by the $  B\exp(-b E)$ is due to a Compton edge of the
scintillator, as it was explained in section~\ref{sec:scinti}. The fit range is 
chosen in a way that this edge can be approximated by the exponential, see the 
curves of the fits, which are plotted in the selected range.
\foreach \n in {15,30,45,60,75,90,105,120}{
\begin{figure}[htpb]
    \centering
    \includegraphics[width=0.9\linewidth]{./analysis/figures/coin_na_\n}
    \caption{Least-squares fit with Gaussian
    peak plus exponential at angle of $\theta = \n^\circ$}
\end{figure}
}

