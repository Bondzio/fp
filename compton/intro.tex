\section{Introduction}

At the beginning of the last century the interpretation of the newly established quantum mechanic was still subject to
widespread debate. The view on a very central subject, the nature of light, was dramatically changed by an experiment 
conducted by Arthur H. Compton in 1923. The data he collected from scattering hard X-rays and $\gamma$-rays at light 
elements could were easily understood under the assumption of discrete photons scattering elastically at free electrons. 
This ruled out all previous attempts to remain with a classical theory and persuaded many physicists of the 
particle nature of radiation. Further refinements of experiments lead to the observation that energy and momentum are 
conserved for \textit{each} scattering event. This ruled out the possibility of a statistical conservation, bases of an
interpretation of quantum mechanics by Born, Kramers and Slater. 

This historic influence in mind, we conduct a modern version of the experiment and seek to reproduce two of the central 
theoretical results:
\begin{itemize}
    \item the energy conservation for all scattering angles;
    \item the differential cross section for Compton scattering.
\end{itemize}
The latter is known as the Klein-Nishina formula and was one of the first results of quantum electro dynamics, published
five years after Compton's experiment.
