\subsection{Theoretische Grundlagen}

\subsubsection{Überblick}
In der Rastertunnelmikroskopie wird die Oberfläche von Festkörpern untersucht. 
Dabei wird der quantenmechanische Tunneleffekt ausgenutzt, der einen minimalen 
Stromfluss dort erlaubt, wo klassisch die Potentialbarriere zu hoch wäre. 
Die Wahrscheinlichkeit, dass ein Elektron durch die Barriere ”tunnelt”, 
hängt stark von der Breite derselben ab – daher kann der Tunnelstrom als 
Messgröße für die Entfernung zwischen Spitze und Oberfläche benutzt werden. 
Mit Hilfe des entsprechenden theoretischen Zusammenhanges und Modellen aus 
der Festkörperphysik können so Bilder von der Oberfläche gemacht werden und 
Parameter wie die Gitterkonstante berechnet werden. Untersucht werden in 
diesem Versuch die Oberflächen von Graphit, einer mit Gold beschichteten 
Struktur und des Halbleiters MoS_2. 

subsubsection{Grundlagen der Festkörperphysik}
Wir gehen in unserer Beschreibung der untersuchten Metalle und Halbleiter 
vom Bändermodell aus. Die Elektronen in den äußersten Atomorbitalen der 
Kristallatome tragen zur chemischen Bindung bei. 
