\subsection{Theoretische Grundlagen}

\subsubsection{Überblick}
In der Rastertunnelmikroskopie wird die Oberfläche von Festkörpern untersucht. 
Dabei wird der quantenmechanische Tunneleffekt ausgenutzt, der einen minimalen 
Stromfluss dort erlaubt, wo klassisch die Potentialbarriere zu hoch wäre. 
Die Wahrscheinlichkeit, dass ein Elektron durch die Barriere ”tunnelt”, 
hängt stark von der Breite derselben ab – daher kann der Tunnelstrom als 
Messgröße für die Entfernung zwischen Spitze und Oberfläche benutzt werden. 
Mit Hilfe des entsprechenden theoretischen Zusammenhanges und Modellen aus 
der Festkörperphysik können so Bilder von der Oberfläche gemacht werden und 
Parameter wie die Gitterkonstante berechnet werden. Untersucht werden in 
diesem Versuch die Oberflächen von Graphit, einer mit Gold beschichteten 
Struktur und des Halbleiters $\mathrm{MoS_2}$. 

\subsubsection{Grundlagen der Festkörperphysik}
Wir gehen in unserer Beschreibung der untersuchten Metalle und Halbleiter 
vom Bändermodell aus. Durch die gegenseitige Beeinflussung der Atome im Gitter
werden die im Einzelatom noch stark voneinander abgetrennten Energie-Eigenzustände
der Elektronen aufgespalten und folgen so dicht aufeinander, dass Elektronen 
sehr leicht zwischen den einzelnen Zuständen wechseln können. Die atomaren
Energieniveaus bleiben jedoch zum Großteil soweit getrennt, dass klar definierte
"Bänder" entstehen. Das für $T = 0˚ K$ äußere Energieband ist das Valenzband.
Die zur chemischen Bindung beitragenden Elektronen gehören genau diesem Band an 
(Valenz = Bindung, Valenzelektronen). Das über dem Valenzband liegende Band wird 
als Leitungsband bezeichnet. Elektronen im Leitungsband sind räumlich nicht mehr 
gebunden, da sich die Orbitale der jeweiligen Atome überlagern – diese Elektronen 
können daher leicht Energie eines elektrischen Feldes aufnehmen und sich in dem 
Gitter bewegen. \\

