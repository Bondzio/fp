\section{Historische Einführung in die Rastertunnelmikroskopie}
Das Konzept des Tunnelns tauchte in der 
Festkörperphysik auf, als versucht wurde, durch Vakuum bzw. durch
eine Vakuumbarriere zu tunneln \cite{binnig1982tunneling}. Diese
waren zunächst aufgrund der Vibrationen nicht erfolgreich. Nun sind
die Vorteile des Vakuumtunnelns aber evident:
\begin{enumerate}
\item Konzeptuell am einfachsten herzustellende Barriere 
\item Freier Zugang der Elektroden für die Untersuchung anderer
physikalischer und chemischer Prozesse
\end{enumerate}
1981 führten die Autoren G.Binnig, H.Rohrer,
Ch.Gerber und E.Weibel in Zürich zum ersten Mal ein erfolgreiches
Tunnelexperiment mit einem justierbarem Vakuum Spalt durch. 
Ziel war hierbei, das Phänomen des Tunnelns so zu erforschen,
um es in der Spektroskopie und andere Methoden einsetzen zu können. 
Offensichtlich war der schwierige Teil der, die Vibrationen,
die vergangene Experimente fehlschlugen ließen, hinreichend zu
unterdrücken, um somit das eigentliche Signal noch identifizieren zu
können. Dies wurde in dem erwähnten Experiment durch eine 
Dämpfung des Tunnelbauteils mithilfe von Leviation durch
Supraleiter-induzierten Magneten sowie
der Steuerung mit Piezoelementen erreicht. Der Trick liegt darin,
die charakteristischen Frequenzen so zu wählen, dass die 
Eigenfrequenzen des Materials für Vibrationen weit darüber liegen.
Dies ist möglich, indem die Größe des Bauteils sehr klein
skaliert wird, somit können sich keine Vibrationen ausbilden.

