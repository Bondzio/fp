%%%%%%%%%%%%%%%%%%%%%%%%%%%%%%%%%%%%%%%%%
% Short Sectioned Assignment
% LaTeX Template
% Version 1.0 (5/5/12)
%
% This template has been downloaded from:
% http://www.LaTeXTemplates.com
%
% Original author:
% Frits Wenneker (http://www.howtotex.com)
%
% License:
% CC BY-NC-SA 3.0 (http://creativecommons.org/licenses/by-nc-sa/3.0/)
%
%%%%%%%%%%%%%%%%%%%%%%%%%%%%%%%%%%%%%%%%%

%----------------------------------------------------------------------------------------
%	PACKAGES AND OTHER DOCUMENT CONFIGURATIONS
%----------------------------------------------------------------------------------------

\documentclass[paper=a4, fontsize=11pt]{scrartcl} % A4 paper and 11pt font size

\usepackage[T1]{fontenc} % Use 8-bit encoding that has 256 glyphs
\usepackage{fourier} % Use the Adobe Utopia font for the document - comment this line to return to the LaTeX default
\usepackage{amsmath,amsfonts,amsthm} % Math packages
\usepackage{natbib}
\usepackage{pgfplots}

\usepackage[utf8]{inputenc} 
\usepackage[ngerman]{babel}

\usepackage{latexsym}
\usepackage{textcomp}
\usepackage[T1]{fontenc}
\usepackage{bm}% bold math
\usepackage{hyperref}
\usepackage{graphicx}
\usepackage{epsfig}
\usepackage{framed,color}
\usepackage[usenames,dvipsnames]{pstricks}
\usepackage{epsfig}

\usepackage{tikz}




\usepackage{lipsum} % Used for inserting dummy 'Lorem ipsum' text into the template

\usepackage{sectsty} % Allows customizing section commands
\allsectionsfont{\centering \normalfont\scshape} % Make all sections centered, the default font and small caps

\usepackage{fancyhdr} % Custom headers and footers
\pagestyle{plain} % Makes all pages in the document conform to the custom headers and footers
\fancyhead{} % No page header - if you want one, create it in the same way as the footers below
\fancyfoot[L]{} % Empty left footer
\fancyfoot[C]{} % Empty center footer
\fancyfoot[R]{\thepage} % Page numbering for right footer
%\renewcommand{\headrulewidth}{0pt} % Remove header underlines
%\renewcommand{\footrulewidth}{0pt} % Remove footer underlines
\setlength{\headheight}{13.6pt} % Customize the height of the header
\usepackage{eso-pic}
\numberwithin{equation}{section} % Number equations within sections (i.e. 1.1, 1.2, 2.1, 2.2 instead of 1, 2, 3, 4)
\numberwithin{figure}{section} % Number figures within sections (i.e. 1.1, 1.2, 2.1, 2.2 instead of 1, 2, 3, 4)
\numberwithin{table}{section} % Number tables within sections (i.e. 1.1, 1.2, 2.1, 2.2 instead of 1, 2, 3, 4)

\setlength\parindent{0pt} % Removes all indentation from paragraphs - comment this line for an assignment with lots of text

%----------------------------------------------------------------------------------------
%	TITLE SECTION
%----------------------------------------------------------------------------------------

\newcommand{\horrule}[1]{\rule{\linewidth}{#1}} % Create horizontal rule command with 1 argument of height

\title{ 
\normalfont \normalsize 
\textsc{Albert-Ludwigs-Universität Freiburg} \\ [25pt] % Your university, school and/or department name(s)
\horrule{0.5pt} \\[0.4cm] % Thin top horizontal rule
\huge Rastertunnelmikroskop \\ % The assignment title
\horrule{2pt} \\[0.5cm] % Thick bottom horizontal rule
}

\author{Friedrich Schüßler und Volker Karle} % Your name

\date{\normalsize\today} % Today's date or a custom date

\begin{document}
\maketitle

\tableofcontents
\thispagestyle{empty}
\newpage
\setcounter{page}{1}


%----------------------------------------------------------------------------------------
%	PROBLEM 1
%----------------------------------------------------------------------------------------

\part{Versuchsprotokoll}
\section{Historische Einführung in die Rastertunnelmikroskopie}
Das Konzept des Tunnelns tauchte in der 
Festkörperphysik auf, als versucht wurde, durch Vakuum bzw. durch
eine Vakuumbarriere zu tunneln \cite{binnig1982tunneling}. Diese
waren zunächst aufgrund der Vibrationen nicht erfolgreich. Nun sind
die Vorteile des Vakuumtunnelns aber evident:
\begin{enumerate}
\item Konzeptuell am einfachsten herzustellende Barriere 
\item Freier Zugang der Elektroden für die Untersuchung anderer
physikalischer und chemischer Prozesse
\end{enumerate}
1981 führten die Autoren G.Binnig, H.Rohrer,
Ch.Gerber und E.Weibel in Zürich zum ersten Mal ein erfolgreiches
Tunnelexperiment mit einem justierbarem Vakuum Spalt durch. 
Ziel war hierbei, das Phänomen des Tunnelns so zu erforschen,
um es in der Spektroskopie und andere Methoden einsetzen zu können. 
Offensichtlich war der schwierige Teil der, die Vibrationen,
die vergangene Experimente fehlschlugen ließen, hinreichend zu
unterdrücken, um somit das eigentliche Signal noch identifizieren zu
können. Dies wurde in dem erwähnten Experiment durch eine 
Dämpfung des Tunnelbauteils mithilfe von Leviation durch
Supraleiter-induzierten Magneten sowie
der Steuerung mit Piezoelementen erreicht. Der Trick liegt darin,
die charakteristischen Frequenzen so zu wählen, dass die 
Eigenfrequenzen des Materials für Vibrationen weit darüber liegen.
Dies ist möglich, indem die Größe des Bauteils sehr klein
skaliert wird, somit können sich keine Vibrationen ausbilden.


\cite{binnig1982surface}
\cite{kittel2013einfuhrung}
\cite{chen1993introduction}

\section{Theorie des Quantentunnelns}
\textit{Quantentunneln}, oder kurz \textit{Tunneln} 
bezeichnet das quantenmechanische Phänomen, wenn 
die Durchtrittswahrscheinlichkeit eines Teilchens
durch eine Potenzialbarriere nicht null ist, selbst wenn die Energie
des Teilchens geringer ist als das Potenzial selbst ($E < V$), was
in der klassischen Mechanik nicht möglich wäre. Dies
spielt eine wichtige Rolle bei
vielen Phänomenen in Natur und Technik,
beispielsweise bei der Kernfusion der Sonne, bei der Diode und
daher auch beim Transistor und somit bei der Funktionionsweise
eines Computers an sich, aber auch beim Quantencomputer oder
eben in unserem Fall beim RTM. Das Phänomen des Quantentunnelns
wurde Anfang des 20ten Jahrhunderts mit der Entdeckung der 
Quantenmechanik postuliert und Mitte des Jahrhunderts bestätigt.
\subsection{Mathematische Herleitung von Quantentunneln}
In den folgenden Ausführungen werden Kenntnisse der Quantenmechanik
vorrausgesetzt. Betrachten wir zunächst die Zeitunabhängige
Schrödingergleichung für ein Teilchen in einer Dimension:
\begin{align}
\left [ \frac{-\hbar^2}{2m}\partial_x^2 + V(x) \right ]\psi(x) = E\psi(x) \\ 
\Leftrightarrow \left [ \frac{-\hbar^2}{2m}\partial_x^2 \right ]\psi(x) = \left [E-V(x) \right ]\psi(x) 
\end{align}
Im Spezialfall wenn $V(x)$ konstant ist, können wir die Gleichung
sofort mit planaren Wellen lösen:
\begin{align}
    k^2 = \frac{2m}{\hbar^2}(V-E)\\
    \psi(x) \sim \exp(kx) 
\end{align}
Wenn $V(x)$ nicht konstant ist, können wir mithilfe der WKB-Methode 
\cite{froman1970transmission}
immerhin noch den Transmissionskoeffizienten berechnen, sofern
das Potenzial zwischen zwei Rändern $x_1$ und $x_2$ eingespannt ist 
und ausserhalb davon null wird. Dazu setzen wir für die 
Wellenfunktion $\psi(x)=\exp(\phi(x))$ an, 
mit einer komplexen Funktion $\phi(x)$. 




	
\bibliographystyle{plain}
\bibliography{Protokoll}
\end{document}
