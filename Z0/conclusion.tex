\section{Conclusion}
We will again state our main results.
The fermionic cross-sections agree with the literature value within one standard deviation:
\begin{align*}
    \Gamma_l &=  \left[ 0.84 \pm 0.004 \right]\mathrm{GeV}\\ 
    \Gamma_h &=  \left[ 1.78 \pm 0.90 \right] \mathrm{GeV}
\end{align*}
Since we get nearly the same values for all three widths, the lepton universality is confirmed. We could also
confirm the number of neutrino generations.
The reason for the very good agreement lies in the \textit{machine learning technique} used in the classification
of the particles. \\
For the properties of the Z boson we found:
\begin{align*}
    \Gamma_Z  &= \left[ 2.558 \pm 0.020 \right] \mathrm{GeV}\\
    M_Z &= \left[ 91.186 \pm 0.008 \right] \mathrm{GeV / c^2} 
\end{align*}
The decay width is slightly larger than the literature value, but the mass agrees within one standard deviation.
The result of the forward backward asymmetry reads
\begin{equation*}
   \sin^2 \theta_W = \sin^2\theta_W = 0.23069 \pm 0.11769.
\end{equation*}
This result is in good agreement with the literature value as well.

In conclusion, we consider this experiment to be quite interesting since we gained deep 
insight into various concepts of experimental particle physics in general and the
fascinating phenomena of data analysis of high energy collider in particular.
