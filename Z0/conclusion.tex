\section{Conclusion}
At this point we restate and discuss our main results.
The fermionic cross sections, 
\begin{align*}
    \Gamma_l &=  \left[ 0.84 \pm 0.004 \right]\mathrm{GeV}\\ 
    \Gamma_h &=  \left[ 1.78 \pm 0.90 \right] \mathrm{GeV} \, ,
\end{align*}
agree with the literature values within one standard deviation:
Since we get nearly the same values for all three widths, the lepton universality is confirmed. We could also
confirm the number of neutrino generations.
For the properties of the Z boson we found
\begin{align*}
    \Gamma_Z  &= \left[ 2.558 \pm 0.020 \right] \mathrm{GeV}\\
    M_Z &= \left[ 91.186 \pm 0.008 \right] \mathrm{GeV / c^2} \, . 
\end{align*}
The decay width is slightly larger than the literature value, but the mass agrees within one standard deviation.
The result for the Weinberg angle using the forward-backward asymmetry reads
\begin{equation*}
   \sin^2 \theta_W = 0.23069 \pm 0.11769.
\end{equation*}
This result is in good agreement with the literature value as well.

One reason for the good agreement lies in the machine learning technique used in the classification
of the particles. We were quite surprised that the algorithm, which is ignorant of the theory behind the
experiment, did so much better in classifying then we could achieve using our background knowledge. 
Baring in mind the considerable difference between the Monte Carlo simulation data used for training and the real 
data, we still see our approach confirmed by the results. 
\\
In conclusion, we consider this experiment to be quite interesting since we gained deep 
insight into various concepts of experimental particle physics in general and the
fascinating aspects of data analysis in high energy physics in particular.
