\section{Theoretical foundations}
\subsection{Historical background}
If one is to sketch the history exploring the weak interaction, 
the starting point is certainly the first formulation of the 
weak interaction, accomplished by Enrico Fermi in 1933. 
He described the nuclear beta decay as an interaction 
between four spin-$\frac{1}{2}$ particles at one vertex, 
introducing the neutrino. Although regarded highly speculative%
\footnote{His first attempt to publish was rejected by the journal
\emph{Nature} for this reason, triggering Fermi to abandon 
theoretical physics and turning to experiments for a while.}, 
it proofed to be useful for physics up at low energies. 
Soon it became clear that this approximation would not work out 
at high energies, where an interaction along an intermediate 
vector boson has to be included into the description. 
Experimentalists had a hard time dealing with this problem, 
as the involved particles do not form bound states (like the 
pions, mediating the interaction in nucleons). A first theory
on weak interaction, written by Glashow, Weinberg and Salam 
in 1968, identified three vector bosons: two charged $W$ bosons 
and a neutral $Z$ boson. A formula for their masses involved a 
parameter $\theta_W$ to be measured. In 1982, this empirical 
input allowed to calculate the masses up to
\begin{equation}
    M_W = 82 \pm 2 \mathrm{GeV / c^2},\qquad 
    M_Z = 92 \pm 2 \mathrm{GeV / c^2}
\end{equation}
The following year finally yielded the first confirmation 
of the existence of $W$ and $Z$ bosons, reported by the group 
of Carlo Rubbia at CERN. 


-> number of particles: neutrino generations: $2.99 \pm 0.06$. 
standard model uses empirical input: CKM matrix, Weinberg angle
look! \cite{pdg} test


\subsection{Cross sections and luminosity}
\subsection{Forward-backward asymmetry}
\subsection{Higher order corrections}
\subsection{Particle-matter interaction}
