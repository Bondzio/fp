\section{Introduction}

Particle physics has been a scientific field with large advances 
during the last decades. Ever larger and more powerful collidors 
produced a vast number of different particles. On of these colliders 
has been the \emph{LEP} (Large Electron-Positron Collider) run by CERN\@.
During a period from 1989 until 2000, electrons and positrons were 
collided in order to measure the electro-weak interaction more precise 
than ever before. In particular, in its first phase the collider 
achieved the production of real Z bosons working at beam energies of 
around 45 GeV (half of the Z boson mass $M_Z = 91$ GeV). Later on, 
the energy was raised to produce pairs of the charged W bosons 
($M_W = 80$ GeV). Before being dismantled in order to give room 
for the LHC (large hadron collider), the electrons reached energies 
of 209 GeV. 


This report deals with data produced by LEP until 1994, thus events 
lying with CMS-energies $\sqrt(s) = M_Z$. We use the properties of 
different decay modes of the $Z$ boson in order to categorize a 
large number of events and calculate the respective cross sections. 
We then deduce the mass of the $Z^0$ and calculate the number of 
neutrino generations. Finally, we compute the forward-backward 
asymmetry of the process $e^-e^+ \to \mu^-\mu^+$, which is used 
to test the electro-weak theory. Finally, one of its central 
parameters, the \emph{Weinberg angle}, is calculated from the 
asymmetry. 
