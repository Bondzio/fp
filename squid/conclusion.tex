\FloatBarrier
\clearpage
\section{Conclusion}
This experiment showed us an application to the fascinating phenomena
of superconductivity. We have learned that by exploiting 
the quantum behavior of Josephson junctions we can reach a far better
precision than the ad-hoc approach with the law of Biot-Savart, which suffered
from insufficient information about the geometry of the experimental setup.
For instance, when we took $R=[50\pm1] \Omega$ to be the resistance of the induction loop,
we measured
\begin{align*}
B_{bs} &= [4.6\pm 2.6] nT   \qquad \quad \text{measured with the Biot-Savart law} \\ 
B_{sq} &= [5.09\pm0.12] nT \qquad \text{measured with the SQUID}.
\end{align*}
For the other values refer to table~\ref{tab:comparison}.
Undoubtedly the SQUID method yields a far better precision.
When calculating the dipole moment, we had to rely on the measured
value of the distance between SQUID and the sample. This is the reason
for the much greater uncertainty for the dipole moments
\begin{align*}
p_{bs} &= [620\pm150] C\cdot nm  \qquad \text{measured with the Biot-Savart law} \\ 
p_{sq} &= [690\pm140] C\cdot nm  \qquad \text{measured with the SQUID}.
\end{align*}
Furthermore, we have measured the magnetic field and the dipole moments for
various other materials. 
For detailed values refer to 
table~\ref{tab:materials}. 
We emphasize that some of these measurements have an large systematic 
error and cannot be cited to as a successful measurement, 
since it was not possible to maintain a stable signal. \\
At last we were able to create the respective polar plots for the 
magnetic field of the various samples. In those plots especially, the
appearance of noise and systematic disturbance is clearly visible. 

In conclusion, we consider this experiment to be quite interesting since we gained deep 
insight into various concepts of condensed matter physics in general and the
fascinating phenomena of superconductivity in particular.
