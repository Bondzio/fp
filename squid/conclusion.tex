\FloatBarrier
\clearpage
\section{Conclusion}
This experiment showed us an application to the fascinating phenomena
of superconductivity. We have learned that by exploiting 
the quantum behaviour of josephson junctions we can reach a far better
precission than the ad-hoc approach with the law of Biot-Savart, which suffered
from insufficient information about the geometry of the experimental setup.
For instance, when we took $R=[50\pm1] \Omega$ to be the resistance of the induction loop,
we measured (for the other values please refer to table~\ref{tab:comparison}.
\begin{align*}
B_{bs} &= [4.6\pm 2.6] nT   \qquad \quad \text{measured with the Biot-Savart law} \\ 
B_{sq} &= [5.09\pm0.12] nT \qquad \text{measured with the SQUID}.
\end{align*}
Apparently the SQUID method yields a far better precission.
When calculating the dipole moment, we could had to rely on the measured
value of the distance betweem SQUID and the sample. This is the reason
for the much greater uncertainty for the dipole moments
\begin{align*}
p_{bs} &= [620\pm150] C\cdot nm  \qquad \text{measured with the Biot-Savart law} \\ 
p_{sq} &= [690\pm140] C\cdot nm  \qquad \text{measured with the SQUID}.
\end{align*}
Furthermore we have measured the magnetic field and the dipole moments for
various other materials (for detailed values please refer to 
table~\ref{tab:materials}). We have to emphasize the fact that some of
these measurements have an huge systematic error and cannot be refered 
to as a sucessfull measurement, since it was not possible to remain a 
stable signal. \\
At last we were able to create the respective polar plots for the 
magnetic field of the various samples. Especially in those plots the
appereance of noise and systematic disturbance is clearly visible (please
refer to the appendix, section~\ref{sec:appendix} for the figures). 

\paragraph{Howbeit,} we consider this experiment to be quite interesting since we gained deep 
insight into various concepts of condensed matter physics in general and the
fascinating phenomena of superconductivity in particular.
