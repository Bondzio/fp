\section{Setup and Procedure}
The experimental setup consists of tube (see figure~\ref{fig:setup1}) being
isolated both thermal and in terms of magnetic fields. Inside
the tube a Squid detector is cooled down below superconducting temperature.
First we use a conductor loop, which will be rotated in a certain frequency with
an external drive. The Squid is connected with an oscilloscope such that 
we can track the magnetic field continuosly.
\begin{figure}[htpb]
    \centering
    \includegraphics[width=0.5\linewidth]{figures/foto5.jpeg}
    \caption{The photography shows the tube, where the squid is operating inside in order
    to screen perturbative magnetic fields and thermic fluctuations. The tube is cooled down
    with liquid nitrogen in order to establish the superconduction. In the lower part of the
    tube is additional space not being cooled down by the nitrogen for the objects to be inserted.
    The magnetic fields of these objects will then be measured by the squid.}
    \label{fig:setup1}
\end{figure}
\paragraph{The procedure} of the conduction of the experiment will be
the following:
\begin{enumerate}
\item Filling the tube with liquid nitrogen in order to establish the superconduction in the
operating squid, which will take about 15 minutes.
\item Calibration of the squid by means of the visible Amplitude in the oscilloscope \textit{Hameg 1508-2}.
Maximizing the quality of the signal with the amplitude VCA and the frequency CVO of the oscillating circuit, as
well as the offset OFF.
\item Measuring the magnetic field of the oscillating unit with different resistances.
\item Measuring the magnetic field of various other objects.
\end{enumerate}


