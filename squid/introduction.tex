\section{Introduction}
In this experiment we are in the fortunate position of having the 
possibility to exploit the fascinating phenomena of superconductivity for
our purposes -- fascinating both theoretically and experimentally. We will
estimate the magnetic field and the dipole moment of a specific conduction loop,
but various other materials as well. In the end we can even make assumptions
about the spatial spread of the magnetic fields of the samples.
\paragraph{The Superconducter} 
was discovered in 1911 by Heike Kamerlingh Onnes, while
investigating the temperature dependence of electrical resistance.
He noticed the disappearance of resistance of mercury below 4.2 K and
after this discovery more materials were found to have this
superconducting property. We will talk about these properties and the
physical intuition behind in the following, section~\ref{sec:theory}.
Part of the scope is 
to get an understanding of superconductivity in general, since we use it
extensively in our experimental setup, the SQUID 
(see section~\ref{sec:techniques} for an comprehensive introduction).


