\section{conclusion}



The values measured in the Haynes \& Shockley experiment 
are displayed in the table~\ref{tab:conc_h_s} below:
\renewcommand{\arraystretch}{1.5}
\begin{table}[htdp]
    \centering
    \caption{
        Results of the Haynes \& Shockley experiment, variating the 
        acceleration voltage or the distance between creating the 
        cloud of free charge and measuring it. The errors correspond 
        to propagated errors of fitting not including systematical ones 
        and are thus underestimated. 
        }
	\begin{tabular}{|p{4cm}|p{3cm}|p{3cm}|p{3cm}|}
		\hline
		\rowcolor{tabcolor}
		Parameter           & $U = \text{const.}$   & $d = \text{const.}$   & Literature~\cite{staatsexamen}
     	\\ \hline
        mobility $\mu_n$    & $2640 \pm 130$        & $3000 \pm 120$        & $3900$
        life time $\tau_n$  & $6.7 \pm 1.0$         & $1.6 \pm 0.5$         & $45 \pm 2$
        diffusion const. $\D_n$& $140 \pm 20$       & $130 \pm 20$          & $101$
		\hline
	\end{tabular}
    \label{tab:h_s_fit_parameters}
\end{table}
For both the electron mobility and diffusion constant we observe agreement in the 
order of magnitude between our results and the given literature values. The 
errors are underestimated due to not including systematical errors in the calculation. 
Other the shortcomings of the electronics (which we experienced in terms 
of loosing the signal), it is clear that the experiment does not allow the 
measurement of these constants in a perfect crystal. This is especially 
true for the life time, which is one order of magnitude smaller then the 
accepted value for germanium. 


All three experiments resulted in a good introduction into semiconductor physics. 
Especially the Haynes and Shockley experiment, notwithstanding the disagreement in 
numerical results, gave a nice insight into models of the reality in semiconductors. 

