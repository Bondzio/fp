\section{Conclusion}
In the first part of the experiment, we basically obtained one 
central results: The approximated band gap energy $E_g$ for germanium 
and silicon. Our measured values lie at 
\begin{align}
    E_{g, \mathrm{Si}} &= (1.14 \pm 0.05) \,\mathrm{eV} \\
    E_{g, \mathrm{Ge}} &= (0.69 \pm 0.03) \,\mathrm{eV}
\end{align}
Both values cover the literature values within one standard deviation.
This accuracy was a rather surprising result to us, as we expected the 
applied method to be of a much lower certainty. 

The values measured in the Haynes \& Shockley experiment 
are displayed in the table~\ref{tab:conc_h_s} below:
\renewcommand{\arraystretch}{1.5}
\begin{table}[H]
    \centering
    \caption{
        Results of the Haynes \& Shockley experiment, variating the 
        acceleration voltage or the distance between creating the 
        cloud of free charge and measuring it. The errors correspond 
        to propagated errors of fitting not including systematical ones 
        and are thus underestimated. 
        }
	\begin{tabular}{|p{4cm}|p{3cm}|p{3cm}|p{3cm}|}
		\hline
		\rowcolor{tabcolor}
		Parameter           & $U = \text{const.}$   & $d = \text{const.}$   & Literature~\cite{staatsexamen} \\ 
        \hline
        mobility $\mu_n$    & $2640 \pm 130$        & $3000 \pm 120$        & $3900$     \\
        life time $\tau_n$  & $6.7 \pm 1.0$         & $1.6 \pm 0.5$         & $45 \pm 2$ \\
        diffusion const. $D_n$ & $140 \pm 20$      & $130 \pm 20$          & $101$     \\
		\hline
	\end{tabular}
    \label{tab:conc_h_s}
\end{table}
For both the electron mobility and diffusion constant we observe agreement in the 
order of magnitude between our results and the given literature values. The 
errors are underestimated due to not including systematical errors in the calculation. 
Other the shortcomings of the electronics (which we experienced in terms 
of loosing the signal), it is clear that the experiment does not allow the 
measurement of these constants in a perfect crystal. This is especially 
true for the life time, which is one order of magnitude smaller then the 
accepted value for germanium. 

Examination of the two semiconductors Si and CdTe gave an insight into the 
characteristics that allow for the usage as detectors for ionizing 
radiation. The central result is the superiority of heavier materials
(here represented by CdTe) over light ones (Si) when the registration of as 
many events as possible is key. The quantitative result, 
the ratio of absorption coefficients has been estimated to be
\begin{align}
    \mathrm{\frac{Abs_{Si}}{Abs_{CdTe}}}(59.5\, \mathrm{keV}) &= (3.51 \pm 0.07) \\
    \mathrm{\frac{Abs_{Si}}{Abs_{CdTe}}}(122.06\, \mathrm{keV}) &= (1.16 \pm 0.08)\\ 
    \mathrm{\frac{Abs_{Si}}{Abs_{CdTe}}}(136.47\, \mathrm{keV}) &= (1.9 \pm 0.6) \, ,
\end{align}
agreeing with the literature values in order of magnitude. The errors 
are those obtained by numerical calculation and do not include 
systematical errors, such as impurities or errors induced by the electronics. 
With one setup and the two possible semiconductors as detectors, one 
would thus need to measure 30 to 60 times as long with the Si semiconductor 
to obtain the same number of counts and thus the same statistical error. 
As a second, somewhat less striking result, we calculated the 
relative energy resolution (RER). The main result here is the increasing 
resolution with increasing energy: The resolution at 122 keV Co peak is 
twice the one at the 59.5 keV Am peak for both semiconductors.  

All three experiments resulted in a good introduction into semiconductor physics. 
Especially the Haynes and Shockley experiment, notwithstanding the disagreement in 
numerical results, gave a nice insight into models of the reality in semiconductors. 

