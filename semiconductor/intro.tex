\section{Introduction}
The modern age of crystal electronics is based up materials which are neither
metals nor insulators; such materials are called semiconductors, 
since their electrical properties are
intermediate between those of metals and insulators. 
This is because of a rather special arrangement of
the energy levels of electrons.\cite{phillips2012bonds}. 
Semiconductors are one of the central elements in the developments 
achieved during the last century. Their essential role in computers, 
solar cells, diodes and sensors lead to an almost daily confrontation 
with technical applications. This experiment gives a first insight 
in some of the key properties of the used materials: 
The \emph{band gap energy}, the \emph{mobility} and other characteristics 
of free electrons and finally the possible application 
of semiconductors as detectors for ionizing radiation. 
We will thus carry out three separate experiments: the basic approximation 
of the band gap energy for germanium (Ge) and silicon (Si) by comparison 
of absorption of transmission of a sample, the very intuitive 
\emph{Haynes \& Shockley} experiment 
which visualizes a moving charge distribution in an electric field 
and finally testing the applicability of two detectors based on the 
semiconductors cadmium telluride (CdTe) and silicon.
