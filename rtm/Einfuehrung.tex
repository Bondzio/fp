\section{Einführung in die Rastertunnelmikroskopie}
In der Rastertunnelmikroskopie (engl. \textit{scanning tunneling
microscope}, \textbf{STM}) wird die Oberfläche von Festkörpern auf atomarer 
Größenskala untersucht. Dabei wird der quantenmechanische Tunneleffekt 
ausgenutzt, der einen minimalen Stromfluss dort erlaubt, wo klassisch 
die Potentialbarriere zu hoch wäre. Eine Spitze, die im Idealfall mit 
einem Atom endet, fährt die Oberfläche in einem festen Raster ab, während 
ein Computer den Stromfluss im nA-Bereich und damit die Verschiebung misst. 
Die Wahrscheinlichkeit, dass ein Elektron durch die Potentialbarriere 
zwischen Oberfläche und Spitze ”tunnelt”, hängt stark von dem Abstand 
dazwischen ab – daher kann der Tunnelstrom als Messgröße für diesen 
Abstand benutzt werden. Mit Hilfe des entsprechenden theoretischen 
Zusammenhanges und Modellen aus der Festkörperphysik können so Bilder 
von der Geometrie der Oberfläche erstellt werden und beispielsweise 
Parameter wie die Gitterkonstante berechnet werden. Gängige Auflösungen liegen
bei 0.1 nm in der Ebene sowie 0.01 nm in der Tiefe. Untersucht werden in 
diesem Versuch die Oberflächen von Graphit, einer mit Gold beschichteten 
Struktur und des Halbleiters $\mathrm{MoS_2}$. 

