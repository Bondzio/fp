\section{Fazit}
Der Versuch hat uns einen schönen Eindruck von der Struktur der Oberfläche 
von Graphit im atomaren Bereich gegeben und vor 
Allem den Möglichkeiten der Rastertunnelmikroskopie gegeben. Die Durchführung erwies 
sich allerdings in vielerlei Hinsicht als deutlich schwieriger, als 
zuvor angenommen. Viele Störungseinflüsse und insbesondere minderwertige Spitzen, 
die die Annahme der einatomigen Spitze mit hoher Wahrscheinlichkeit nicht erfüllen, 
führten neben vielen Fehlversuchen, die lediglich Rauschen hervorbrachten, zu einigen 
wenigen Aufnahmen atomarer Auflösung. Für Graphit waren so die erwarteten drei Punkte 
pro Sechseck zu erkennen. Neben einem leichten Rauschen hat sich in der Analyse jedoch 
gezeigt, dass diese Bilder 'verzogen' waren, also nicht die tatsächliche Symmetrie 
der Oberfläche erkennbar war. Dementsprechend sind auch die berechneten Gitterkonstanten 
nicht alle mit dem Literaturwert vereinbar. Das gilt in unserem Fall lediglich für 
zwei von dei Richtungen, während die dritte stark gestaucht erscheint. \\ 

Auf höheren Skalen waren facettenartige Strukturen sowohl für Graphit als auch für 
die Gold-Probe erkennbar. Allerdings sticht auch ein weiteres Muster ins Auge: die Aufnahmen 
erscheinen entlang jeweils einer horizontalen und einer vertikalen Achse gespiegelt. 
Zu diesem Zeitpunkt gibt es für dieses Phänomen keine zufriedenstellende Erklärung – 
wir schlagen einen Fehler im Betriebsmodus des Messinstrumentes vor, müssten für eine 
genauere Untersuchung des Gerätes in Betracht ziehen. 


