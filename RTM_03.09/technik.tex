\subsection{Technische Aspekte für unser Experiment}
In unserem Fall besteht die Spitze des RTMs aus einer möglichst spitzen
Drahtspitze, durch die der Tunnelstrom geleitet wird. Dazu wird an die Spitze eine 
Tunnelspannung angelegt und so nahe an die Probe angenähert, bis der Tunnelstrom 
im nA-Bereich fliessen kann.
Dies erfordert eine ziemlich präzise Steuerung der Spitze, welche nun weiter erläutert wird.
Der Tunnelstrom muss einerseits konstant gehalten werden (jedenfalls nach der gängigen Methode),
andererseits darf die Distanz nicht zu schnell geändert werden, da sonst eine Kollision mit der
Probe stattfindet. Dafür gibt es einen Regelkreis, der kontinuierlich den angegeben Wert
(Führungsgröße) mit dem
tatsächlichen Wert (Regelgröße) vergleicht und
die Steuerung daraufhin verändert. Hier wird ein PID-Regler
eingesetzt, welcher auch bei Störeinflüssen seine Funktion, den Strom konstant zu halten,
noch erfüllt, wenn die Störungen nicht zu hoch sind. Im Allgemeinen existieren viele
verschiedene Regler, die für verschiedene Verhaltensformen des Signals angepasst sind. 
Hierfür wollen wir eine kurze Einführung in die Kinematik geben \cite{regelungstechnik}.
\subsubsection{Einführung in die Kontrolltheorie und Kinematik}
Die Kontrolltheorie beschäftigt sich mit der Beeinflussung von Systemen, um bestimmten
Ausgangsgrößen einen gewünschten zeitlichen Verlauf aufzuprägen\cite{regelungstechnik}.
Der gewünschte zeitliche Verlauf der Ausgangsgrößen wird durch die Sollgrößen oder Sollwerte
definiert. Eingangsgröße der Steuerung ist also der Sollwert $w$, ihre Ausgangsgröße ist die 
Stellgröße $u$, welche zusammen mit der Störgröße $d$ wieder eine Eingangsgröße bildet. Die 
Anforderungen an die Regelung sind klar: Die Regelabweichung soll im stationären Zustand
 nach Beendigung
eines Einschwingvorgangs möglichst klein sein, im Falle einer Führungsgrößenänderung oder einer
Störung soll die entstandene Regelabweichung möglichst schnell wieder eliminiert werden und
Die Stabilität des Gesamtsystems muss durchgehend gewährtleistet werden. 
Die \textit{Laplace-Transformation} \cite{regelungstechnik2} dient als wichtigstes Hilfsmittel
bei regelungstechnischen Aufgaben, die mit Differenzialgleichungen beschrieben werden können; dies
ist insbesondere beim PID-Regler der Fall. Die Laplace-Transformation wird folgendermaßen definiert:
\begin{align}
F(s) &= \mathcal{L}\left [f(x)\right ] = \int_{0}^{\infty}f(x)e^{-sx}dx \\
f(x) &= \mathcal{L}^{-1}\left [F(s)\right ] = 
\frac{1}{2\pi i} \int_{c - i\infty}^{c+ i\infty} F(s)e^{sx}ds
\end{align} 

Wobei als Argument die komplexe Variable $s=\sigma + i\omega$ auftritt (somit handelt es sich
bei der schon erwähnten Fourier-Transformation um einen Spezialfall mit $F(s = ik)$) und
das vorkommende $c\in \mathbb{R}$ so gewählt werden muss, dass der Integrationsweg in der 
Konvergenzhalbebene längs einer Parallelen zur imaginärenAchse im Abstand $c$ verläuft, wobei
$c$ größer als die Realteile sämtlicher singulärer Punkte von $F(s)$ sein muss.
Hier die wichtigsten Eigenschaften der Laplace-Transformation:
\begin{align}
&\mathcal{L}\left [a f(x) + b g(x)\right ]
    = a F(s) + b G(s) 
    &\mbox{ (Linearität) }\\

&\mathcal{L}\left [f(ax)\right ]
    = \frac{1}{a}F(\frac{s}{a}) 
    &\mbox{ (Ähnlichkeit) }\\

&\mathcal{L}\left [f(x-a)\right ]
    = \frac{1}{a}F(\frac{s}{a}) 
    &\mbox{ (Translation) }\\

&\mathcal{L}\left [\frac{d^nf(x)}{dx^n}\right ]
    = s^nF(s) - \sum_{i=1}^{n}s^{n-i}\frac{d^{i-1}f(x)}{dt^{i-1}} 
    &\mbox{ (Differentiation) }\\

&\mathcal{L}\left [\int_{0}^{t}f(x)dx\right ]
    = \frac{1}{s}F(s) &\mbox{ (Integration) }\\

&\mathcal{L}\left [f(x) * g(x)\right ](k)
    = \mathcal{F}\left [f(x)\right ]\mathcal{F}\left [g(x)\right ]
    \! &\mbox{ (Faltungseigenschaft) }\\
&\mathcal{F}_k\left [\left | F(k) \right |^2\right ](x)
   =  \int_{-\infty}^{\infty}\bar{f}(\tau)f(\tau + x) d\tau 
   &\mbox{ (Wiener-Khinchin Theorem) }
\end{align}




