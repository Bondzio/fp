\section{Theorie des Quantentunnelns}
\textit{Quantentunneln}, oder kurz \textit{Tunneln} 
bezeichnet das quantenmechanische Phänomen, wenn 
die Durchtrittswahrscheinlichkeit eines Teilchens
durch eine Potenzialbarriere nicht null ist, selbst wenn die Energie
des Teilchens geringer ist als das Potenzial selbst ($E < V$), was
in der klassischen Mechanik nicht möglich wäre. Dies
spielt eine wichtige Rolle bei
vielen Phänomenen in Natur und Technik,
beispielsweise bei der Kernfusion der Sonne, bei der Diode und
daher auch beim Transistor und somit bei der Funktionionsweise
eines Computers an sich, aber auch beim Quantencomputer oder
eben in unserem Fall beim RTM. Das Phänomen des Quantentunnelns
wurde Anfang des 20ten Jahrhunderts mit der Entdeckung der 
Quantenmechanik postuliert und Mitte des Jahrhunderts bestätigt.
\subsection{Mathematische Herleitung von Quantentunneln}
In den folgenden Ausführungen werden Kenntnisse der Quantenmechanik
vorrausgesetzt. Betrachten wir zunächst die Zeitunabhängige
Schrödingergleichung für ein Teilchen in einer Dimension:
\begin{align}
\left [ \frac{-\hbar^2}{2m}\partial_x^2 + V(x) \right ]\psi(x) = E\psi(x) \\ 
\Leftrightarrow \left [ \frac{-\hbar^2}{2m}\partial_x^2 \right ]\psi(x) = \left [E-V(x) \right ]\psi(x) 
\end{align}
Im Spezialfall wenn $V(x)$ konstant ist, können wir die Gleichung
sofort mit planaren Wellen lösen:
\begin{align}
    k^2 = \frac{2m}{\hbar^2}(V-E)\\
    \psi(x) \sim \exp(kx) 
\end{align}
Wenn $V(x)$ nicht konstant ist, können wir mithilfe der WKB-Methode 
\cite{froman1970transmission}
immerhin noch den Transmissionskoeffizienten berechnen, sofern
das Potenzial zwischen zwei Rändern $x_1$ und $x_2$ eingespannt ist 
und ausserhalb davon null wird. Dazu setzen wir für die 
Wellenfunktion $\psi(x)=\exp(\phi(x))$ an, 
mit einer komplexen Funktion $\phi(x)$. 

