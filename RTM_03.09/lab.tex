\section{Durchführung des Versuchs}

Die Durchführung des Versuches gestaltete sich deutlich schwieriger als zuvor angenommen. 
Erst im Laufe des zweiten Versuchstages gelang uns eine Aufnahme der Graphitoberfläche mit 
atomarer Auflösung. Als Ursache ist vor allem die Verwendung unbrauchbarer Spitzen zu sehen, 
deren Unbrauchbarkeit jedoch erst nach langer Zeit bemerkbar wurde. Dazu kamen immer wieder 
kleinere Probleme mit dem Messgerät, bei dem zum Teil wichtige Funktionen versagten. 

\subsection{Ablauf des Experiments nach Laborprotokoll}
Die gesamten Untersuchunge wurden im 'constant current'-Betriebsmodus gemacht. Dabei wurde 
der Strom in fast allen Fällen bei der Standarteinstellung $I = 1\mathrm{nA}$ belassen. 
Ausnahmen davon sind extra markiert worden. 
Die ersten Versuche wurden, soweit nicht anders angegeben,  mit den Standarteinstellungen 
des Programms unternommen. Zu den ersten Versuchen mit gebrauchten Spitzen zum Kennenlernen 
des Programms gibt es keine Aufzeichnungen.
Die Herstellung der Spitze hat einiges an Übung erfordert. Bei den ersten Spitzen hat das 
Reißen nicht geklappt, sodass ein unter der Lupe relativ glatter Schnitt zu erkennen war. 
Brauchbare Ergebnisse lieferten diese Spitzen nicht. Die meisten Spitzen, die wir verwendeten, 
waren aus alten Spitzen 'recycelt', indem an der Spitze nachgeschnitten bzw. neu abgerissen 
wurde. Tendenziell bessere Ergebnisse lieferten die Spitzen die direkt vom langen Draht 
abgerissen wurden, da hier das Abreißen einfacher zu realisieren war. In der folgenden Tabelle 
sind 'recycelte' Spitzen mit 'r', ohne Weiterbehandlung wiederverwendete mit 
'w' und neue Spitzen mit 'n' gekennzeichnet. $U$ ist die Spannung, die an die Spitze angelegt 
wurde. Für die automatische Annäherung wurde jeweils mit ca. 80mV angefangen, danach bis 
in 10mV Schritten bis auf ca. 50mV gesenkt. Die Annäherungsgeschwindigkeit $v$ dabei leicht 
von 48\% bis 35\% heruntergefahren. 
\\\\
\begin{tabular}{l l p{12cm}}
     & Spitze & Beschreibung \\
    1   & r & Spitze und Probe nach erfolgreicher Annäherung beim Abrastern 
kollidiert (bei 600nm Rasterlänge) \\
    2   & w & U bis minimal 50mV, auf verschiedenen Größenskalen kein Signal (I = 0) \\
    3   & r & U = 64mV, Symmetrische Struktur bei 600nm erkennbar, bei 180nm 
regelmäßige Struktur in der Größenordnung von 50nm zu erkennen, bei 10nm Rauschen 
(Abb. \ref{fig:graphit_01}) \\
    4   & r & unter der Lupe vielversprechendere Schnitt. 'Advance'-Funktion geht 
nicht. Ein Annähern mit der Hand führt zur Kollision \\
    5   & n & 'Advance'-Funktion nach Putzen sämtlicher Teile und Neustart des Systems wieder 
funktionstüchtig. Die ersten Messungen bei 600nm, 180nm und 30nm sind konsistent (es wurde 
jeweils auf den Bereich des Maximums vergrößert). Bei 10nm ist jedoch nur Rauschen zu erkennen.
(Abb. \ref{fig:graphit_02}). \\
    6   & r & Bei verschiedenen Spannungen nach vollendetem 'Approach' kein Signal \\
    7   & r & Gleiches Problem wie bei '6' \\
    8   & w & Kein gutes Signal, Spitze offenbar unbrauchbar
(Abb. \ref{fig:graphit_03}) \\
    9   & n & siehe weiter unten \\ 
   10   & r & Kollision gemeldet nach Annäherung mit Geschwindigkeit 45\% bei
$U = 80\mathrm{mV}$, gemessener Tunnelstrom nicht homogen. \\
   11   & r & erneut Kollision bei erstem Scan und gleichen Einstellungen, bei kleineren 
Skalen (100nm, 10nm) lediglich Rauschen.  \\
\end{tabular}
\\\\
Anhand der neunten Spitze, die neu vom Draht gerissen wurde,  wird exemplarisch das Vorgehen 
dargestellt. Dabei ist $n$ der Schritt, $v$ die eingestellte Annäherungsgeschwindigkeit für 
die automatische Annäherung, $U$ die angelegte Spannung , $a$ die Skala 
und $z_{\mathrm{max}}$ der maximale berechnete Höhenunterschied.
\\\\
\begin{tabular}{l l l l l p{8cm}}
$n$ & $v$ & $U$ & $a$ & $z_{\mathrm{max}}$ & Beobachtung \\
1   & 45\%  & 80mv  & 600nm & 27nm  & Symmetrische Struktur über den gesamten Scanbereich, 
starkes Rauschen 
(Abb. \ref{fig:graphit_04_01}) \\
2   & 40\%  & 60mv  & 100nm & 2.7nm  & Fast ausschließlich Rauschen zu erkennen, leichte 
Wölbung, Tunnelstrom $I$ nicht konstant 
(Abb. \ref{fig:graphit_04_02}) \\
3   & 38\%  & 50mv  & 10nm & 1.4nm  & Horizontale Linien (Rauschen) bei konstantem $I$, 
keine Gitterstruktur erkennbar.
(Abb. \ref{fig:graphit_04_03}) \\
\end{tabular}
\\\\
Der erste und einzige erfolgreiche Versuch wurde mit der 12. Spitze, die wiederum neu 
abgerissen wurde, unternommen. In der folgenden Tabelle sind die einzelnen Schritte 
aufgelistet. Die automatische Annährungsgeschwindigkeit liegt mit Ausnahme des ersten 
Schrittes (46\%) immer bei 40\%, der Tunnelstrom $I$ bei 1nA (mit Ausnahme des letzten 
Schrittes mit 2nA). Mit 'px/l' ist die Anzahl der Messpunkte bzw. Pixel pro Linie und 
damit auch die Anzahl der Linien in z-Richtung bezeichnet, mit $t$ die Zeit, die pro 
Abfahren einer Linie. 
\\\\
\begin{tabular}{p{3pt} p{12pt} p{9pt} p{9pt} p{9pt} p{12pt} p{14pt} p{9cm}}
$n$ & Abb.      & $\frac{U}{\mathrm{mV}}$ & $\frac{\mathrm{px}}{\mathrm{l}}$ & 
    $\frac{t}{\mathrm{s}}$ & $\frac{a}{\mathrm{nm}} $ & 
    $\frac{z_{\mathrm{max}}}{\mathrm{nm}}$ & Beobachtung \\
1   & \ref{fig:graphit_06_01}& 80 & 128 & 0.4 & 600 & 17 & großflächige Struktur mit 
Stufen und konstanter Höhe im Zentrum; Wiederholung der Messung bei 55mV ergab keine 
sichtbare Veränderung \\
2   & \ref{fig:graphit_06_02}& 55 & 256 & 0.4 & 180 & 2.36 & Vergrößerung in den Bereich bei 
ca. $(x, y) = (500\mathrm{nm}, 0\mathrm{nm})$ im Bezug auf Abb \ref{fig:graphit_06_01}; 
Abstufungen und Interferenzartige Ringe zu erkennen \\
3   & \ref{fig:graphit_06_03}& 55 & 256 & 0.4 &  30 & 0.27 & Trotz Rauschen ist eine 
regelmäßige Struktur zu erkennen \\
4   & \ref{fig:graphit_06_04}& 55 & 256 & 0.4 &  10 & 0.16 & Gitter aus gerade angeordneten 
Punkten sichtbar\\
5   & \ref{fig:graphit_06_05}& 55 & 256 & 0.4 &  10 & 0.21 & Wiederholung an gleicher Stelle, 
verzogenes Bild mit schlechterer Auflösung. Wärmedrift?\\
6   & \ref{fig:graphit_06_06}& 55 & 128 & 0.4 &   3 & 0.16 & \\
7   & \ref{fig:graphit_06_07}& 55 & 256 & 0.8 &   3 & 0.16 & Wiederholung an gleicher Stelle, 
starkes Rauschen, stark verzogen: Wärmedrift, zu langsam?\\
8   & \ref{fig:graphit_06_08}& 45 & 256 & 0.8 &   3 & 0.16 & trotz starken Rauschens Punkte 
sichtbar \\
9  & \ref{fig:graphit_06_09}& 35 & 256 & 0.8 &   3 & 0.27 & Rauschen und Drift erkennbar, 
keine Punkte mehr zu sehen \\ 
10  & \ref{fig:graphit_06_10}& 35 & 256 & 0.8 &  30 & 1.55 & Wechsel der Position, jetzt bei 
ca. $(x, y) = (100\mathrm{nm}, 450\mathrm{nm})$ im Bezug auf Abb \ref{fig:graphit_06_01}; Stufe zu erkennen, 
starkes Rauschen \\
11  & \ref{fig:graphit_06_11}& 35 & 256 & 0.8 &  10 & 0.27 & Streifen mit unterschiedlich 
guter Auflösung, Gitter im unteren Streifen gut zu erkennen \\
12  & \ref{fig:graphit_06_12}& 35 & 256 & 0.8 &  10 & 0.27 & Abschirmbox zum ersten Mal benutzt 
(während des Messvorgangs - Störung als schwarzer Streifen bei $y \approx 1\mathrm{nm}$ 
erkennbar); das vorher sehr verrauschte Bild ist etwas besser; weiterhin Bänder mit 
unterschiedlicher Auflösung und Drift \\
13  & \ref{fig:graphit_06_13}& 35 & 256 & 0.8 &   3 & 0.27 & Rauschen, kaum Struktur zu 
erkennen\\ 
14  & \ref{fig:graphit_06_14}& 35 & 256 & 0.8 &   3 & 0.21 & Wiederholung der Messung 14, 
deutlich besseres Bild; auf parallel angeordneten Streifen liegende, punktförmige 
Erhöhungen, die annähernd gleichseitige Dreiecke bilden, wie es für die Abbildung der 
Oberfläche von Graphit erwartet wurde. \\ 
15  & \ref{fig:graphit_06_15}& 35 & 256 & 0.8 & 0.8 & 0.16 & fast ausschließlich Rauschen zu 
erkennen \\
16  & \ref{fig:graphit_06_16}& 35 & 256 & 0.8 & 1.6 & 0.16 & regelmäßig angeordnete Punkte 
wegen starken Rauschens nur andeutungsweise zu erkennen \\
17  & \ref{fig:graphit_06_17}& 35 & 256 & 0.8 & 3.2 & 0.27 & nach Veränderung des Regelstromes 
deutlich schlechtere Bildqualität als bei Versuch 15; Streifen mit starker Störung \\
\end{tabular}

\newcommand{\picwidth}{0.45\textwidth}
\centering
\begin{figure}
    \begin{subfigure}[b]{\picwidth}
        \includegraphics[width=\textwidth]{data/Graphit/pic_01_01_600nm}
        \caption{}
        \label{fig:graphit_01_01}
    \end{subfigure}\qquad
    \begin{subfigure}[b]{\picwidth}
        \includegraphics[width=\textwidth]{data/Graphit/pic_01_02_180nm}
        \caption{}
        \label{fig:graphit_01_02}
    \end{subfigure}
    \begin{subfigure}[b]{\picwidth}
        \includegraphics[width=\textwidth]{data/Graphit/pic_01_03_10nm}
        \caption{}
        \label{fig:graphit_01_03}
    \end{subfigure}
    \caption{STM-Aufnahmen von Graphit, Spitze Nr. 3}\label{fig:graphit_01}
\end{figure}

\centering
\begin{figure}
    \begin{subfigure}[b]{\picwidth}
        \includegraphics[width=\textwidth]{data/Graphit/pic_02_01_600nm}
        \caption{}
        \label{fig:graphit_02_01}
    \end{subfigure}\qquad
    \begin{subfigure}[b]{\picwidth}
        \includegraphics[width=\textwidth]{data/Graphit/pic_02_02_180nm}
        \caption{}
        \label{fig:graphit_02_02}
    \end{subfigure}
    \begin{subfigure}[b]{\picwidth}
        \includegraphics[width=\textwidth]{data/Graphit/pic_02_03_30nm}
        \caption{}
        \label{fig:graphit_02_03}
    \end{subfigure}\qquad
    \begin{subfigure}[b]{\picwidth}
        \includegraphics[width=\textwidth]{data/Graphit/pic_02_04_10nm}
        \caption{}
        \label{fig:graphit_02_04}
    \end{subfigure}
    \caption{STM-Aufnahmen von Graphit, Spitze Nr. 5}\label{fig:graphit_02}
\end{figure}

\centering
\begin{figure}
    \begin{subfigure}[b]{\picwidth}
        \includegraphics[width=\textwidth]{data/Graphit/pic_03_01_50nm}
        \caption{}
        \label{fig:graphit_03_01}
    \end{subfigure}\qquad
    \begin{subfigure}[b]{\picwidth}
        \includegraphics[width=\textwidth]{data/Graphit/pic_03_02_15nm}
        \caption{}
        \label{fig:graphit_03_02}
    \end{subfigure}
    \caption{STM-Aufnahmen von Graphit, Spitze Nr. 8}\label{fig:graphit_03}
\end{figure}

\centering
\begin{figure}
    \begin{subfigure}[b]{\picwidth}
        \includegraphics[width=\textwidth]{data/Graphit/pic_04_01_600nm}
        \caption{}
        \label{fig:graphit_04_01}
    \end{subfigure}\qquad
    \begin{subfigure}[b]{\picwidth}
        \includegraphics[width=\textwidth]{data/Graphit/pic_04_02_100nm}
        \caption{}
        \label{fig:graphit_04_02}
    \end{subfigure}
    \begin{subfigure}[b]{\picwidth}
        \includegraphics[width=\textwidth]{data/Graphit/pic_04_03_10nm}
        \caption{}
        \label{fig:graphit_04_03}
    \end{subfigure}
    \caption{STM-Aufnahmen von Graphit, Spitze Nr. 9}\label{fig:graphit_04}
\end{figure}

\centering
\begin{figure}
    \begin{subfigure}[b]{\picwidth}
        \includegraphics[width=\textwidth]{data/Graphit/pic_06_01_600nm}
        \caption{}
        \label{fig:graphit_06_01}
    \end{subfigure}\qquad
    \begin{subfigure}[b]{\picwidth}
        \includegraphics[width=\textwidth]{data/Graphit/pic_06_02_180nm}
        \caption{}
        \label{fig:graphit_06_02}
    \end{subfigure}
    \begin{subfigure}[b]{\picwidth}
        \includegraphics[width=\textwidth]{data/Graphit/pic_06_03_30nm}
        \caption{}
        \label{fig:graphit_06_03}
    \end{subfigure}\qquad
    \begin{subfigure}[b]{\picwidth}
        \includegraphics[width=\textwidth]{data/Graphit/pic_06_04_10nm}
        \caption{}
        \label{fig:graphit_06_04}
    \end{subfigure}
    \begin{subfigure}[b]{\picwidth}
        \includegraphics[width=\textwidth]{data/Graphit/pic_06_05_10nm}
        \caption{}
        \label{fig:graphit_06_05}
    \end{subfigure}\qquad
    \begin{subfigure}[b]{\picwidth}
        \includegraphics[width=\textwidth]{data/Graphit/pic_06_06_3nm}
        \caption{}
        \label{fig:graphit_06_06}
    \end{subfigure}
    \caption{STM-Aufnahmen von Graphit, Spitze Nr. 12, Messungen 1 bis 6}\label{fig:graphit_06}
\end{figure}
\centering
\begin{figure}
    \begin{subfigure}[b]{\picwidth}
        \includegraphics[width=\textwidth]{data/Graphit/pic_06_07_3nm}
        \caption{}
        \label{fig:graphit_06_07}
    \end{subfigure}\qquad
    \begin{subfigure}[b]{\picwidth}
        \includegraphics[width=\textwidth]{data/Graphit/pic_06_08_3nm}
        \caption{}
        \label{fig:graphit_06_08}
    \end{subfigure}
    \begin{subfigure}[b]{\picwidth}
        \includegraphics[width=\textwidth]{data/Graphit/pic_06_09_3nm}
        \caption{}
        \label{fig:graphit_06_09}
    \end{subfigure}\qquad
    \begin{subfigure}[b]{\picwidth}
        \includegraphics[width=\textwidth]{data/Graphit/pic_06_10_30nm}
        \caption{}
        \label{fig:graphit_06_10}
    \end{subfigure}
    \begin{subfigure}[b]{\picwidth}
        \includegraphics[width=\textwidth]{data/Graphit/pic_06_11_10nm}
        \caption{}
        \label{fig:graphit_06_11}
    \end{subfigure}\qquad
    \begin{subfigure}[b]{\picwidth}
        \includegraphics[width=\textwidth]{data/Graphit/pic_06_12_10nm}
        \caption{}
        \label{fig:graphit_06_12}
    \end{subfigure}
    \caption{STM-Aufnahmen von Graphit, Spitze Nr. 12, Messungen 7 - 12}\label{fig:graphit_06}
\end{figure}
\centering
\begin{figure}
    \begin{subfigure}[b]{\picwidth}
        \includegraphics[width=\textwidth]{data/Graphit/pic_06_13_3nm}
        \caption{}
        \label{fig:graphit_06_13}
    \end{subfigure}\qquad
    \begin{subfigure}[b]{\picwidth}
        \includegraphics[width=\textwidth]{data/Graphit/pic_06_14_3nm}
        \caption{}
        \label{fig:graphit_06_14}
    \end{subfigure}
    \begin{subfigure}[b]{\picwidth}
        \includegraphics[width=\textwidth]{data/Graphit/pic_06_15_800pm}
        \caption{}
        \label{fig:graphit_06_15}
    \end{subfigure}\qquad
    \begin{subfigure}[b]{\picwidth}
        \includegraphics[width=\textwidth]{data/Graphit/pic_06_16_1600pm}
        \caption{}
        \label{fig:graphit_06_16}
    \end{subfigure}
    \begin{subfigure}[b]{\picwidth}
        \includegraphics[width=\textwidth]{data/Graphit/pic_06_17_3nm_set_point_2nA}
        \caption{}
        \label{fig:graphit_06_17}
    \end{subfigure}
    \caption{STM-Aufnahmen von Graphit, Spitze Nr. 12, Messungen 13 - 17}\label{fig:graphit_06}
\end{figure}


