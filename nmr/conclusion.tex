\section{Conclusion}
\paragraph{}
We conclude the experiment by revising what we learned and to which extend we were 
able to reproduce accepted atomic constants for the probes under examination. 
We did get a good insight into the basic principles of NRM-spectroscopy and possibilities 
to measure the corresponding splitting of energy levels. Of course, todays applications 
of the effect are of a much more exact scale. However, we were still able to reproduce 
the corresponding gyromagnetic ratios and nuclear g-factors of
the flourine isotope $^{19}$F in a fluid and in Teflon as well as the proton $^1$H in water 
and glycol. The measured values are 
\begin{itemize}
    \item
        for $^{19}$F: $\gamma_N = (253 \pm 3)$ MHz/T (fluid) and 
        $\gamma_N = (250 \pm 3)$ MHz/T (Teflon);
    \item
        for $^1$H: $\gamma_N = (268 \pm 3)$ MHz/T (water) and 
        $\gamma_N = (269 \pm 3)$ MHz/T (glycol).
\end{itemize}
The respective literature values are $251$ MHz / T  and $268$ MHz / T, respectively. 
One of the central results is the considerable proximity of these values, making 
$^{19}$F similarly attractive for applycations of the NMR-spectroscopy. 

The second part of the experiment did not work out well. On the contrary, we were not able 
to measure any signal with the Lock-in method and applying a sine modulated sawtooth signal 
to the external $\B$-field. The reasons for the failure are suspected in the highly sensitive 
and unstable electronics. However, personal failure in setting the correct signal is also 
not beyond the bounds of possibility. 
