\section{Introduction}

In this experiment, we measure we examine the fascinating effect of nuclear 
magnetic resonance, discovered in 1936 and nowadays used in important applycations 
in medicine and biology. 
We measure the associated resonance frequency in order to compute 
gyromagnetic ratio of protons in water, glycole and $^19$F
(as a fluid and in Teflon), as well as the nuclear magnetic moment of the fluor. 
In order to measure the spin, we exploit the Zeeman effect
by measuring the absorption of a high frequency magnetic field. 
The measurement of the correspoding resonance frequency is done achieved 
with two different methods: first, measuring the absorption peak directly, 
secondly, using the \emph{Lock-In}-method. 

