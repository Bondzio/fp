\section{Theoretical backgrounds}
\subsection{Nuclear spin}
Assuming a particles bescribed by its wavefunction $\Psi$, we can describe 
its fundamental properties by quantum numbers corrisponding to eigenstates
of different operators. Aside from the quantum number $n$ corresponding to the 
Hamiltonian $\hat{H}$ and thus to the energy level, particles can be further characterized 
by the orbital angular momentum quantum number $l$ and the spin quantum number 
$s$ as well as the projections of tboth of them onto the $z$-axis, given by 
$m_l$ and $m_s$, respectively. For the nucleus with wavefunction $\Psi_N$, 
spin and angular momentum are coupled to the 
nuclear spin with corresponding operator $\hat{I}$. Its properties are given by:
\begin{align}
    \hat{I} \Psi_N &= \mathbf{I} \Psi_N \\ 
\intertext{for which}
    \mathbf{I}^2 &= \hbar^2 I(I + 1)  &\text{(absolute value)} \\
    \left[\hat{H}, \hat{I}^2\right]_- &= 0  &\text{(constant)}
\end{align}
For the quantum number of nuclear spin, we have $I \in {0, \frac{1}{2}, 1, \ldots}$, while $\hbar$ is the 
reduced Planck constant. A second constant quantum number corresponding to the nuclear spin can be defined:
the projection on the $z$-axis, which in the experiment is given by the direction of the external $B$-field.
It is given by
\begin{align}
    I_z &= m_I \hbar \\
    m_I &\in \left\{-I, -I + 1, \ldots, I - 1, I\right\}
\end{align}
Thus, we have $2I + 1$ values for $m_I$. Depending on whether $I$ is integer or half integer, 
we have to apply Bose- or Fermi-statistics; the particles are called bosons or fermions, 
accordingly~\cite{Demtroeder1}.

\subsection{Magnetic moment}
Magnetic moment
\begin{equation}
    \mathbf{\mu} \mu = \gamma_K \mathbf{I}
\end{equation}
unit: \emph{nuclear magnetcon}
\begin{equation}
    \mu_K = \frac{\hbar e}{2 m_p}
\end{equation}


\subsection{Nuclear magnetic resonance}
\subsection{Relaxation}
\subsection{Hall effect and sensor}
\subsection{Lock-in method}


\subsection{Materiales used in the experiment}

(PTFE, know by the brand name Teflon), 
