\section{Procedure}
\label{sec:procedure}
In order to derive the magnetic moments and gyromagnetic ratios, 
we use the relationship to applied magnetic fields given by the Zeeman effect 
as established in the theory section \ref{sec:theory). 
The point of resonance leads to the appearance of absorption lines 
in the strength of the highly oscillating field. These absorption lines 
will then be measured. 
The steps undertaken in this experiment are the following:
\begin{enumerate}
\item
measuring the homogenity of the magnetic field in order to define a stable working point;
\item
measuring the resonance frequency for a given external magnetic field by searching for 
equidistant absorption peaks;
\item
measuring the resonance frequency with a lock-in amplifier and sine modulated saw tooth signal.
\end{enumerate}
From the obtained data, we then calculate the following quantities:
\begin{enumerate}
\item
nuclear magnetic moment of $^19$F;
\item
gyromagnetic moment of proton in $^1$H, glycole and fluor.
\end{enumerate}


\section{Measurements}

\subsection{Measuring the magnetic field}
We measured the magnetic field inside the hole for the probe with the Hallo eefect sensor described in the setup section. 
The sensor was initialized for $B = 0$ mT before the measurement. We first measured over the 
entire range accessibile for a constant current $I$. Results are shown in table \ref{tab:b_height}. 
Not all the collected data is displayed, as the linear dependence is shown already by a subset. Further 
the dependency on $U$ is not displayed, as the magnetic field induced electrically only depends on the 
$I$.
\begin{table}[htdp]
\centering
    \begin{tabular}{|p{1.34cm}|p{2.16cm}|p{0.8cm}|p{1.34cm}|p{2.16cm}|p{0.8cm}|p{1.34cm}|p{2.16cm}|}
        \cline{1-2}\cline{4-5}\cline{7-8}
        $z$ / mm \cellcolor{LightCyan}& $B$ / mT \cellcolor{LightCyan}&&
        $z$ / mm \cellcolor{LightCyan}& $B$ / mT \cellcolor{LightCyan}&&
        $z$ / mm \cellcolor{LightCyan}& $B$ / mT \cellcolor{LightCyan}\\ 
        \cline{1-2}\cline{4-5}\cline{7-8}
        0 & 228 &&14 & 349 &&26 & 349 \\ 
        3 & 341 &&15 & 349 &&27 & 349 \\ 
        4 & 346 &&16 & 350 &&28 & 349 \\ 
        5 & 348 &&17 & 350 &&29 & 349 \\ 
        6 & 349 &&18 & 350 &&30 & 349 \\ 
        7 & 350 &&19 & 350 &&31 & 349 \\ 
        8 & 350 &&20 & 350 &&32 & 348 \\ 
        9 & 350 &&21 & 350 &&33 & 348 \\ 
        10 & 350 &&22 & 350 &&34 & 347 \\ 
        11 & 349 &&23 & 350 &&35 & 347 \\ 
        12 & 349 &&24 & 349 &&36 & 346 \\ 
        13 & 349 &&25 & 349 &&37 & 345 \\ 
        \cline{1-2}\cline{4-5}\cline{7-8}
    \end{tabular}
    \caption{
        Measurement of magnetic field $B$ at height $z$ for $I = 2.62$ A.Corresponding uncertainties: $\Delta z = 0.4$ mm, $\Delta B = 5$ mT.
        }
    \label{tab:b_height}
\end{table}


 
Then, we measured the dependency of $B(I)$ 
in a broader range of $I \in (0.57 2.58)$ A and for a smaller range, closer to the suspected region of resonance 
for the given HF generator (table \ref{tab:B_I}). 


\subsection{Direct measurement of resonance frequency}

\subsection{Measurements with Lock-In method}
\subsubsection{Calibration of phase difference}

\section{Interpretation}

\subsection{Homogenity of $B$}
The results of the measurement of the homogenity of the magnetic field $B$ are plotted in figure 
\ref{fig:b_height}. As one can see, a good working point is defined for $z = -2$ cm, since 
this is close to the midpoint of the observed plateau. The exitence of the plateau indicates the homogenity 
of the field, which is a necessary foundation for the further analysis. 
