\section{Experimental setup}

\section{Procedure}
\label{sec:procedure}
In order to derive the magnetic moments and gyromagnetic ratios, 
we use the relationship to applied magnetic fields given by the Zeeman effect 
as established in the theory section \ref{sec:theory)}. 
The point of resonance leads to the appearance of absorption lines 
in the strength of the highly oscillating field. These absorption lines 
will then be measured. 
The steps undertaken in this experiment are the following:
\begin{enumerate}
\item
measuring the homogenity of the magnetic field in order to define a stable working point;
\item
measuring the resonance frequency for a given external magnetic field by searching for 
equidistant absorption peaks;
\item
measuring the resonance frequency with a lock-in amplifier and sine modulated saw tooth signal.
\end{enumerate}
From the obtained data, we then calculate the following quantities:
\begin{enumerate}
\item
nuclear magnetic moment of $^19$F;
\item
gyromagnetic moment of proton in $^1$H, glycole and fluor.
\end{enumerate}


