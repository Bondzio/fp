\FloatBarrier
\section{Conclusion}
\paragraph{Diffraction gratings}~\\
The first part of the experiment went well in terms of the data obtained as well 
as mirroring the theoretical considerations. We did not measured 
physical constants but properties of gratings. We thus are not able to compare to 
literature values and have to rely on a appropriate approximation of the errors.
During the first part, 
we measured the lattice constant of the sine grating to 
\begin{equation}
    K_\mathrm{sin, nom} =(1.01 \pm 0.02) \, \mu\mathrm{m} \,.
\end{equation}
where the small error of $2\%$ reflects the relatively easy 
and straight forward measurement of the lattice constant. 
All further calculation rely on the calibration done with the 
grating of known lattice constant $K_\mathrm{Calibration} = 0\,$mm.
The correspondence between $\theta$ and $t$ agrees well 
with the expectation of a linear relation. 
With this correspondence, we calculated the lattice constants of five 
undeclared gratings. For the given experimental setup, we further calculated 
the maximum resolution for each lattice being illuminated by a laser beam of 
diameter $\phi = 2.9 \pm 0.5\,$mm.
The results for both values are summarized in the table below:
\begin{table}[H]
    \centering
	\begin{tabular}{|p{3.82cm}|p{3.82cm}|p{3.82cm}|}
		\hline
		\rowcolor{tabcolor}
        Grating & $\overline{K_i}$ / ($\mu$m) & $a$ \\ \hline
        $1$  & $ 125.8 \pm 2.5  $ & $ 230 \pm 40 $ \\
		$2$  & $ 34.1 \pm 0.8   $ & $ 509 \pm 88 $ \\
		$3$  & $ 100.8 \pm 1.2  $ & $ 287 \pm 49 $ \\
		$4$  & $ 100.8 \pm 1.0  $ & $ 258 \pm 44 $ \\
		$5$  & $ 50.7 \pm 1.2   $ & $ 342 \pm 59 $ \\
		\hline
	\end{tabular}
    \caption{
        Calculated lattice constants $\overline{K_i}$ and 
        resolutions $a$ of amplitude gratings one to five. 
        }
    \label{tab:gratings_K_conc}
\end{table}
Using the approximation of the Fourier series for the aperture function, 
we calculated the width of the gaps $b$ and the ratio of $b$ to the 
lattice constant $K_1$ with
\begin{align}
    b &= (26.6 \pm 5.3)\, \mu\mathrm{m} \\
    \frac{b}{K} &= 0.21 \pm 0.04  \, .
\end{align}

\paragraph{Phase grating}~\\
The part of the experiment about the acousto-optic effect and the corresponding 
Raman-Nath theory did show the results theoretically expected. However, 
due to the already described shortcomings in the setup, the quality of the 
results is heavily reduced and fitting under the assumption of non-overlapping 
peaks yields quite inconsistent results. 

It is clear, that a modification to the setup would have been useful, 
namely the reallocation of the diodes to points where they would not interfere 
with the track along the beam path, such that the lens $l_3$ could have been placed 
correctly (in distance of focal length $30\,$cm to the diodes). 
