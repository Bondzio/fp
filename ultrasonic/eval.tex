\newcommand{\figdir}{analysis/figures/}

\section{Evaluation}
\subsection{Techniques used in the evaluation}
All calculations in this section are done with scripts written in 
the \textit{python} programming language~\cite{python}, relaying in several 
packages:
\begin{itemize}
    \item
        \textit{matplotlib}~\cite{Hunter2007} for plotting,
    \item
        \textit{scipy}~\cite{scipy} for fitting, and 
    \item
        \textit{uncertainties}~\cite{uc} for error propagation.
\end{itemize}
The latter applies gaussian error propagation for correlated and uncorrelated variables. 
We will thus not explicitly write down the formulas for the error propagation 
for each quantity calculated but instead state the numerical result, only. 
We will, however make a quick remark on the use of cavariance matrices in 
error propagation: Contrary to measured data, which in our case is usually 
expected to be uncorrelated, all fitted data yields variables that in general correlate. 
The propagation is then done as follows:
Let's assume we have random
variables $x_0,...,x_N$ which are correlatated through the $N\times N$ Matrix $cov(x_i,x_j)$.
For a scalar function $f(x_0,...,x_N) \rightarrow \mathbb{R}$, the variance is estimated (linearly) by:
\begin{equation}
Var[f] = \sigma^2 = \sum_{i,j} \frac{\partial f}{\partial x_i} \frac{\partial f}{\partial x_j} cov(x_i,x_j) \,.
\end{equation} 
If instead, $\mathbf{f}$ is a vector field in $m$ dimensions, namely 
$\mathbf{f}(x_0,...,x_N) \rightarrow \mathbb{R}^m$, then the components of $\mathbf{f}$ 
are further correlated. We can write down the relation between the covariance matrices $V$ and $U$ of 
$\mathbf{x}$ and $\mathbf{f}$, respectively, in matrix relations:
\begin{equation}
    U = A V A^T
\end{equation}
where $A$ is the matrix defined by 
\begin{equation}
    A_{ij} = \left[ \frac{\partial f_i}{\partial x_j}\right]_{\mathbf{x} = \mathbf{\mu}}
\end{equation}
with expectation value $E[\mathbf{x}] = \mathbf{\mu}$.~\cite{cowan1998statistical}

\subsection{Lattice constant of sine grating}
Observing the intensity on the screen, we noticed that the beam was not 
diffracted in the plane $y = const.$ (where y denotes the horizontal direction). 
We thus notated the coordinates on the graph paper. The measured values 
are displayed in table \ref{tab:sine_distances}. The estimated error of 
$s_{xy} = 1$ mm in both directions mainly stems from the fact that the edge of the beam 
has only a limited sharpness. 
\renewcommand{\arraystretch}{1.5}
\begin{table}[htdp]
    \centering
    \begin{tabular}{|p{6.18cm}|p{3.82cm}|p{3.82cm}|}
        \hline
        \rowcolor{LightCyan}
        Maximum & $x$ / mm & $y$  / mm \\ \hline
        $0.$        & 3     & 3\\
        $1.$, left & 48    & -1\\
        $1.$, right  & -42   & 7\\
        \hline
    \end{tabular}
    \caption{
        Measurements of positions of maxima for the sine amplitude grating. 
        The error is estimated to be $s_{xy} := s_x = s_y = 1$ mm. 
        }
    \label{tab:sine_distances}
\end{table}
The distance $d_\mathrm{sin}$ between the zeroth and first maximum is calculated by taking half of 
the distance between the two first maxima:
\begin{align}
    d_\mathrm{sin}   &= \frac{1}{2} \sqrt{(x_l - x_r)^2 + (y_l - y_r)^2} = 45.1 \mathrm{mm} \\
    \begin{split}
        s_{d_\mathrm{sin}} &= \sqrt{\sum{
                \left(\frac{\partial d_\mathrm{sin}}{\partial x_i}\right)^2 s_{x_i}  + 
                \left(\frac{\partial d_\mathrm{sin}}{\partial y_i}\right)^2 s_{y_i} 
                }} \\
        &= \frac{s_{xy}}{\sqrt{2}} \\
        &= 0.7 \, \mathrm{mm}\, ,
    \end{split}
\end{align}
where $x_r, x_l$ and $y_r, y_l$ correspond to the coordinates of right and left 
maximum, respectively and $s_{d_\mathrm{sin}}$ denotes the error calculated by gaussian error propagation.
The distance between screen and grating was $l_\mathrm{sin} = (56 \pm 2)$ mm. By geometric construction, 
we can identify the angle $\theta$ between the lines connecting grating and the maxima 
of zeroth and first order on the screen, respectively. Using further the 
necessary condition for positive interference (see eqn \eqref{eq:inter_cond} in the theory section), 
we calculate the lattice constant $K_\mathrm{sin}$ for the sine grating:
\begin{align}
    \sin(\theta)&= \frac{d_\mathrm{sin}}{\sqrt{l_\mathrm{sin}^2 + d_\mathrm{sin}^2}} \\
    \sin(\theta)&= \frac{m\lambda}{K_\mathrm{sin}} \\
    \Rightarrow \qquad 
    K_\mathrm{sin}    &= m \lambda\sqrt{\left(\frac{l_\mathrm{sin}}{d_\mathrm{sin}}\right)^2 + 1} 
        = (1.01 \pm 0.02) \, \mu\mathrm{m} \\
\end{align}
Although no nominal value is stated, the measured value suggests, that the grating has been 
constructed to a nominal value of $K_\mathrm{sin, nom} = 1$ $\mu$m. 

\subsection{Calibration}
For all further measurements, we needed to calibrate the setup and use the gauge gratings 
to establish the relationship between the signal observed on the oscilloscope and the 
actual distance between the peaks which would be observed on a screen at the position of diode 1. 
For this part, we set the lenses such that the beam would be widened and colliminated. We tested the collimination 
with the graph paper. The measured distances can be seen in the handwritten records, see appendix \ref{sec:records}.
The small focal lenght of lens 3 ($f_3 = 300$ mm) made it impossible to focus the beam exactly onto diode 1. We 
chose the closest position allowed by the setup, such the the lense would not interfere with the second beam. 
The only distance used in for further calculations is the distance $l$ between the grating and the diode, 
which is given by 
\begin{equation}
    l = \left[(297 \pm 3) + (140 \pm 3)\right] \, mathrm{mm}
    = (437 \pm 4) \, mathrm{mm} \, .
\end{equation}

