\clearpage
\subsection{Ultrasonic evaluation}
As already stated we have measured the intensity distribution of the laser
passing through a phase grating, in particular the phase grating created
by ultrasonic waves passing through the medium. We evaluate these diffraction patterns
by determining the position of the maxima and their intensities with an algorithm
(see figure~\ref{fig:ultrasonic1}) by extremal analysis. The error of this analysis
is given by the width of the respective peaks and the error of the oscillscope 
\textit{Hameg HM1508-2} is given in the used resolution by
\begin{equation}
S_U = 3 \% \times 1\mathrm{V} = 0.03 \textrm{V}
\end{equation}
The conversion from time to angle was given in the previous section (see
figure\ref{fig:calibrate_fit} for the conversion fit). In the following we 
will give all information in the captions of the figures and tables, respectively.
We will give the least square fits for the maxima from order zero to three and 
their coefficients along with the covariance matrices.

\begin{figure}[H]
\label{fig:ultrasonic1}
\centering
    \begin{subfigure}[b]{\picwidth}
        \includegraphics[width=1.0\textwidth]{analysis/figures/raman_001}
        \caption{}
        \label{fig:raman_001}
    \end{subfigure}
    \begin{subfigure}[b]{\picwidth}
        \includegraphics[width=1.0\textwidth]{analysis/figures/raman_007}
        \caption{}
        \label{fig:raman_007}
    \end{subfigure}
    \begin{subfigure}[b]{\picwidth}
        \includegraphics[width=1.0\textwidth]{analysis/figures/raman_014}
        \caption{}
        \label{fig:raman_014}
    \end{subfigure}
    \begin{subfigure}[b]{\picwidth}
        \includegraphics[width=1.0\textwidth]{analysis/figures/raman_020}
        \caption{}
        \label{fig:raman_020}
    \end{subfigure}
    \caption{This figures constitute a selection of the available measurements in the ultrasonic experiment. We denoted the 
    different orders by different symbols (see the respective legends). The maxima were estimated by an algorithm which seeks for local extrema
    and evaluates them with a certain threshold. The threshold is determined by the minimal distance of maxima, estimated by the theoretical 
    relationship $sin(\theta) = m \lambda / \Lambda$ (See the text for more explanations). The intensities are given by the measured Voltage,
    normed by the zeroth Maximum at $U=0V$.}\label{fig:raman}
\end{figure}
\newpage
\begin{figure}
    \centering
    \includegraphics[width=1\textwidth]{analysis/figures/besselfit_000}
    \caption{The measurements and the least square fit with the maxima of order zero. 
    The errorbars are given by 3\% of the extent of the oscilloscope,
    while the light green surface is the upper and lower limit of the fitted function, given
    by the covariance matrix of the coefficients. We fit the data with the function derived in the theoretical
    section, but extent it by an factor $c$ in order to take into account an additional (constant) noise
    influencing the photodiodes.\\
    $\Rightarrow$ We notice the agreement of our measurements with the Raman Nath theory. Using the
    coefficient $\alpha$ from the fit we will calculate the sonic wave length in the following section.}
    \label{fig:besselfit_000}
\end{figure}
\begin{SCfigure}
\caption{
The covariance matrix of the coefficients, calculated by the least square method. The off-diagonals are
one order of magnitude less than the errors, except for the releation between $A$ and $c$, which accounts for the
uncertainty of the estimated background. Keeping in mind that an higher background would lead to a smaller
amplitude (given the same data) we notice that the absolute error of $A$ and $c$ is much higher than the error for $\alpha$.
}
 \begin{tabular}{|r|r|r|r|}
 \hline 
\cellcolor{tabcolor}&\cellcolor{tabcolor}$\alpha$&\cellcolor{tabcolor}$A$&\cellcolor{tabcolor}$c$\\ \hline 
 \cellcolor{tabcolor}$\alpha$&$0.00010$ &$-0.00028$ &$0.00036$ \\ 
\cellcolor{tabcolor}$A$&$-0.00028$ &$0.00132$ &$-0.00130$ \\ 
\cellcolor{tabcolor}$c$&$0.00036$ &$-0.00130$ &$0.00151$ \\ \hline \hline
\cellcolor{tabcolor}$\alpha$&\multicolumn{3}{r|}{$0.21839 \pm 0.00989$ }\\ 
\cellcolor{tabcolor}$A$&\multicolumn{3}{r|}{$0.90833 \pm 0.03628$ }\\ 
\cellcolor{tabcolor}$c$&\multicolumn{3}{r|}{$0.11427 \pm 0.03881$ }\\ 
\hline\end{tabular}
\end{SCfigure}

\newpage

\begin{figure}[htpb]
    \centering
    \includegraphics[width=1\textwidth]{analysis/figures/besselfit_001}
    \caption{The measurements and the least square fit with the maxima of order zero, where
    the maxima were calculated with $I = (I_{\mathrm{left}} +  I_{\mathrm{right}})/2$.
    We do not want to repeat the considerations done before, see for figure~\ref{fig:besselfit_000}.
    Additionally we notice that the method estimating the maxima is encumbered by the 
    bad resolution of the peaks itself, such that they are not seperated as they should. 
    Following from this we receive a huge systematic error, not being reflected in the
    statistical errors resulting from the least squares fit. \\
    $\Rightarrow$ The important characteristic value $\alpha$ disagrees with the result of the maxima of zeroth order, which
    should in theory be the same. Noting that the $\chi^2$ test is lower then 1 (which would rather indicate that
    the given errors are too small) we remark that the systematic errors have an huge impact on the result, reflecting
    also in the disagreement within the paramters $A$ and $c$. 
}
    \label{fig:besselfit_001}
\end{figure}


\begin{SCfigure}
\caption{
As before only $A$ and $c$ are coupled, which is also visible in the
bigger error. Since $\alpha$ is (according to the least squares fit)
indepent of the others, the systematic error described before (see caption of figure~\ref{fig:besselfit_001}) 
cannot depend on the Amplitude or a constant background noise alone, but has to be related to the increased curvature
visible in an higer $\alpha$.}
 \begin{tabular}{|r|r|r|r|}
 \hline 
\cellcolor{tabcolor}&\cellcolor{tabcolor}$\alpha$&\cellcolor{tabcolor}$A$&\cellcolor{tabcolor}$c$\\ \hline 
 \cellcolor{tabcolor}$\alpha$&$0.00006$ &$-0.00004$ &$-0.00000$ \\ 
\cellcolor{tabcolor}$A$&$-0.00004$ &$0.00532$ &$-0.00110$ \\ 
\cellcolor{tabcolor}$c$&$-0.00000$ &$-0.00110$ &$0.00029$ \\ \hline \hline
\cellcolor{tabcolor}$\alpha$&\multicolumn{3}{r|}{$0.30305 \pm 0.00792$ }\\ 
\cellcolor{tabcolor}$A$&\multicolumn{3}{r|}{$0.82599 \pm 0.07293$ }\\ 
\cellcolor{tabcolor}$c$&\multicolumn{3}{r|}{$0.25782 \pm 0.01709$ }\\ 
\hline\end{tabular}
\end{SCfigure}
\newpage
\begin{figure}[htpb]
    \centering
    \includegraphics[width=1\textwidth]{analysis/figures/besselfit_002}
    \caption{This is the fit of the maxima of second order. For
    introductory remarks please see figure~\ref{fig:besselfit_000} and figure~\ref{fig:besselfit_001}.
    
    }
    \label{fig:besselfit_002}
\end{figure}

\begin{SCfigure}
\caption{
BLABLABALBLABLABLAB
BLABLABALBLABLABLAB
BLABLABALBLABLABLAB
BLABLABALBLABLABLAB
BLABLABALBLABLABLAB
BLABLABALBLABLABLAB
BLABLABALBLABLABLAB
}
 \begin{tabular}{|r|r|r|r|}
 \hline 
\cellcolor{tabcolor}&\cellcolor{tabcolor}$\alpha$&\cellcolor{tabcolor}$A$&\cellcolor{tabcolor}$c$\\ \hline 
 \cellcolor{tabcolor}$\alpha$&$0.00037$ &$0.00028$ &$-0.00013$ \\ 
\cellcolor{tabcolor}$A$&$0.00028$ &$0.00992$ &$-0.00137$ \\ 
\cellcolor{tabcolor}$c$&$-0.00013$ &$-0.00137$ &$0.00029$ \\ \hline \hline
\cellcolor{tabcolor}$\alpha$&\multicolumn{3}{r|}{$0.36608 \pm 0.01934$ }\\ 
\cellcolor{tabcolor}$A$&\multicolumn{3}{r|}{$0.82888 \pm 0.09961$ }\\ 
\cellcolor{tabcolor}$c$&\multicolumn{3}{r|}{$0.13956 \pm 0.01695$ }\\ 
\hline
\end{tabular}
\end{SCfigure}

