\section{Introduction}

Diffraction has long been a known phenomenon, dating back to Newton's 
experiments with prisms and the first diffraction grating being discovered 
during the same period by James Gregory.%
\footnote{%
The first experiments were done with feathers -- citing James Gregory:
"Let in the sun's light by a small hole ta a darkened house, and at the hole place 
a feather, (the more delicate and white the better for this purpose,) 
and it shall direct to a white wall or paper opposite to it a number of small circles 
and ovals, (if I mistake them not,) whereof one is somewhat white, 
(to wit, the middle, which is opposite to the sun,) 
and all the rest severally coloured."~\ref{rigaud1841correspondence}
}% 
They are also subject to several experiments already performed during 
prior laboratory works. This experiment will again focus on diffraction 
on gratings but going further into details. Part of the scope is to 
get an understanding of the close relationship between the aperture function 
and the intensity distribution of light after passing the grating, 
which is based on the mathematical concept of Fourier transformations. 
The grating in this part act on the amplitude of the incoming beam, 
corresponding to an aperture function of real values.

A second part of the experiment introduces the field of acousto-optics, 
where the interactions of sound and light waves are studied. Here, 
diffraction is caused by the dependence of the refractive index on 
density. In contrast to the optical gratings described above, 
the arising gratings modulate only the phase. The so-called phase gratings
are represented by aperture functions of complex values and 
modulus one. 


