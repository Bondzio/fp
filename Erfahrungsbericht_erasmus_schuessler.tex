\documentclass[a4paper,12pt]{article}
\usepackage{amssymb} % needed for math
\usepackage{amsmath} % needed for math
\usepackage[utf8x]{inputenc} % this is needed for umlauts
\usepackage[ngerman]{babel} % this is needed for umlauts
\usepackage[T1]{fontenc}    % this is needed for correct output of umlauts in pdf
\usepackage[margin=2.5cm]{geometry} %layout
\usepackage{fancyhdr}  % needed for the footer
\usepackage{lastpage}  % needed for the footer
\usepackage{hyperref}  % links im text
\usepackage{color, colortbl}  % farbige Tabellenzellen
\usepackage{tabularx}
\clubpenalty  = 10000 % Schusterjungen verhindern
\widowpenalty = 10000 % Hurenkinder verhindern
 
%%%%%%%%%%%%%%%%%%%%%%%%%%%%%%%%%%%%%%%%%%%%%%%%%%%%%%%%%%%%%%%%%%%%%
% Hier eigene Daten einfügen                                        %
%%%%%%%%%%%%%%%%%%%%%%%%%%%%%%%%%%%%%%%%%%%%%%%%%%%%%%%%%%%%%%%%%%%%%
\newcommand{\Jahr}{2013 / 2014}          % Typ: "2011 / 2012" oder "2012"
\newcommand{\Semester}{} % "Wintersemester" oder "Sommersemester"
\newcommand{\Datum}{\today}            % Wann wurde der Bericht erstellt?
\newcommand{\Semesteranzahl}{5}        % Das Fachsemester als Zahl
\newcommand{\Gesamtsemesterzahl}{6}    % Die gesamte Anzahl an Semestern
\newcommand{\Abschluss}{B. Sc.}
\newcommand{\Studienfach}{Physik}
\newcommand{\University}{Universität Freiburg}
\newcommand{\Nachname}{Schuessler}
\newcommand{\Vorname}{Friedrich}
\newcommand{\Strasse}{Schwarzwaldstr. 19}
\newcommand{\Hausnummer}{19}
\newcommand{\PLZ}{79110}
\newcommand{\Ort}{Freiburg}
\newcommand{\Email}{f.schuessler@posteo.de}
%%%%%%%%%%%%%%%%%%%%%%%%%%%%%%%%%%%%%%%%%%%%%%%%%%%%%%%%%%%%%%%%%%%%%
\hypersetup{ 
  pdfauthor   = {\Vorname~\Nachname}, 
  pdfkeywords = {Erasmus; Tor Vergata; \Vorname~\Nachname}, 
  pdftitle    = {Erfahrungsbericht Erasmus ~\Vorname~\Nachname~-~\Jahr} 
} 
 
\pagestyle{fancy}
\fancyhf{}
\renewcommand{\headrulewidth}{0pt}
\renewcommand{\footrulewidth}{0pt}
\fancyfoot[R]{Seite~\thepage~von \pageref{LastPage}}
 
\definecolor{LightCyan}{rgb}{0.88,1,1}
 
\pagenumbering{arabic}
 
\begin{document}
 
\title{Erfahrungsbericht Erasmus \Jahr}
\author{\Vorname \Nachname}
\date{\Datum}
 
\section*{Erfahrungsbericht Erasmus 2013/14}
\begin{tabularx}{\textwidth}{@{}llllX}
Name, Vorname:   & \Nachname, \Vorname \\
Partnerhochschule:	& Università Roma „Tor Vergata“, Italien  \\
Studienfach:         & \Studienfach, \Abschluss \\
Studienjahr:	& \Jahr \\
Betreuungspersonen: & \\
in Freiburg: & Dr. Wolfgang Kamke, kamke@physik.uni-freiburg.de \\
in Rom:	& Dr. Laura Calconi, laura.calconi@uniroma2.it \\
		& Prof. Dr. Anna Di Ciaccio, anna.diciaccio@roma2.infn.it \\
\end{tabularx}
 
%%%%%%%%%%%%%%%%%%%%%%%%%%%%%%%%%%%%%%%%%%%%%%%%%%%%%%%%%%%%%%%%%%%%%
% Hier bitte Text einfügen!                                         %
%%%%%%%%%%%%%%%%%%%%%%%%%%%%%%%%%%%%%%%%%%%%%%%%%%%%%%%%%%%%%%%%%%%%%
\subsection*{Vorbereitung und Ankunft}
Die Entscheidung, nach Italien zu gehen, war zu Beginn des Erasmus-Aufenthalts noch nicht sehr alt. Daher hatte ich lediglich ein Semester Sprachkurs hinter mir und musste mich auf meine Spanischkenntnisse und einen schnelles Einleben verlassen. Auch auf anderen Ebenen war der Anfang alles andere als fest durchgeplant: Welche Vorlesungen wann angeboten wurden, war im Internet nicht wirklich zu erkennen, und die Wohnungssuche musste ebenfalls warten, bis ich vor Ort war. In allen drei Fällen haben sich die anfänglichen Schwierigkeiten jedoch schnell beheben lassen. \\
Den Umzug nach Rom habe ich mit einem Zug bestritten – das dauert zwar eine Weile, gibt aber schon einmal einen schönen Eindruck von Italien. Ansonsten gibt’s günstige Flüge. \\
Am Anfang muss man eine Menge Papierkram erledigen. Dazu lohnt es sich, alle Dokumente gesammelt mitzunehmen und immer ein paar Kopien des Personalausweises dabeizuhaben.

\subsection*{Sprache}
Für mich waren die ersten Monate auf Grund der geringen Vorkenntnisse geprägt von einem enormen Lernzuwachs und vielen Verständnisproblemen. Von der Uni in Rom wurde ein Sprachkurs angeboten, der gerade in der Anfangszeit sehr viel weitergeholfen hat, später weniger intensiv wurde und vor allem dafür gesorgt hat, das ich viele von den anderen Erasmus-Student\_innen kennengelernt habe. Die Vorlesungen waren alle auf Englisch; allerdings gibt es auch einige, die auf Englisch gehalten werden. Gerade bei den zum Teil sehr kleinen Mastervorlesungen bieten manche Professor\_innen auch an, auf Englisch zu unterrichten.  

\subsection*{Wohnen}
Von der Universität wurden Wohnungen in einem Wohnheim angeboten, die ich jedoch nicht in Anspruch genommen habe. Anstatt dessen habe ich ein paar Tage in einem Hostel verbracht und auf dem freien Markt nach Alternativen gesucht, da ich gerne mit Italiener\_innen zusammenleben wollte. In Rom ist der Wohnungsmarkt etwas entspannter, sodass ich viele Angebote hatte und schon nach 3 Tagen eine Wohnung hatte. Zur Wohnungssuche gibt es folgende Möglichkeiten: \\
\begin{itemize}
\item der Erasmus-Verein „Erasmus in Campus“, der auch die Begrüßung und Sprachkurse anbiete, hilft bei der Unterbringung in WGs oder Wohnungen (viele WGs entlang der U-Bahn Linie A)
\item Kleinanzeigen in der Porta-Portese – gedruckt oder online: www.portaportese.it
\item online: \begin{itemize}
	\item http://www.bakeca.it
	\item http://www.easystanza.it
	\item http://www.idealista.it
	\end{itemize}
\item Aushänge in der Uni (vor allem Wohnungen in der Umgebung der Uni) 
\end{itemize}
Bei den letzteren drei Möglichkeiten wird man meistens gleich damit konfrontiert, alles auf Italienisch zu regeln – eine gute Kommunikation auf Englisch ist eher ein Glücksfall. Viele wollen keinen Vertrag, der weniger als ein Jahr läuft – aber Nachfragen lohnt sich trotzdem… Oft ist es so, dass die Wohnungssuche über den/die Vermieter\_in geregelt wird. WG-Besichtigungen oder gar Castings sind nicht üblich, sodass man auf dem freien Markt oft einen nicht sehr gemeinsamen WG-Alltag antrifft. Relativ üblich ist in Italien auch, ein Zimmer zu teilen. Wer das nicht will, kann sämtliche mit „Posto Letto“ oder „Doppio“ überschriebenen Angebote überspringen. Bei „Erasmus in Campus“ kann man angeben, ob man mit anderen Erasmusstudent\_innen oder Italiener\_innen zusammenleben möchte. Das Wohnheim, der „Campus X“, liegt noch 3 km weiter außerhalb der Stadt als die Uni selbst und ist damit besonders abends sehr schwer zu erreichen. \\
Die Mietpreise sind in Rom fast immer über denen Freiburgs. Wer eine Wohnung ohne direkte Metro-Anbindung findet, zahlt mit Nebenkosten zwischen 320 – 350 EUR, wer in der Nähe der „Linea A“ der Metro wohnen möchte, kommt wohl kaum unter 400 EUR. Doppelzimmer sind deutlich günstiger zu haben. Die Preise sind meistens ohne Nebenkosten angegeben; ich habe wohl ca. 80 EUR monatlich draufgezahlt. Viele Zimmer sind schon (karg) möbliert, sodass man für die kurze Zeit keine großen Anschaffungen machen muss. 

\subsection*{Studieren an der Tor Vergata}
Ich habe in den beiden Semestern vor allem Kurse belegt, die es so in Freiburg nicht gibt. Wirklich in den Studienplan hätten lediglich Meccanica Statistica 1 (als Theo 5) und Elementi di Fisica Nucleare e Subnucleare (als Ex 5) gepasst. Leider wird Mecc. Statistica 1 im Sommersemester (SS) und der zweite Teil im Wintersemester (WS) angeboten. Den möglichen Ersatz für Ex 5 habe ich auch nicht gehört, da mir von anderen Studierenden dringend davon abgeraten wurde, bei der entsprechenden Professorin eine Vorlesung zu hören. Zudem geben  die angegebenen Vorlesungen allesamt weniger ECTS Punkte als die Vorlesungen in Freiburg, sodass eine Anrechnung nicht mit Sicherheit garantiert ist (zu diesem Zeitpunkt weiß ich noch nicht, was mir angerechnet wird). 
Der Stil der größeren Vorlesungen mit ca. 30 Studierenden hat mir insgesamt nicht sehr gefallen. Diese Vorlesungen (für mich waren das Mecc. Statistica 1 bei Luca Biferale, SS, und Mecc. Quantistica 2 bei Emanuele Pace, WS) bestehen meistens aus 90 Minuten Monolog ohne viel Zwischenfragen, sowohl von Seiten der Lehrenden als auch der Studierenden. Dazu gibt es immer mal wieder Übungen, die allerdings einen anderen Stellenwert haben als in Freiburg: Meistens sind es relativ kleine Rechenaufgaben, die während der Vorlesung gestellt werden (am Anfang habe ich das oft nur im Nach-herein von meinen Mitstudent\_innen erfahren), und an anderer Stelle von einem Assistenten vorgerechnet werden. Die eigenen Ausarbeitungen werden nicht kontrolliert. Einen großen Stellenwert haben Klausuren und Zwischenprüfungen, die die Vorlesungen in viele kleine Abschnitte unterteilen und dazu führen, dass die Bibliothek voll ist mit Leuten, die die entsprechenden Kapitel in den vorgegebenen Büchern durchlesen und so von einer Note zur nächsten hetzen. \\
Die kleineren Vorlesungen, insbesondere solche aus dem Masterbereich, haben mir deutlich besser gefallen. Oft sind nur drei bis sechs Personen pro Vorlesung anwesend, und das Verhältnis zwischen Studierenden und Professor\_in ist deutlich entspannter – es gibt mehr Rückfragen und Möglichkeiten, selbst kleine Projekte zu machen. Ich habe hier eine Vorlesung zu numerischer Simulation von Differentialgleichungen gehört („Modullistica Numerica“ bei Chiara Cagnazzo, WS), die sich vor allem an Klimasimulationen orientiert hat, eine Vorlesung zu deterministischem Chaos und dem Verhalten dynamischer Systeme (Sistemi Dinamici bei Roberto Benzi, SS) sowie eine Vorlesung zu Klimatologie („Climatologia“ bei Federico Fierli, SS), die sich ebenfalls zum Großteil mit Simulation und Auswertung von Klimadaten befasste, aber mehr auf einen Überblick der Disziplin und weniger in die technischen Details ging als „Modullistica Numerica“. Zusätzlich habe ich eine Vorlesung zu algebraischer Topologie aus der Mathe gehört („Geometria 3“ bei Vincenzo di Gennaro, WS), die zwar relativ groß aber gut strukturiert war, und am „Laboratorio di Calcolo Numerico e Informatica“ (Francesco Berilli, WS) teilgenommen, welches eine Einführung ins wissenschaftliche Programmieren in den Sprachen C und Fortran gibt. Hier war zwar die Vorlesung nicht so gut – sie bestand aus einem Ablesen von Folien – aber die Übungen waren gut für mich, da ich vorher noch keine Erfahrung mit Programmierung gemacht hatte. \\
Die Auswahl der Kurse habe ich vor Ort und zu Beginn des Semesters getroffen. Oft überschneiden sich Kurse, wenn man nicht dem festen Schema der italienischen Studienverlaufspläne folgt. Für die Kurse muss man sich in Rom erst kurz vor er Prüfung offiziell anmelden und die Änderungen des Learning Agreements waren auch kein Problem.

\subsection*{Was die Uni sonst so bietet}
Nicht viel. Es gibt einige Konzerte, allerdings insgesamt deutlich weniger Rahmenprogramm als das an der Uni in Freiburg der Fall ist. Das Wohnheim hat ein Angebot für diverse Sport- und Tanzkurse. Das größte Angebot findet sich jedoch in der Stadt – dort findet man von Sport über Musik und Tanz alles. Diese Kurse kosten natürlich – ich habe beispielsweise für einen Tanzkurs 30 Euro im Monat bezahlt – aber sie lohnen sich nicht zuletzt, um Leute auch außerhalb der Uni kennenzulernen. Oft lohnt es sich, unter den Studieren mal nachzufragen, was die so machen – durch ein paar Zufälle bin ich so an die Dinge gekommen, die mir in dem Jahr am meisten Spaß gemacht haben. \\
Wer die Stadt Rom bis in die letzten Ecken kennenlernen möchte, hat viele Möglichkeiten: Vom Erasmus gibt es immer wieder Führungen, die entweder kostenlos sind oder sehr wenig kosten. Für mich hat es auch sehr gut funktioniert, die Stadt auf eigene Faust zu durchstreifen, mir einzelne Ziele im Stadtführer rauszusuchen oder auf gut Glück unterwegs zu sein. Einige der interessantesten Ecken habe ich erst im späteren Verlauf des Jahres von Römer\_innen gezeigt bekommen. 

\subsection*{Technische Details}
Ich habe mich während des Jahres auf meine Visa-Card und Onlinebanking verlassen und bin damit sehr gut zurecht gekommen. Das einzige, was ich damit nicht machen konnte, war einen Internet-Vertrag abzuschließen – das haben dann meine Mitbewohner übernommen. Ich habe im Vorfeld dabei allerdings herausgefunden, dass es durchaus kostengünstige oder gar -lose Möglichkeiten gibt, ein Konto zu eröffnen (viele Italiener\_innen behaupteten das Gegenteil). 
Handykarten gibt’s von verschiedenen Anbietern, die Konditionen unterscheiden sich kaum und sind recht günstig. Für Gespräche nach Deutschland habe ich vor allem Skype benutzt. Allerdings sind auch dort die Konditionen heute mit einer deutschen Simkarte nicht schlecht. \\
Ohne öffentlichen Nahverkehr kommt man in Rom kaum zurecht, auch wenn dieser immer mal wieder streikt. Es gibt sage und schreib zwei U-Bahnlinien für die tagsüber bis zu 5 Mio. Mensch in der Stadt – allerdings geht die eine davon fast bis an die Uni Tor Vergata. Ich habe mir ein Jahresabo für den gesamten Nahverkehr gekauft (250 EUR), die Monatstickets kosten 35 EUR. Vergünstigungen für Studis gibt’s nicht. \\
Mit der italienischen Post ist nicht zu Spaßen! Wer ein Päckchen ohne Sendungsverfolgung verschickt oder bekommen soll, muss damit rechnen, dass es nicht ankommt. Also: Kein Geld oder Wertgegenstände verschicken… Ansonsten kommen Briefe und Postkarten meist nach gut einer Woche an. Wichtige Dokumente habe ich meistens gefaxt – die Erasmusbeauftragte Laura Calconi ist da sehr zuvorkommend. 

\subsection*{Essen und Trinken}
Großartig! Jedenfalls dann, wenn ich Essen gegangen bin oder in einer italienischen Familie gegessen habe. Natürlich lassen sich die Restaurants in Rom auch gut bezahlen – darauf sollte man eventuell mit der Zeit achten, zumal auch die jungen Leute nicht vor Preisen zurückschrecken, die über typischem deutschen Studierendenniveau liegen. Und dann noch ein kleiner Haken: Die Verpflegungssituation an der Uni ist nicht gerade optimal. Wer günstig Mittagessen will, kann ich die 2 km entfernte Mensa mit dem Bus fahren (10 min) und bekommt dort nach einer Anmeldung für ca. 3 EUR Mittagessen. Ansonsten gibt es an der beiden Bars warmes Essen für ca. 5 EUR, während viele Italiener\_innen in die benachbarte Einkaufsmeile gehen und dort essen (8 – 10 EUR). 

\subsection*{Abschließende Worte}
Alles in allem habe ich ein sehr schönes Jahr in Rom verbracht, viel von dem Land kennengelernt und einen guten Start in die Sprache bekommen. Ich hätte es wohl sehr bereut, wenn ich nur ein Semester dort geblieben – wirklich angekommen bin ich erst im Februar, nachdem ich in eine sehr schöne WG gezogen bin und die Sprache deutlich besser beherrschte und enger mit einigen Italiener\_innen befreundet war!

\end{document}
