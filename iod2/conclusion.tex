\section{conclusion}
We can conclude the experiment stating that it did, over all, give a good 
insight into the techniques of vibrational spectroscopy, the 
characteristic parameters of interest and the underlying 
theoretical models. 
It is quite fascinating, how many parameters 
can be deducted from the measured spectrum. 
On the other hand, this strong reduction to a small number of 
measured quantities leads to very instable results, as uncertainties 
add up during the error propagation. While some variables seem 
to be quite consistens with the givien data in literature, others 
are off theses values by more the twice the standard deviation. 
Some of these difficulties are connected to the external conditions, 
namely the setup and the spectroscopy having only a limited resolution. 
Others, as discussed, may be due to systematic errors such as a 
slightly off-focus arrangement of the beam path. As a third point, 
the question about how to estimate the uncertainty of the spectrometer 
remains open, with indications, that the chosen value is too high. 
At last, the theoretical assumptions are only valid to a certain order. 
It is, however, not within the scope of this report to assess the validity or precision of 
the assumptions made theoretically.



