\subsection{Overview}
In the following experiment, we will analyse the spectral band of absorbtion of the iodine 2 molecule 
with spectroscopal methods. For the given conditions, the expected observation stems almost 
entirely from one electronic transition, the 
\begin{equation}
    B ^3\Pi_{\sigma \, \mathrm{u}}^{+} \quad <- \quad X ^1\Sigma_{\sigma \, \mathrm{u}}^{+}
\end{equation}
transition. Since iodine is a relativly heavy di-atomic molecule, it has vibrational modes that 
split up the total energy of the molecule considerably and are visible in the spectrum with the 
resolution of the given spectrometer. The energy is further changed by rotational modes of the 
atoms, which however don't show up as separate lines but rather widen the observed bands, as the 
resolution is not high enough. 
The observed data is then used to calculated characteristical constants for the molecule, such as 
the dissipation energy $D_e$ and the mean radius $r_c$. 

\subsection{Historic review}
The structure of molecular iodine ($I_2$) has been studied various times in physical chemistry 
during the last century. One of the first thorough measurements of the spectral bands was done in 
1923 by the German physicist Reinhard Meck \cite{mecke1923bandenspektrum}. The vibrational states 
were correctly numbered for the first time by Loomis \cite{loomis1927correlation}, who analysed 
the fluorecence spectrum earlier measured by Wood \cite{wood1911} for $v = 26$.


