\subsection{Overview}
In the following experiment, we will analyse the spectral band of absorbtion of the iodine 2 molecule 
with spectroscopal methods in order to study the vibrational modes and basic properties 
of the diatomic molecule. 
For the given conditions, the expected observation stems almost 
entirely from one electronic transition, the 
\begin{equation}
     X ^1\Sigma_{\sigma \, \mathrm{u}}^{+} \quad \rightarrow \quad B ^3\Pi_{\sigma \, \mathrm{u}}^{+}
\end{equation}
transition. Since iodine is a relativly heavy diatomic molecule, it has vibrational modes that 
split up the total energy of the molecule considerably and are visible in the spectrum with the 
resolution of the given spectrometer. The energy is further changed by rotational modes of the 
atoms, which however don't show up as separate lines but rather widen the observed bands, as the 
resolution is not high enough. 
The observed data is then used to calculated characteristical constants for the molecule, such as 
the dissociation energy $D_e$ for ground and first excited energy level, as well as 
and the mean radius $r_e$. The calculated parameters are finally compared to literature values 
and used to plot an approximation to the real potential, the so-called Morse potential.  

\subsection{Historical review and general properties of the I2 molecule}
The structure of molecular iodine ($I_2$) has been studied various times in physical chemistry 
during the last century. One of the first thorough measurements of the spectral bands was done in 
1923 by the physicist Reinhard Meck \cite{mecke1923bandenspektrum}. The vibrational states 
were renumbered in an attempt to correct them by Loomis \cite{loomis1927correlation}, who analysed 
the fluorecence spectrum earlier measured by Wood \cite{wood1911} for $v = 26$. However, they were 
correctly numbered only in 1965 by Steinfeld~et.~al.~\cite{steinfeld1965spectroscopic}. 

Iodine is especially suitable for the study of vibrational modes. Due to its high mass of 
$m_{I} = 126.9$u~\cite{weisstein}, the separation between neighbouring vibrational modes 
is high enough to be resolved by 
the given spectrometer even at room temperature. Another useful property is the fact, that
iodine sublimates in sufficient amounts at room temperature and low pressures, even though 
the boiling point is at  $184.3^{\circ}$C~\cite{weisstein}. In the experiment, a pipe at 0.5 Torr = 6.6e-4 atm 
is filled with some grains of cristal iodine. Finally, its spectrum is comparatively easy 
to analyse, as iodine has only one stable isotope, namely $_53^127 I$, putting it in favor over 
other diatomic molecules having absorption spectra in a similar range of visible light, such 
as bromine and chlorine~\cite{staatsexamen}.
