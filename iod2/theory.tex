\subsection{Theoretical Foundations}
When describing molecules in the framework of quantum mechanics, we are faced with the problem of 
finding solutions to the non-relativistic and time-independent Schrödinger-equation
\begin{equation}
    \hat{H} \psi = \Big[- \frac{\hbar^2}{2m} \nabla^2 + V(\mathbf{r}) \Big] \psi = E \psi
\end{equation}
where $\psi$ are eigenfunctions of the hamilton operator $\hat{H}$.
To a few problems we can find the analytical solution to this equation, but to most
of the problems we need to approximate the solution with regards to an analyical
solution which is given. In the following sections we will introduce
the time-independent Perturbationtheory and give an example with the 
anharmonic oscillator how such an approximation can take place.
\subsubsection{Timeindependent Perturbationtheory}
We are following~\cite{fliessbach2008quantenmechanik}.
Lets assume we know the Eigenvalues and -functions of a specific
Hamiltonoperator $\hat{H_0}$ of the
unperturbed system (we further assume the eigenspaces to be
seperable, so we have no degeneracy. The case of degeneracy
can be done analogously) 
, numbered with $n\in \mathbb{N}$:

\begin{equation}
\hat{H_0} |n\rangle = \epsilon_n |n\rangle 
\end{equation}
and now we search for the solution of the perturbated Hamiltonian:
\begin{equation}
    \hat{H}|\psi \rangle = E|\psi \rangle 
\end{equation}
by means of the initial Hamiltonian:
\begin{equation}
    \hat{H} = \hat{H_0} + \lambda \hat{V}
\end{equation}
where we perturbate the original Hamiltonian with the Potential
$\hat{V}$ and give an approximate Soluation as long as Perturbation
is not of the same order of magnitude as $H_0$. 
The next step will be to apply this to the anharmonic 
Oscillator in order to get an estimation about the energyeigenvalues.
At first we have introduced a new parameter $\lambda$, which is the 
strength of the perturbation; for $\lambda = 1$ we have the problem
we want to solve, for $\lambda = 0 $ we arrive at the unperturbed
problem. Now we can expand in this parameter:
\begin{align}
    E_n(\lambda) &= \epsilon_n + \sum_{\nu =1}^{\infty} \lambda^\nu
    E_{n,\nu} \\ 
    |\psi_n(\lambda)\rangle &=
    |n\rangle + \sum_{\nu =1}^{\infty} \lambda^\nu
    |\psi_{n,\nu}(\lambda)\rangle
\end{align}
As result the solutions we are looking for will become:
\begin{align}
    E_n &= \epsilon_n + E_{n,1} + E_{n,2} + \cdots\\ 
    |\psi_n \rangle &= |n\rangle + |\psi_{n,1} \rangle + \cdots 
\end{align}
We assume for now that this series is converging and plug it into
the Schrödingerequation:
\begin{equation}
    (\hat{H_0} + \lambda \hat{V})
   \left( |n\rangle + \sum_{\nu =1}^{\infty} \lambda^\nu
        |\psi_{n,\nu}(\lambda)\rangle \right) =
\epsilon_n + \sum_{\nu =1}^{\infty} \lambda^\nu
    E_{n,\nu}  
    \left( |n\rangle + \sum_{\nu' =1}^{\infty} \lambda^{\nu'}
        |\psi_{n,\nu'}(\lambda)\rangle \right) 
\end{equation}
Now we can impose that the different powers of $\lambda$ have to
match on both sides and receive a set of equations:
\begin{align}
    \hat{H_0} |n\rangle &= \epsilon_n |n\rangle\\
    \hat{H_0} |\psi_{n,1}\rangle + \hat{V}|n\rangle &=
    \epsilon_n |\psi_{n,1}\rangle + E_{n,1}|n\rangle \label{eq:dev1}\\
    \hat{H_0} |\psi_{n,2}\rangle + \hat{V}|\psi_{n,1}\rangle &=
    \epsilon_n |\psi_{n,2}\rangle + E_{n,1}|\psi{n,1}\rangle 
    + E_{n,2}|n\rangle \label{eq:dev2}\
\end{align}
This set of equations yield recursive the Solutions of $E_{n,1}$,
$E_{n,2}$ and so on. Now we expand these single Wavefunctions 
into the new basis of energyeigenfunctions $|m \rangle$ with $m\in 
\mathbb{N}$:
\begin{equation}
    |\psi_{n,1}\rangle = \sum_{m=1}^{\infty} a_{nm} |m \rangle
    \label{eq:psi_expansion}
\end{equation}
We can plug this expansion now into into equation~\eqref{eq:dev1}:
\begin{equation}
    \sum_{m=1}^{\infty} (\epsilon_n - \epsilon_m) a_{nm} |m\rangle
    + E_{n,1} |n\rangle = \hat{V}|n \rangle
\end{equation}
With the Projection onto the dual eigenstate $\langle k|$ we can
use the orthonormality $\langle k | m \rangle = \delta_{m,k}$ 
of the energyeigenfunctions:
\begin{equation}
     (\epsilon_n - \epsilon_k) a_{nk} 
     + E_{n,1}\delta_{n,k}  = \langle k |\hat{V}|n \rangle
\end{equation}
For the case $k = n$ we get:
\begin{equation}
    E_{n,1} = \langle k |\hat{V}|n \rangle
\end{equation}
and for $k\neq m$:
\begin{equation}
    a_{n,k} = \frac{\langle k |\hat{V}|n \rangle}
    {\epsilon_n - \epsilon_k} \label{eq:a_nk}
\end{equation}
Equation~\eqref{eq:a_nk} together with~\eqref{eq:psi_expansion} 
gives us now:
\begin{equation}
    |\psi_{n}\rangle = |n\rangle + \lambda \sum_{m\neq n}^{\infty}
    \frac{\langle m |\hat{V}|n \rangle}{\epsilon_n - \epsilon_m}  
    |m \rangle + \lambda a_{n,n} + \mathcal{O}(\lambda^2)
\end{equation}
Now we can set $a_{n,n}=0$ (see footnote~\footnote{
Again we can use orthonormality:
\begin{equation*}
    1\overset{!}{=} \langle \psi_n|\psi_{n}\rangle 
    = \left | 1 + \lambda a_{nn} \right |^2 + \lambda^2
    \sum_{m\neq n}^{\infty}
    \left | \frac{\langle m |\hat{V}|n \rangle}
        {\epsilon_n - \epsilon_m} \right |^2 = 
    1 + \lambda (a_{n,n} + a^*_{n,n}) + \mathcal{O}(\lambda^2)
\end{equation*}
When $\lambda > 0 $ this results into:
\begin{equation*}
    a_{n,n} + a^*_{n,n} = 0 \Rightarrow \Re  [a_{n,n}] = 0
\end{equation*}
Since the Wavefunctions 
are invariant with respect to a phase $\phi$,
we can set without limitations $a_{nn}$ to zero.} for
further details) and we arrive
at:
\begin{equation}
    |\psi_{n}\rangle = |n\rangle + \lambda \sum_{m\neq n}^{\infty}
    \frac{\langle m |\hat{V}|n \rangle}{\epsilon_n - \epsilon_m}  
    |m \rangle  + \mathcal{O}(\lambda^2)
\end{equation}
For the Energy we can make use of the third equation and go to
the second order:
\begin{equation}
|\psi_{n,2}\rangle  = \sum_{m=1}^{X} b_{n,m} |m \rangle 
\end{equation}
We can plug this analogously into equation~\eqref{eq:dev2} and
arrive at:
\begin{equation}
 \sum_{m}^{\infty} (\epsilon_n - \epsilon_m) b_{nm} |m\rangle +
 \sum_{m}^{\infty} E_{n,1} a_{n,m} |m \rangle + E_{n,2} |n \rangle =
 \sum_{m}^{\infty} a_{n,m} \hat{V} |m \rangle
\end{equation}
Where again we use the orthonormality. For $k=n$ we get:
\begin{equation}
    E_{n,2} = \sum_{m}^{\infty} a_{n,m} \langle n |\hat{V}|m\rangle
= \sum_{m\neq n}^{\infty}\frac{|\langle n | \hat{V} | m \rangle |^2}
    {\epsilon_n - \epsilon_m}
\end{equation}
For $k\neq n$ we would arrive at the explicit formula for $b_{n,m}$
but this is not necessary for the further computations.
As final result we therefore get:
\begin{align}
    E_n &\approx \epsilon_n + \langle n|\hat{V} | n \rangle +
\sum_{m\neq n}^{\infty}\frac{|\langle n | \hat{V} | m \rangle |^2}
    {\epsilon_n - \epsilon_m} \\
    |\psi_n \rangle &\approx 
    |n\rangle + \sum_{m\neq n}^{\infty}
    \frac{\langle m |\hat{V}|n \rangle}{\epsilon_n - \epsilon_m}  
    |m \rangle 
\end{align}
\subsection{The anharmonic oscillator}
Lets assume we have solved the harmonic oscillator with 
\begin{align}
    \hat{H_0} &= \frac{\hat{p}^2}{2m} 
    + \frac{1}{2} m \omega_0^2\hat{x}^2 \\
    \hat{H_0}|n \rangle &= \epsilon_n |n \rangle \\
    \epsilon_n &= \hbar \omega_0 \left (n+ \frac{1}{2} \right)
\end{align}
Now perturbe this system with a kubic potential potential:
\begin{equation}
\hat{V} =\lambda \hat{x}^3
\end{equation}
Hence we can use the perturbation theory in linear order.
Since the parity of $\hat{V}$ is odd, the expectation value has 
to be zero:
\begin{equation}
    E_{n,1} = \lambda \langle n | \hat{x}^3 | n \rangle = 0 
\end{equation}
The wavefunctions have to be corrected as follows:
\begin{equation}
    |\psi_{n,1} \rangle = \sum_{m\neq n}^{\infty}
    \frac{\langle m |\hat{V}|n \rangle}{\epsilon_n - \epsilon_m}  
    |m \rangle 
    = \frac{\lambda}{\hbar \omega}
    |\psi_{n,1} \rangle = \sum_{m\neq n}^{\infty}
    \frac{\langle m |\hat{x}^3|n \rangle}{\epsilon_n - \epsilon_m}  
    |m \rangle 
\end{equation}
Now we have to calculate the matrix elements. Therefore we
introduce the creation- and annihilationoperator 
which are defined by $\hat{x}$:
\begin{equation}
    \hat{x} = \sqrt{\frac{\hbar}{2m\omega}}(\hat{a}^\dagger +
        \hat{a})
\end{equation} 
Now we can calculate iteratively the matrixelements
\footnote{
    Here we make use how $\hat{a}$ and $\hat{a}^\dagger$ act 
    on the eigenfunctions $|n\rangle$:
\begin{align*}
    \hat{x}|n'\rangle = \sqrt{\frac{\hbar}{2m\omega}}
    \left ( \sqrt{n' + 1}|n'+1\rangle 
        + \sqrt{n'}|n' - 1 \rangle \right ) 
\end{align*}
Applying once more:
\begin{align*}
\hat{x}^2|n'\rangle = \frac{\hbar}{2m\omega}
\left ( \sqrt{(n' + 1)(n' + 2)}|n'+2\rangle 
            + (2n' + 1) |n' \rangle
            + \sqrt{(n'(n'-1)}|n' - 2 \rangle \right ) 
\end{align*}
And applying for the third time:
\begin{align*}
    \hat{x}^3|n'\rangle &= \sqrt[3]{\left (\frac{\hbar}{2m\omega}\right )^2}
( \sqrt{(n' + 1)(n' + 2)(n'+3)}|n'+3\rangle 
    + 3 \sqrt{(n'+1)^3}|n' +1 \rangle  \\
 &+ 3 \sqrt{(n')^3}|n' -1 \rangle   
            + \sqrt{(n'(n'-1)(n'-2)}|n' - 3 \rangle  ) 
\end{align*}
Now this yields the matrixelements which we need:
\begin{align}
    \langle n | \hat{x}^3 | n' \rangle &=  
    \sqrt{\left (\frac{\hbar}{2m\omega}\right)^3}
( \sqrt{(n' + 1)(n' + 2)(n'+3)}\delta_{n,n'+3} 
    + 3 \sqrt{(n'+1)^3}\delta_{n,n'-1}  \\
    &+ 3 \sqrt{(n')^3}\delta_{n,n'-1}   
    + \sqrt{(n'(n'-1)(n'-2)}\delta_{n,n'-3}  ) 
\end{align}

}. After applying the justified indices, since only $n'=n\pm 1$ and
$m = n \pm 3 $ are not zero, we arrive at:
\begin{equation}
\begin{aligned}
    |\psi_{n,1} \rangle &= \frac{\lambda}{\hbar \omega}
    \sqrt{\left(\frac{\hbar}{2m\omega}\right)^3}
    ( -\frac{1}{3}\sqrt{(n + 1)(n + 2)(n+3)}|n+3\rangle 
    - 3 \sqrt{(n+1)^3}|n +1 \rangle  \\
 &+ 3 \sqrt{(n)^3}|n -1 \rangle  
    +\frac{1}{3} \sqrt{(n(n-1)(n-2)}|n - 3 \rangle  ) 
\end{aligned}
\end{equation}
We can also calculate the energy corrections:
\begin{equation}
    E_{n,2} = \langle n | \hat{V} | \psi_{n,1} \rangle 
    = \lambda \langle n | \hat{x}^3 | \psi_{n,1} \rangle 
\end{equation}
Which yields:
\begin{equation}
\begin{aligned}
    E_{n,2} &= \frac{\hbar^2 \lambda^2}{2 m^3 \omega ^4}
    \left[-\frac{1}{3}(n+1)(n+2)(n+3)
        -9(n+1)^3 + 9n^3 + \frac{1}{3}n(n-1)(n-2)
        \right ] \\
    &= -\frac{\hbar^2 \lambda^2}{2 m^3 \omega ^4}
    \left[ 30n^2+ 30n + 11 \right ]\\
    &= -\frac{30 \hbar^2 \lambda^2}{2 m^3 \omega ^4}
    \left[\left(n + \frac{1}{2} \right)^2 + \frac{11}{30} \right ]\\
\end{aligned}
\end{equation}
If we had a look at the closed form for all orders, we would
notice the possibility to expand the Energy Difference in terms
of powers of the original Energy $(n+\frac{1}{2})$ which we will
state here without proof:
\begin{equation}
    E_n = \hbar \omega_{0,0} \left(n + \frac{1}{2} \right) 
    - \hbar \omega_{0,1} \left(n + \frac{1}{2} \right)^2  
    + \hbar \omega_{0,2} \left(n + \frac{1}{2} \right)^3  
    + \cdots
\end{equation}
We did not include the small derivation independent of $n$,
since we will look only at energydifferences. Notice that
the final result is only valid for a small, odd Perturbation. 
\subsubsection{Energylevels and Wavefunctions
    of two-atomic molecules}
First we will write down the full Equations of the two-atomic
molecule and investigate after which parts we can further
simplify:
\begin{equation}
        \hat{H}\psi = E\psi 
\end{equation}
Where we can expand the Hamiltonian in the position-space,
where we already split into the electrons ($i$) and the 
two nucleons $A$ and $B$:
\begin{equation}
    \hat{H} = \frac{-\hbar^2}{2} 
        \underbrace{\left(
        \sum_{i}{\frac{\nabla_i^2}{m_e}}
        +\frac{\nabla_A^2}{M_A} +\frac{\nabla_B^2}{M_B}
\right)}_{
\substack{\text{kinetic energy}\\\text{of electrons and nucleons}}}
+ \underbrace{\sum_{i>j}{\frac{e^2}{|r_i - r_j|}}
    }_{\substack{\text{potential energy}\\\text{electron-electron}}}
 - \underbrace{\sum_{i}{\frac{Z_A e^2}{|r_i - r_A|}}
 }_{\substack{\text{potential energy}\\\text{electron-nucleon $A$}}}
 - \underbrace{\sum_{i}{\frac{Z_B e^2}{|r_i - r_B|}}
 }_{\substack{\text{potential energy}\\\text{electron-nucleon $B$}}}
 +  \underbrace{\frac{Z_A Z_B e^2}{|r_A - r_B|}
 }_{\substack{\text{potential energy}\\\text{nucleon $A$ - nucleon $B$}}}
\end{equation}
Now we split the electronsolutions and the nucleonsolutions such
that the solutions of the electrons change only little when the
distance of the nucleons change, which is known as the
\textsc{BORN-OPPENHEIMER}-Approximation \cite{staatsexamen}:
\begin{align}
    \psi &= \psi_E ( \cdots r_i \cdots) \psi_N(r_A, r_B) \\
    \hat{ H}_E\psi_E &= \left(- \sum_{i}{\frac{\hbar^2\nabla_i^2}{2m_e}}
    + \sum_{i>j}{\frac{e^2}{|r_i - r_j|}}
    - \sum_{i}\frac{Z_A e^2}{|r_i - r_A|}
    - \sum_{i}\frac{Z_B e^2}{|r_i - r_B|}
    + 
    \right ) \psi_E = 
 E_E \psi_E \\
 \hat{H}_N\psi_N &= \left (
 -\frac{\hbar^2\nabla_A^2}{2M_A} -\frac{\hbar^2\nabla_B^2}{2M_B}
    - \sum_{i}\frac{Z_A e^2}{|r_i - r_A|}
    - \sum_{i}\frac{Z_B e^2}{|r_i - r_B|}
 + \frac{Z_A Z_B e^2}{|r_A - r_B|}
 \right ) \psi_N
=  E_N \psi_N 
\end{align}
Where we fixed the positions for the Hamiltonian
of the electrons such that they are not included in the
wavefunction.
