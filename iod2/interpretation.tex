\section{Evaluation}

\subsection{Measuring bias of spectrometer}
To be able to identify the absorption spectrum of the iodine 2 molecule, we have to analise 
the measurements for callibration with the Na- and Hg-vapour lamps, respectively. The measured
spectrum over the entire range of the spectrometer is plotted in figure \ref{fig:specturm_all}. 
To locate the maxima, we magnify the recordings in a region around the correspoding maximum. 
For sodium, this is done in figure \ref{fig:na_max}. One can see the two characteristic peaks 
at 
\begin{eqnarray*}
    \lambda_\mathrm{max, 1} = 589.1 \pm \unc \nm \\
    \lambda_\mathrm{max, 2} = 589.6 \pm \unc \nm.
\end{eqnarray*}
The uncertainty is half the value of 
resolution (6nm). It corresponds with the sharpness of the maxima 
(fwhm about 0.4nm as seen in the figures). 
These correspond with the values given in literature \cite{nist}, namely 
\begin{eqnarray*}
    \lambda_\mathrm{max, 1} = 589.0 \nm \\
    \lambda_\mathrm{max, 2} = 589.6 \nm.
\end{eqnarray*}
In the recorded spectrum, we also see another, smaller peak at 
\begin{equation}
    \lambda = 588.4 \pm \unc \nm.
\end{equation} 
This peak is not found in literature \cite{nist} and seems to be an artefact of the measurement. 
In fact, as the light from the Na lamp was not colliminated and focused very well, one could 
suspect diffration at the entrance of the spectrometer as the cause, as the spectrometer 
doesn't measure wavelength directly but rather the intensity corresponding to a certain angle. 

For the Hg lamp, we observe similar results. As seen in \ref{fig:spectrum_all}, intensities 
of the minima varied quiet drastically. For that reason, we took various measurements of which 
we can use one for the first maximum (fig. \ref{fig:hg1_max}) and another at much lower 
intensity for the three other ones (figs. \ref{fig:hg2_max}, \ref{fig:hg3_max}). 
We identify the following maxima:
\begin{eqnarray*}
    \lambda_\mathrm{max, 1} &=& 435.5 \pm \unc \ (435.83)\nm \\
    \lambda_\mathrm{max, 2} &=& 545.9 \pm \unc \ (545.9)\nm \\
    \lambda_\mathrm{max, 3} &=& 576.8 \pm \unc \ (576.8)\nm \\
    \lambda_\mathrm{max, 4} &=& 576.9 \pm \unc \ (576.9)\nm.
\end{eqnarray*}
The values in parenthesis cite literature values \cite{nist}. \\
As the spectrometer yields results lying within the literature values for the 
stated uncertainty, we conclude that there is no bias in the measured wavelength.

\begin{figure}
\centering
\includegraphics[width=\pltw]{analysis/figures/spectrum_all.pdf}
\caption{Measured spectrum of Na and Hg lamp}
\label{fig:spectrum_all}
\end{figure}

\begin{figure}
\centering
\includegraphics[width=\pltw]{analysis/figures/na_max.pdf}
\caption{Measured spectrum of the Na lamp, detail at the 
characteristic orange double line}
\label{fig:na_max}
\end{figure}

\begin{figure}
\centering
\includegraphics[width=\pltw]{analysis/figures/hg1_max.pdf}
\caption{Measured spectrum of the hg lamp, detail at first maximum}
\label{fig:hg1_max}
\end{figure}

\begin{figure}
\centering
\includegraphics[width=\pltw]{analysis/figures/hg2_max.pdf}
\caption{Measured spectrum of the hg lamp, detail at second maximum}
\label{fig:hg2_max}
\end{figure}

\begin{figure}
\centering
\includegraphics[width=\pltw]{analysis/figures/hg3_max.pdf}
\caption{Measured spectrum of the hg lamp, detail at thrid and fourth maximum}
\label{fig:hg3_max}
\end{figure}

\subsection{Spectrum of halogen lamp}
As we look out to measure the absorption spectrum of iodine, we first take a quick look at 
the specturm of the background from which photons are to be absorbed. The measured spectrum 
of the used halogen lamp within the range of wavelength in question is shown in figure 
\ref{fig:specturm_halogen_full}.
\begin{figure}
\centering
\includegraphics[width=\pltw]{analysis/figures/halogen_02.pdf}
\caption{Measured spectrum of the halogen lamp over wavelength 
for the entire spectral window, 
no correction for diffraction in light}
\label{fig:spectrum_halogen_full}
\end{figure}
To be able to compare this to the absorption spectrum, we firth need to make the same 
corrections for diffraction in light as we do for the latter. We do so by linearly 
interpolating given literature values of the diffraction index, given in table 
\ref{tab:diff_air}.

\begin{table}[h]
\centering
\begin{tabular}{| c |l|l|l|l|}
\hline
$\lambda [\nm]$              & 680  & 600  & 540  & 500 \\ \hline
$(n_\mathrm{air} - 1) \cdot 10^{-4}$ & 2.76 & 2.77 & 2.78 & 279 \\ \hline
\end{tabular}
\caption{Diffraction indices for air at room temperatur and normal pressure for 
chosen wavelength $\lambda$, taken from \cite{staatsexamen}.}
\label{tab:diff_air}
\end{table}

The linear interpolation is done by the following equation:
\begin{eqnarray}
    n_\lambda &=& \frac{(2.79 - 0.01 j)}{10 ^{4}} - \
    \frac{(\lambda - \lambda_\mathrm{lower})} {10^{6} \cdot \
(\lambda_\mathrm{upper} - \lambda_\mathrm{lower})} + 1 \\
    \lambda &=& n_\lambda \, \lambda,
    \label{eqn:lin_interpol}
\end{eqnarray}
where j is the index of the collum of table \ref{tab:diff_air} within which $\lambda$ lies, 
$\lambda_\mathrm{upper}$ and $\lambda_\mathrm{lower}$ are the upper and lower bounds of the 
intervall (more precisely: $\lambda \in (\lambda_\mathrm{upper}, \lambda_\mathrm{lower}]$ ). 

The corrected values of $\lambda$ are now restricted to values of 
$\lambda \in [500\nm, 620\nm]$ and converted to $\cm$ by taking th inverse. The result 
can be observed in figure \ref{fig:spectrum_halogen_red}.
\begin{figure}
\centering
\includegraphics[width=\pltw]{analysis/figures/halogen_red.pdf}
\caption{Measured spectrum of the halogen lamp over wavenumber for reduced range, 
corrected for diffraction}
\label{fig:spectrum_halogen_red}
\end{figure}

\FloatBarrier

\subsection{Iodine absorption spectrum}

\subsubsection{Identifying progressions}
The process of identifying progressions turned out to be quiet difficult – reflecting the 
rather unsteady course of history. 
The absorption spectrum and the found progressions are shown in figure \ref{fig:absorp}.
The measured wavelenghts are reduced to vacuum wavelength by equation \eqref{eqn:lin_interpol} 
and then transformed to wavenumbers in $\cm$. 
As a first step, we preselected minima by an algorithm checking for each point to be lower than 
its four neighbouring points and added further points by hand later on. For identifying corresponding 
progression to each of the selected points, we used the starting point of the 'zeroth progression' 
$v'' = 0  \rightarrow v' = 25$ with a band at $545.8 \nm = 18321 \cm$, as given by \cite{staatsexamen}. 
In our measurement, we found a mimima at $\sigma_{0, 25} = 18319 \pm 10\nm$. From that line, we could first 
identify the progression members down to $v' = 47$, where the last three points could be added using 
the first order approximation of the energy differences, namely their negative linear correlation 
to the energy (see figure \ref{fig:absorp_detail_03}). We then used the same relationship to identify 
points for $v' < 25$, which overlap with point of the progression $v''(1) \rightarrow v'$, the 
'first progression', as shown in figure \ref{fig:absorp_detail_02}. 
We found much less values for this progression, as 
it overlaps with the progression $v''(1) \rightarrow v'$, 'second progression', as well, 
as one can observe in figure \ref{fig:absorp_detail_01}. 
In order to identify the points all progressions correctly, we repeatedly 
referred to \cite{staatsexamen}. 
All identified points with respective wavenumbers 
$\sigma_{v'', v'}, v'' \in \{0, 1, 2\}$ are shown in table \ref{tab:prog}, 
while the energy differences $\Delta G_{v''}(v' + 1 / 2) = \sigma_{v'', v'} - \sigma_{v'', v' + 1}$ are 
listed in table \ref{tab:dG}.

\begin{table}[h]
\centering
\small
\input{"analysis/table1.txt"}
\caption{Identified members of progressions of vibrational modes $v'' \rightarrow v'$ 
and corresponding wavenumbers $\sigma_{v'', v'} = G'(v') - G''(v'')$ with uncertainties
$\Delta \sigma_{0, v'}$. }
\label{tab:prog}
\end{table}

\begin{table}[h]
\centering
\small
\input{"analysis/table2.txt"}
\caption{Differences in energy 
$\Delta G_{v''} = \Delta G_{v''} (v' + \frac{1}{2}) = G_{v''}(v') - G_{v''}(v' + 1)$ 
between successive members of the progressions $v'' \rightarrow v'$, $v'' \in \{0, 1, 2\}$. 
$ \Delta (\Delta G) $ are the corresponding uncertainties. 
 }
\label{tab:dG}
\end{table}

\begin{figure}
    \centering
    \includegraphics[width=\pltw]{analysis/figures/absorp_03.pdf}
    \caption{Measured absorption spectrum of iodine 2 molecules. The plot show the spectrum of 
    the halogen lamp having passed the iodine pipe. The local minima correspond to maxima of 
    absorption of the iodine. Three progressions of vibrational states $v'' \rightarrow v'$
    within the electronic transition 
    $ X ^1\Sigma_{\sigma \, \mathrm{u}}^{+} \quad \rightarrow \quad B ^3\Pi_{\sigma \, \mathrm{u}}^{+}$
    are identified with colored dots. }
    \label{fig:absorp}
\end{figure}

\begin{figure}
    \centering
    \begin{subfigure}[b]{\mpltw}
        \includegraphics[width=\textwidth]{analysis/figures/absorp_03_detail_01.pdf}
        \caption{Overlap of $v'=1$ and $v'=2$}
        \label{fig:absorp_detail_01}
    \end{subfigure}\qquad
    \begin{subfigure}[b]{\mpltw}
        \includegraphics[width=\textwidth]{analysis/figures/absorp_03_detail_02.pdf}
        \caption{Overlap of $v'=0$ and $v'=1$"}
        \label{fig:absorp_detail_02}
    \end{subfigure}
    \begin{subfigure}[b]{\mpltw}
        \includegraphics[width=\textwidth]{analysis/figures/absorp_03_detail_03.pdf}
        \caption{End of $v'=0'$}
        \label{fig:absorp_detail_03}
    \end{subfigure}
    \caption{Details of \ref{fig:absorp}: members of progressions at points, where 
    identification is done manually, comparing with \cite{staatsexamen}.
    }
    \label{fig:absorp_detail}
\end{figure}



\FloatBarrier

\subsubsection{Birge-Sponer plots}
The energy differences $\Delta G_{v''}(v' + 1/2)$ of the identified progressions can 
be plotted over the corresponding wavenumbers $v' + 1/2$. The resulting so-called 
Birge-Sponer plots, figures 
\ref{fig:b_s_0},  \ref{fig:b_s_1},  \ref{fig:b_s_2}, 
are used to calculate the vibrational constants $\omega_e'$ and $\omega_e' x_e'$. 

% Birge Sponer plots
\begin{figure}
    \centering
    \includegraphics[width=\pltw]{analysis/figures/b_s_0.pdf}
    \caption{Birge-Sponer plot for the progression $v'' = 0 \rightarrow v'$.  
    }
    \label{fig:b_s_0}
\end{figure}

Instead of deducing these variable graphically, we used the expansion 
\eqref{eqn:G(v)_taylor} and did a 
polinomial fit using the linear least square method \cite{cowan1998statistical}. 
The fitting was done with the function 'polyfit' of the python library 'SciPy'~\cite{scipy}, 
while all propagation of errors was calculated using the python package 
'uncertainties'~\cite{uc}.
As we have only a limited number of points, we used a second degree polynomial
for the zeroth progression and a linear fit for the first and second progression. 
The plots show the best fit as well as estimates for the uncertainties, 
calculated by adding and subtracting the standard deviation of the 
polynome evaluated at each point $v'$. 
In order to maintain readability, we first discuss our results for the zeroth 
progression, and then add the values calculated for the first and second progression 
without further explaining each step, if they are done analogously.

\subsubsection{Progression $v'' = 0 \rightarrow v'$}
The polynomial fit for 
\begin{equation}
    P_2(v') = p_2 {v'}^2 + p_1 v' + p_0
\end{equation}
yields the following values and their uncertainties expressed in form of 
the covariance matrix:
\begin{eqnarray}
    p_2 &=& 2.037\mathrm{e}-03 \cm\\
    p_1 &=& -2.188\mathrm{e}+00 \cm\\
    p_0 &=& 1.337\mathrm{e}+02 \cm\\
    \mathrm{cov}(i, j) &=& 
    \begin{pmatrix}
        4.027\mathrm{e}-05 &-2.509\mathrm{e}-03 &3.604\mathrm{e}-02 \\
        -2.509\mathrm{e}-03 &1.588\mathrm{e}-01 &-2.317\mathrm{e}+00 \\
        3.604\mathrm{e}-02 &-2.317\mathrm{e}+00 &3.457\mathrm{e}+01 \\
    \end{pmatrix}
\end{eqnarray}
The entries of the covariance 
matrices are given in $\mathrm{cm^{-2}}$.

Comparing $P_2$ to \eqref{eqn:dG1_2} up to the second order lets us solve for 
the vibrational parameters $\omega_e'$, $\omega_e x_e'$, $\omega_e y_e'$. The system of linear equations in 
${v'}^2, v'$ and $1$ has the solutions:
\begin{eqnarray}
    \omega_e y_e' &=& \frac{p_0}{3} \\
    \omega_e x_e' &=& p_2 - \frac{p_1}{2} \\
    \omega_e'     &=& \frac{11}{12}p_2 - p_1 + p_0.
\end{eqnarray}
Inserting the values from above and doing the error propagation yields 
\begin{eqnarray}
    \omega_e y_e' &=& 6.789\mathrm{e}-04 \cm \nonumber \\
    \omega_e x_e' &=& 1.096\mathrm{e}+00 \cm \nonumber \\
    \omega_e' &=& 1.359\mathrm{e}+02 \cm \nonumber \\
    \mathrm{cov}(i, j) &=& 
    \begin{pmatrix}
        4.474\mathrm{e}-06 &4.317\mathrm{e}-04 &1.286\mathrm{e}-02 \\
        4.317\mathrm{e}-04 &4.225\mathrm{e}-02 &1.278\mathrm{e}+00 \\
        1.286\mathrm{e}-02 &1.278\mathrm{e}+00 &3.943\mathrm{e}+01 \\
    \end{pmatrix}
\\ \Rightarrow \qquad
    \omega_e y_e' &=& 0.0007 \pm 0.0021 \cm\\
    \omega_e x_e' &=& 1.10 \pm 0.21 \cm\\
    \omega_e' &=& 135.9 \pm 6.3 \cm
\end{eqnarray} 
We observe that discrepancy between these values and the stated literature values 
(\ref{tab:lit_val}). $\omega_e y_e'$ is practically zero within it's uncertainty. 
Both $\omega_e'$ and $\omega_e x_e'$ are larger than the literature values by 
1.6 standard deviations. 


Using these results, we can calculate the dissociation energies $D_e'$ and $D_0'$. For the 
prior one we get with equation \eqref{eqn:D_e}:
\begin{equation}
    D_e' = \frac{ \omega_e'^2}{ 4  \omega_e x_e'} =\left(4.2 \pm 0.4\right) \times 10^{3} \cm.
\end{equation}
The latter ist calculated using \ref{eqn:D_0}. Before applying this formula, we need to find 
the intersect with the $\Delta \sigma == 0$-axis, which corresponds just to 
$\nu_\mathrm{diss}$. Since we don't observe all transitions, we solve $P_2(v_\mathrm{diss}') = 0$, and 
take the sum over all values $0 < v' + 1/2 < v_\mathrm{diss}'$. In order to estimate the 
uncertainties, we also calculated $v_\mathrm{diss, upper}$ and  $v_\mathrm{diss, lower}$ with 
the corresponding upper und lower limit shown in \ref{fig:b_s_0} and evaluated the sum over 
these polynomes. The dissociation quantum numbers and resulting energies are 
\begin{eqnarray}
    v_\mathrm{diss}' &=& 64.5 \\
    D_0' = \sum_{v' = 0 }^{v'_\mathrm{diss}} P_2 \left (v' + \frac{1}{2} \right ) &=& 4257 \cm \\
    v_\mathrm{diss,\, upper}' &=& 69.5 \\
    D_{0,\, \mathrm{upper}}' &=& 4419 \cm\\
    v_\mathrm{diss,\, lower}' &=& 61.5 \\
    D_{0,\, \mathrm{lower}}' &=& 4122 \cm
\end{eqnarray}
We can summariye these results using the maximum of the distances as the uncertainty of $D_0'$
\begin{equation}
    D_0' = \left(4.26 \pm 0.16\right) \times 10^{3} \cm
\end{equation}
Both values coincide and math also the given literature value within ther uncertainty. 

In order to calculate the excitation energy $\sigma_{00}$, we take an arbitrary point of our 
progression and extrapolate the difference between this transition 
$v''= 0 \rightarrow v'$ and the $v'' = 0 \rightarrow v' = 0$ transition. If we take the initial 
point for identifying our progression, $\sigma_{0, 25} = 18319 \pm 10\nm$, in order 
to minimize the error, we get.
\begin{eqnarray}
    \sigma_{00} &:=& G(v' = 0) = G(v' = 25) - (P_2(24.5) - P_2(0.5)) \nonumber \\
                 &\, =& 18364 \pm 12 \cm
\end{eqnarray}
A comparison between the uncertainties of $\sigma_{0, 25}$ and $\sigma_{00}$ shows, 
that the uncertainty of the latter one stems mostly from the given uncertainty 
of $\sigma_{0, 25}$, while the one of the fit is about a magnitude smaller. This, 
as well as simply looking at the plot and recognizing the small deviations from 
the fit indicates, that initial errors might have been assumed to high. 

With the parameters claculated until now, we can deduce the energy $E_\mathrm{diss}$ 
for which the iodine molecule dissociates in the experiment, 
namely the dissociation energy of the $B$-state given in units of the 
energy defined by the ground state. This can be calculated by
\begin{equation}
    E_\mathrm{diss} := \sigma_{00} + D_0'
    \label{eq:E_diss}
\end{equation}
The numerical result for the zeroth progression is
\begin{equation}
    E_\mathrm{diss} = \left(22.62 \pm 0.16\right) \times 10^{3} \cm
\end{equation}


\subsubsection{Progressions $v'' = (1, 2) \rightarrow v'$}
\begin{figure}
    \centering
    \includegraphics[width=\pltw]{analysis/figures/b_s_1.pdf}
    \caption{Birge-Sponer plot for the progression $v'' = 1 \rightarrow v'$.  
    }
    \label{fig:b_s_1}
\end{figure}

\begin{figure}
    \centering
    \includegraphics[width=\pltw]{analysis/figures/b_s_2.pdf}
    \caption{Birge-Sponer plot for the progression $v'' = 2 \rightarrow v'$.  
    }
    \label{fig:b_s_2}
\end{figure}

The first an second progression are fitted on a linear function of $v'$, since 
the lack of a higher numerb of points or higher accuray doesn't allow higher order fits.
The function to be fitted is given by
\begin{equation}
    P_1(v') = p_1 v' + p_0.
\end{equation}
In oder to get the vibrational parameters $w_e$ and $w_e x_e$, we compare $P_1$ to 
\eqref{eqn:G(v)_taylor} up to the first order and get the foolowing results: 
\begin{eqnarray}
    w_e x_e' &=& -\frac{p_1}{2} \\
    w_e &=&' p_0 - p_1.
\end{eqnarray}

With these equations, we con now compute the same parameters as done before for the 
zeroth progression:
\begin{itemize}
    \item For $v'' = 1 \rightarrow v'$:
        \begin{eqnarray}
            p_2 &=& -2.225\mathrm{e}+00 \cm \nonumber \\
            p_1 &=& 1.355\mathrm{e}+02 \cm \nonumber \\
            \mathrm{cov}(i, j) &=& 
            \begin{pmatrix}
                6.430\mathrm{e}-02 &-1.320\mathrm{e}+00 \\
                -1.320\mathrm{e}+00 &2.785\mathrm{e}+01 \\
            \end{pmatrix}
            \\ \Rightarrow \qquad
            \omega_e x_e' &=& 1.113\mathrm{e}+00 \cm \nonumber \\
            \omega_e' &=& 1.378\mathrm{e}+02 \cm \nonumber \\
            \mathrm{cov}(i, j) &=& 
            \begin{pmatrix}
                1.607\mathrm{e}-02 &6.920\mathrm{e}-01 \\
                6.920\mathrm{e}-01 &3.055\mathrm{e}+01 \\
            \end{pmatrix}
            \\ \Rightarrow \qquad
            \omega_e x_e' &=& 1.11 \pm 0.13 \cm\\
            \omega_e' &=& 137.8 \pm 5.5 \cm \\
            D_e' &=& \left(4.2 \pm 0.4\right) \times 10^{3} \cm \\
            v_\mathrm{diss}' &=& 64.5 \\
            D_0' &=& 4257 \cm \\
            v_\mathrm{diss,\, upper}' &=& 69.5 \\
            D_{0,\, \mathrm{upper}}' &=& 4419 \cm\\
            v_\mathrm{diss,\, lower}' &=& 61.5 \\
            D_{0,\, \mathrm{lower}}' &=& 4122 \cm \\
            \Rightarrow \qquad
            D_0' &=& \left(4.26 \pm 0.16\right) \times 10^{3} \cm
        \end{eqnarray}
        If we pick $v' = 19$ for it being close to $v' = 0$ and the $\Delta G(v' + 1/2)$ 
        for $v' < 19$ close to the linear fit $P_1(v' + 1/2)$, we get the following results for 
        $\sigma_{00}$ and $E_\mathrm{diss}$:
        \begin{eqnarray}
            G'(v' = 19) &=& 17589.9 \pm 9.3 \cm \\
            \sigma_{00} &=& 17605.5 \pm 9.5 \cm \\
            E_\mathrm{diss} &=& \left(21.73 \pm 0.30\right) \times 10^{3} \cm
        \end{eqnarray}
    \end{itemize}
We again see a strong displacement of $\omega_e'$ and $\omega_e x_e'$. Both are too high 
by 2.2 and 2.7 times the standard deviation compared to the literature values, while 
$D_e'$ and $D_0'$ cover again the given value for the dissociation energy.

\begin{itemize}
    \item For $v'' = 2 \rightarrow v'$:
        \begin{eqnarray}
            p_2 &=& -1.613\mathrm{e}+00 \cm \nonumber \\
            p_1 &=& 1.244\mathrm{e}+02 \cm \nonumber \\
            \mathrm{cov}(i, j) &=& 
            \begin{pmatrix}
                4.888\mathrm{e}-02 &-6.853\mathrm{e}-01 \\
                -6.853\mathrm{e}-01 &1.009\mathrm{e}+01 \\
            \end{pmatrix}
            \\ \Rightarrow \qquad
            p_2 &=& -1.61 \pm 0.22 \cm\\
            p_1 &=& 124.4 \pm 3.2 \cm \\
            \omega_e x_e' &=& 8.067\mathrm{e}-01 \cm \nonumber \\
            \omega_e' &=& 1.261\mathrm{e}+02 \cm \nonumber \\
            \mathrm{cov}(i, j) &=& 
            \begin{pmatrix}
                1.222\mathrm{e}-02 &3.671\mathrm{e}-01 \\
            3.671\mathrm{e}-01 &1.151\mathrm{e}+01 \\
        \end{pmatrix}
        \\ \Rightarrow \qquad
        \omega_e x_e' &=& 0.81 \pm 0.11 \cm\\
        \omega_e' &=& 126.1 \pm 3.4 \cm \\
        D_e' &=&\left(4.9 \pm 0.4\right) \times 10^{3} \cm \\
        v_\mathrm{diss}' &=& 76.5 \\
        D_0' &=& 4799 \cm \\
        v_\mathrm{diss,\, upper}' &=& 86.5 \\
        D_{0,\, \mathrm{upper}}' &=& 5339 \cm\\
        v_\mathrm{diss,\, lower}' &=& 69.5 \\
        D_{0,\, \mathrm{lower}}' &=& 4383 \cm \\
        D_0' &=& \left(4.8 \pm 0.5\right) \times 10^{3} \cm 
    \end{eqnarray}
    Here we choose $v' = 12$ for the same reasons as before: 
    \begin{eqnarray}
        G'(v' = 12) &=& 16675.1 \pm 8.3 \cm \\
        \sigma_{00} &=& 16686.4 \pm 8.5 \cm \\
        E_\mathrm{diss} &=& \left(21.5 \pm 0.5\right) \times 10^{3} \cm
    \end{eqnarray}
\end{itemize}
In the latter case, the values for $\omega_e'$ and $\omega_e x_e$ 
actually cover the literatur values. However, the values for 
$D_e'$ and $D_0'$ turn out to be much higher than the values 
calculated before. Especially the value $D_0'$, depending on the 
extrapolation of the polynome, has to be considered with caution 
as the little number of points doesn't allow to resolve the higher 
orders of $\Delta G$. 


\subsubsection{Parameters for the ground state}
With the values for $\Delta G(v' + 1/2)$ calculated in \ref{tab:dG}, we can calculate the vibrational coefficients for 
the ground state by applying equation \eqref{eqn:dG1_2} for $v'' \in \{0, 1, 2\}$:
\begin{eqnarray}
    \Delta G''\left ( \frac{1}{2} \right ) &=& G''(1) - G''(0) \nonumber\\
        &=& (G'(n) - G''(0)) - (G'(n) - G''(1)) \nonumber \\
        &=& \omega_e'' - 2 \omega_e x_e'' \\
    \Delta G''\left ( \frac{3}{2} \right ) &=& G''(2) - G''(1) \nonumber\\
        &=& (G'(n) - G''(1)) - (G'(n) - G''(2)) \nonumber \\
        &=& \omega_e'' - 4 \omega_e x_e'' \\
  \Rightarrow \qquad  \omega_e x_e'' &=& \frac{1}{2} \left ( \Delta G''\left ( \frac{1}{2} \right ) - 
        \Delta G''\left ( \frac{3}{2} \right ) \right )
\end{eqnarray}
As seen in table \ref{tab:dG}, we get corresponding pairs for 
$\Delta G''\left ( \frac{1}{2} \right )$ at $v' + 1/2 \in \{18.5, 19.5, \ldots, 26.5\}$ 
and for 
$\Delta G''\left ( \frac{3}{2} \right )$ at $v' + 1/2 \in \{15.5, 16.5, 17.5, 18.5, 19.5\}$.
Taking weighted averages $\overline{\Delta G''\left ( \frac{1}{2} \right )}$ and 
$\overline{\Delta G''\left ( \frac{3}{2} \right )}$, 
where weights are constituited by the reciprocal variances, 
and subtracting yields
\begin{eqnarray}
    \omega_e'' &=& 213.8 \pm 4.6 \cm \\
    \omega_e x_e'' &=& 0.5 \pm 1.4 \cm.
\end{eqnarray}
A comparison with the literautre values (table \ref{tab:lit_val}) shows 
good agreement for both values. For $\omega_e x_e''$ however, the stated 
uncertainty makes it virtually impossible to use it for further calculations.
The discussion about correctly indicating uncertainties from the beginning 
applies here as well. 


For small vibrations, we make the approximation of a harmonic oscillator and 
can now calculate classical frequency $f$ and the spring constant $k$ with 
\eqref{eqn:freq}:
\begin{eqnarray}
    f_e'' &=& \left(1.282 \pm 0.028\right) \times 10^{12}\ \mathrm{Hz} \\
    k_e'' &=& 6.84 \pm 0.30 \mathrm{\frac{kg}{s^2}}
\end{eqnarray}

Approximating the ground state for low vibrational modes with the Morse function, 
we can calculate the dissociation energy of the ground state with the values for 
$\omega_e''$ and $\omega_e x_e''$ with \eqref{eqn:D_e}:
\begin{equation}
    D_e'' =\left(25 \pm 74\right) \times 10^{3} \cm
\end{equation}
where, as stated before, the enormous uncertainty of $3$ times the nominal value 
doesn't aloow to much interpretation. We can, however, compare this value with a 
second one, which we get from approximating the 
dissociation energy as the difference between $E_\mathrm{diss}$ and the difference in 
energy of the two resulting dissociated atoms. If dissociation takes place directly 
from the ground level, the resulting atoms are both in their ground states $^2 P_{3/2}$, 
while for dissociation from the excited state, one of the atoms is found in the first 
excited state $2^ P_{1/2}$. The energy difference between the two states is given by 
$\Delta E = 7603.15 \cm$~\cite{nist}. We thus get the following results for the energy 
levels calculated with each of the three progressions:
\begin{eqnarray}
    D_{e, i}'' &:=& E_\mathrm{diss,\, i} - \Delta E\\
    D_{e, 0}'' &\ =&\left(15.02 \pm 0.16\right) \times 10^{3} \cm \\
    D_{e, 1}'' &\ =&\left(14.13 \pm 0.30\right) \times 10^{3} \cm \\
    D_{e, 2}'' &\ =&\left(13.9 \pm 0.5\right) \times 10^{3} \cm
\end{eqnarray}
In the literature, this value is given as $12452.5 \pm 1.5 \cm$ (table \ref{tab:lit_val}), 
so we see much better agreement compare to the value $D_e''$ computed before. Even in this 
case, there is no agreement within the standard deviation. 

\subsubsection{Morse potential}
We can use the derived parameters to calculate and plot the Morse potential for the 
excited state. For that, we calculate $a'$ with equation \eqref{eqn:wx_e} and 
use the $r_e' = 3.035 \ \AA$ calculated before \eqref{eqn:r_e1}:
\begin{eqnarray}
    a_i &=& \sqrt{\frac{4 \pi c \mu {\omega_e x_e}_i'}{\hbar}} \\
    a_0' &=& 2.02 \pm 0.19 \ \AA^{-1} \\
    a_1' &=& 2.04 \pm 0.12 \ \AA^{-1} \\
    a_2' &=& 1.74 \pm 0.12 \ \AA^{-1} \\
\end{eqnarray}
For the value $a_0'$ obtained from the zeroth progression, we plot the Morse potential 
in figure \ref{fig:morse}.

\begin{figure}
    \centering
    \includegraphics[width=\pltw]{analysis/figures/morse_plot_0.pdf}
    \caption{Morse plot for the first excited state 
    $B ^3\Pi_{\sigma \, \mathrm{u}}^{+}$ of the iodine 2 molecule. 
    The parameters $a' = 2.02 \pm 0.19\ \AA^{-1}$ and dissociation energy 
    $D_e' = \left(4.2 \pm 0.4\right) \times 10^{3} \cm$
    are obtained from the progression $v'' = 0 \rightarrow v'$. 
    The radius $r_e' = 3.035 \ \AA$ is calculated from literature values in \eqref{eqn:r_e1}. 
    }
    \label{fig:morse}
\end{figure}

\FloatBarrier

\subsection{Evaluating the uncertainties}
Summarizing the observation made above, we observe strong deviations 
of our measurements and the given literature values This is especially 
the case for the parameters of the excited state, where the values 
$\omega_e'$ and $\omega_e x_e'$ tend to be too high. Looking 
at the quality of the fit, on has to ask whether there is indeed 
a bias measuring the spectrum, or some error introduced elsewhere. 
The reasoning behind is, that the given standard deviations 
seem to be estimated much to high - one would expect more fluctuation 
around the fitted polynome. This observation can be made more 
precise using the $\chi^2$-test. For a variable $y_i$ in a sample of 
size $N$ assumed to follow 
a gaussian disribution around its expectation value, which in case of 
an underlying functional relation to another variable $x_i$ is assumed 
to be $\lambda(x_i; \theta)$, where $\theta$ are the parameters of 
the fit, we can define the quantity
\begin{equation}
    \chi^2(\theta) = \sum_{i, j= 1}^N (y_i - \lambda(x_i; \theta))
        (V^{-1}) (y_j - \lambda(x_j; \theta)), 
\end{equation}
where $V^{-1}$ is the inverse of the covariance matrix $\mathrm{cov}(i,j)$. 
The least square fit is done by simply minimizing the quantity. If the 
variables do follow the suspected functional relation and are distributed 
with a variance $\sigma_i^2$, then $chi^2$ should follow the 
$chi^2$-distribution with expectation value $n_d$, where $n_d$ is 
the number of points minus the number of paramter $\theta$.
One can thus test the applied hypothesis by calculating $chi^2$ and 
dividing it by $n_d$, getting a numerical value for the 
\emph{godness-of-fit}. For all three progressions, 
we indeed get values for $chi^2 / n_d$ considerable smaller than one, 
an indicator of either a bad hypothesis or an overestimation 
of the uncertainties (for details refer to \cite{cowan1998statistical}).
The result of the test are shown below
\begin{itemize}
    \item For $v'' = 0 \rightarrow v'$:
        \begin{eqnarray}
            \chi^2 &=& 0.631 \\
            n_d &=& 27 \\
            \text{goodness-of-fit: } \chi^2 / n_d &=& 0.023
        \end{eqnarray}
    \item For $v'' = 1 \rightarrow v'$:
        \begin{eqnarray}
            \chi^2 &=& 0.410 \\
            n_d &=& 10 \\
            \text{goodness-of-fit: } \chi^2 / n_d &=& 0.041
        \end{eqnarray}
    \item For $v'' = 2 \rightarrow v'$:
        \begin{eqnarray}
            \chi^2 &=& 0.25 \\
            n_d &=& 9 \\
            \text{goodness-of-fit: } \chi^2 / n_d &=& 0.027
        \end{eqnarray}
\end{itemize}
One would thus have to reassess the stated uncertainty of the spectrometer - 
something already indicated by the strong alignment of the values relative 
to their seen in the Birge-Sponer plots.

Yet another point to be made is concerning the way we recorded the spectrum. 
The measured values we used for our evaluation where intergrated over a rather 
long time span of 50ms and taking the mean over 100 measurements. 
This leads to relatively stable and smooth data, but
also takes away the possibiliy to see much smaller peaks, as fluctuations tend 
to even out these peak. A good approch to test this hypothesis would be to 
compare data sets taken over a long time with those averaged of a much shorter 
period. 

