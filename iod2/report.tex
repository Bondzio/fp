%%%%%%%%%%%%%%%%%%%%%%%%%%%%%%%%%%%%%%%%%
% Short Sectioned Assignment
% LaTeX Template
% Version 1.0 (5/5/12)
%
% This template has been downloaded from:
% http://www.LaTeXTemplates.com
%
% Original author:
% Frits Wenneker (http://www.howtotex.com)
%
% License:
% CC BY-NC-SA 3.0 (http://creativecommons.org/licenses/by-nc-sa/3.0/)
%
%%%%%%%%%%%%%%%%%%%%%%%%%%%%%%%%%%%%%%%%

%----------------------------------------------------------------------------------------
%	PACKAGES AND OTHER DOCUMENT CONFIGURATIONS
%----------------------------------------------------------------------------------------

\documentclass[paper=a4, fontsize=12pt, xcolor=dvipsnames]{scrartcl} % A4 paper and 11pt font size

\usepackage[T1]{fontenc} % Use 8-bit encoding that has 256 glyphs
\usepackage{fourier} % Use the Adobe Utopia font for the document - comment this line to return to the LaTeX default
\usepackage{amsmath,amsfonts,amsthm} % Math packages
\usepackage{natbib}
\usepackage{pgfplots}
\usepackage{wrapfig}
\usepackage{sidecap}

%\usepackage{minted}

% Bibliographie auf deutsch
%\usepackage{harvard}
%\renewcommand{\harvardand}{und} 

\usepackage{xcolor}
\usepackage[utf8]{inputenc} 
%\usepackage[ngerman]{babel}

\usepackage{latexsym}
\usepackage{textcomp}
\usepackage[T1]{fontenc}
\usepackage{bm}% bold math
\usepackage{hyperref}
\usepackage{graphicx}
\usepackage{caption}
\usepackage{subcaption}
\usepackage{verbatim}
\usepackage{epsfig}
\usepackage{framed,color}
\usepackage[usenames,dvipsnames]{pstricks}
\usepackage{epsfig}
\usepackage{tikz}
\usepackage{lipsum} % Used for inserting dummy 'Lorem ipsum' text into the template
\usepackage{sectsty} % Allows customizing section commands
\allsectionsfont{\centering \normalfont\scshape} % Make all sections centered, the default font and small caps

\usepackage{fancyhdr} % Custom headers and footers
\pagestyle{plain} % Makes all pages in the document conform to the custom headers and footers
\fancyhead{} % No page header - if you want one, create it in the same way as the footers below
\fancyfoot[L]{} % Empty left footer
\fancyfoot[C]{} % Empty center footer
\fancyfoot[R]{\thepage} % Page numbering for right footer
%\renewcommand{\headrulewidth}{0pt} % Remove header underlines
%\renewcommand{\footrulewidth}{0pt} % Remove footer underlines
\setlength{\headheight}{13.6pt} % Customize the height of the header
\usepackage{eso-pic}
\numberwithin{equation}{section} % Number equations within sections (i.e. 1.1, 1.2, 2.1, 2.2 instead of 1, 2, 3, 4)
\numberwithin{figure}{section} % Number figures within sections (i.e. 1.1, 1.2, 2.1, 2.2 instead of 1, 2, 3, 4)
\numberwithin{table}{section} % Number tables within sections (i.e. 1.1, 1.2, 2.1, 2.2 instead of 1, 2, 3, 4)

\setlength\parindent{0pt} % Removes all indentation from paragraphs - comment this line for an assignment with lots of text
\setcapindent{1cm} 

%----------------------------------------------------------------------------------------
%	TITLE SECTION
%----------------------------------------------------------------------------------------

\newcommand{\horrule}[1]{\rule{\linewidth}{#1}} % Create horizontal rule command with 1 argument of height

\title{ 
\normalfont \normalsize 
\textsc{Albert-Ludwigs-University Freiburg} \\ [25pt] % Your university, school and/or department name(s)
\horrule{0.5pt} \\[0.4cm] % Thin top horizontal rule
\huge $J_2$-Molecule \\ % The assignment title
\horrule{2pt} \\[0.5cm] % Thick bottom horizontal rule
}

\author{Friedrich Schüßler and Volker Karle} % Your name

\date{\normalsize\today} % Today's date or a custom date

\begin{document}
\maketitle

\newpage
\tableofcontents
\thispagestyle{empty}
\newpage
\setcounter{page}{1}


%----------------------------------------------------------------------------------------
%	PROBLEM 1
%----------------------------------------------------------------------------------------


\section{Introduction}
\subsection{Overview}
In the following experiment, we will analyse the spectral band of absorbtion of the iodine 2 molecule 
with spectroscopal methods. For the given conditions, the expected observation stems almost 
entirely from one electronic transition, the 
\begin{equation}
    B ^3\Pi_{\sigma \, \mathrm{u}}^{+} \quad <- \quad X ^1\Sigma_{\sigma \, \mathrm{u}}^{+}
\end{equation}
transition. Since iodine is a relativly heavy di-atomic molecule, it has vibrational modes that 
split up the total energy of the molecule considerably and are visible in the spectrum with the 
resolution of the given spectrometer. The energy is further changed by rotational modes of the 
atoms, which however don't show up as separate lines but rather widen the observed bands, as the 
resolution is not high enough. 
The observed data is then used to calculated characteristical constants for the molecule, such as 
the dissipation energy $D_e$ and the mean radius $r_c$. 

\subsection{Historic review}
The structure of molecular iodine ($I_2$) has been studied various times in physical chemistry 
during the last century. One of the first thorough measurements of the spectral bands was done in 
1923 by the German physicist Reinhard Meck \cite{mecke1923bandenspektrum}. The vibrational states 
were correctly numbered for the first time by Loomis \cite{loomis1927correlation}, who analysed 
the fluorecence spectrum earlier measured by Wood \cite{wood1911} for $v = 26$.



\section{Measurements}
\clearpage
\section{Interpretation}
\clearpage
\section{Conclusion}
\cleardoublepage

\phantomsection

\addcontentsline{toc}{section}{Bibliography und List of figures}
\bibliographystyle{plain}
\bibliography{report}
\listoffigures

\end{document}
