\section{Measurements for Faraday effect}
\subsection{Currents respective Angles}
The very first measurement with $I=0.0A$ yields as expected a rotation of the polarization
with $\alpha = 0.6 \pm 0.2$. We will choose the reference state of half-shade Polarimeter 
to be clearly dark, since the precision is best in this case.



    \begin{table}[htdp]
        \begin{tabular}{|l||p{1.1cm}|p{1.1cm}|p{1.1cm}|p{1.1cm}|p{1.1cm}|p{1.1cm}|p{1.1cm}|p{1.1cm}|p{1.1cm}|p{1.1cm}|}
        \hline
            \multicolumn{11}{|c|}{\cellcolor[RGB]{206,250,201}$
            \mathbf{Measurement \quad 2.1}$} \\
\textbf{angle $\alpha$}& 0.60& 0.90& 1.60& 2.00& 2.50& 3.10& 3.40& 3.60& 4.40& 5.10 \\
\textbf{Current $I$}& 0.00& 0.20& 0.40& 0.60& 0.80& 1.00& 1.20& 1.40& 1.60& 1.80 \\

        \hline
        \end{tabular}
        \begin{tabular}{|l||p{1.1cm}|p{1.1cm}|p{1.1cm}|p{1.1cm}|p{1.1cm}|p{1.1cm}|p{1.1cm}|p{1.1cm}|p{1.1cm}|p{1.1cm}|}
        \hline\textbf{angle $\alpha$}& 5.70& 6.30& 6.70& 7.10& 7.90& 8.00& 8.40& 9.20& 9.50& 10.10 \\
\textbf{Current $I$}& 2.00& 2.20& 2.40& 2.60& 2.80& 3.00& 3.20& 3.40& 3.60& 3.80 \\

        \hline
        \end{tabular}
    \begin{tabular}{|l||p{1.1cm}|p{1.1cm}|p{1.1cm}|p{1.1cm}|p{1.1cm}|p{1.1cm}|}
    \hline\textbf{angle $\alpha$}& 10.70& 11.20& 11.70& 12.40& 12.80& 13.10 \\
\textbf{Current $I$}& 4.00& 4.20& 4.40& 4.60& 4.80& 4.94 \\

    \hline
    \end{tabular}
    \caption{Measurement done by Friedrich Schüßler}
    \label{Power05}
    \end{table}


    \begin{table}[htdp]
        \begin{tabular}{|l||p{1.1cm}|p{1.1cm}|p{1.1cm}|p{1.1cm}|p{1.1cm}|p{1.1cm}|p{1.1cm}|p{1.1cm}|p{1.1cm}|p{1.1cm}|}
        \hline
            \multicolumn{11}{|c|}{\cellcolor[RGB]{206,250,201}$
            \mathbf{Measurement \quad 2.2}$} \\
\textbf{angle $\alpha$}& 0.80& -0.40& -0.50& -1.10& -1.70& -1.90& -2.70& -3.20& -4.00& -4.20 \\
\textbf{Current $I$}& 0.00& -0.20& -0.40& -0.60& -0.80& -1.00& -1.20& -1.40& -1.60& -1.80 \\

        \hline
        \end{tabular}
        \begin{tabular}{|l||p{1.1cm}|p{1.1cm}|p{1.1cm}|p{1.1cm}|p{1.1cm}|p{1.1cm}|p{1.1cm}|p{1.1cm}|p{1.1cm}|p{1.1cm}|}
        \hline\textbf{angle $\alpha$}& -4.90& -5.50& -5.70& -6.20& -6.80& -7.10& -7.60& -8.70& -9.50& -10.00 \\
\textbf{Current $I$}& -2.00& -2.20& -2.40& -2.60& -2.80& -3.00& -3.20& -3.60& -3.80& -4.00 \\

        \hline
        \end{tabular}
    \begin{tabular}{|l||p{1.1cm}|p{1.1cm}|p{1.1cm}|p{1.1cm}|p{1.1cm}|}
    \hline\textbf{angle $\alpha$}& -10.40& -10.80& -11.40& -11.80& -12.10 \\
\textbf{Current $I$}& -4.20& -4.40& -4.60& -4.80& -4.90 \\

    \hline
    \end{tabular}
    \caption{Measurement done by Friedrich Schüßler}
    \label{Power05}
    \end{table}

    \begin{table}[htdp]
        \begin{tabular}{|l||p{1.1cm}|p{1.1cm}|p{1.1cm}|p{1.1cm}|p{1.1cm}|p{1.1cm}|p{1.1cm}|p{1.1cm}|p{1.1cm}|p{1.1cm}|}
        \hline
            \multicolumn{11}{|c|}{\cellcolor[RGB]{206,250,201}$
            \mathbf{Measurement \quad 2.3}$} \\
\textbf{angle $\alpha$}& 1.50& 2.20& 2.80& 3.10& 3.20& 3.60& 3.80& 3.90& 4.10& 4.20 \\
\textbf{Current $I$}& 0.40& 0.79& 0.94& 1.03& 1.11& 1.27& 1.33& 1.37& 1.42& 1.48 \\

        \hline
        \end{tabular}
        \begin{tabular}{|l||p{1.1cm}|p{1.1cm}|p{1.1cm}|p{1.1cm}|p{1.1cm}|p{1.1cm}|p{1.1cm}|p{1.1cm}|p{1.1cm}|p{1.1cm}|}
        \hline\textbf{angle $\alpha$}& 4.60& 5.70& 5.80& 6.10& 6.70& 7.10& 7.90& 8.20& 8.30& 8.50 \\
\textbf{Current $I$}& 1.64& 1.99& 2.07& 2.21& 2.44& 2.59& 2.90& 3.00& 3.06& 3.14 \\

        \hline
        \end{tabular}
    \begin{tabular}{|l||p{1.1cm}|p{1.1cm}|p{1.1cm}|p{1.1cm}|p{1.1cm}|p{1.1cm}|}
    \hline\textbf{angle $\alpha$}& 9.90& 10.30& 11.00& 11.60& 11.80& 12.40 \\
\textbf{Current $I$}& 3.69& 3.83& 4.07& 4.35& 4.40& 4.59 \\

    \hline
    \end{tabular}
    \caption{Measurement done by Volker Karle. As You notice, in this case we did not use linear
        increasing reference points for the \textbf{Current $I$} but a randomized array
        in order to decrease a possible systematic (human) error. In the evaluation you 
        can inspect the results and see that this method was quite sucessfull.}
    \label{Power05}
    \end{table}

    \begin{table}[htdp]
        \begin{tabular}{|l||p{1.1cm}|p{1.1cm}|p{1.1cm}|p{1.1cm}|p{1.1cm}|p{1.1cm}|p{1.1cm}|p{1.1cm}|p{1.1cm}|p{1.1cm}|}
        \hline
            \multicolumn{11}{|c|}{\cellcolor[RGB]{206,250,201}$
            \mathbf{Measurement \quad 2.4}$} \\
\textbf{angle $\alpha$}& 0.50& 0.80& 1.40& 1.80& 2.30& 2.90& 3.60& 4.10& 4.30& 4.90 \\
\textbf{Current $I$}& 0.00& 0.20& 0.40& 0.60& 0.80& 1.00& 1.20& 1.40& 1.60& 1.80 \\

        \hline
        \end{tabular}
        \begin{tabular}{|l||p{1.1cm}|p{1.1cm}|p{1.1cm}|p{1.1cm}|p{1.1cm}|p{1.1cm}|p{1.1cm}|p{1.1cm}|p{1.1cm}|p{1.1cm}|}
        \hline\textbf{angle $\alpha$}& 5.50& 6.20& 6.50& 7.10& 7.60& 8.00& 8.70& 8.90& 9.50& 10.20 \\
\textbf{Current $I$}& 2.00& 2.20& 2.40& 2.60& 2.80& 3.00& 3.20& 3.40& 3.60& 3.80 \\

        \hline
        \end{tabular}
    \begin{tabular}{|l||p{1.1cm}|p{1.1cm}|p{1.1cm}|p{1.1cm}|p{1.1cm}|}
    \hline\textbf{angle $\alpha$}& 10.80& 11.30& 11.60& 12.10& 12.20 \\
\textbf{Current $I$}& 4.00& 4.20& 4.40& 4.60& 4.70 \\

    \hline
    \end{tabular}
    \caption{Measurement done by Volker Karle. Here we used a linear increasing reference point
        for the \textbf{Current $I$} again in order to compare the result to the randomized
        array which was used in Measurement 2.3.}
    \label{Power05}
    \end{table}
  
\subsection{Measurement of $2\epsilon$}

\begin{center}
 The Measurement of the angles for the $2\epsilon$ yield:
\begin{align}
    \alpha_0 = 170 \pm 5, \quad
    \alpha_1 = 8   \pm 3 \\
    2\epsilon = \alpha_0 - \alpha_1 = 162.0 \pm 6   
\end{align}
\end{center}

