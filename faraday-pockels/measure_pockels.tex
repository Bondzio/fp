\newcommand{\figdirpockels}{analysis_pockels/figures/}

\section{Measurements for Pockels effect}
\paragraph{Diffraction and Bifringence}
The inspection of the experimental setup revealed certain aspects
which we might need to include in our interpretation of the results
later. First we noticed minor diffraction phenomena already between
the exit of the Laser's Gaussian Beam and the beginning of the
Pockelcell, as well as bifringence phenomena which caused difficulties
in focusing the laser.  
\paragraph{Low frequency oscillations} Within the lasersignal 
we observed frequencies about $20 Hz$ with a intensity of $10mV/$diff.
We did not expect these frequencies at first; we will look into it
in the later progress of the experiment.
We did now certain measurements for analyzing the behaviour of a 
certain degree of the analysator.

\begin{figure}
    \begin{subfigure}[b]{\picwidth}
        \includegraphics[width=\textwidth]{\figdirpockels 12sawtooth1}
        \caption{}
    \end{subfigure}\qquad
    \begin{subfigure}[b]{\picwidth}
        \includegraphics[width=\textwidth]{\figdirpockels 12sawtooth2}
        \caption{}
    \end{subfigure}
    \begin{subfigure}[b]{\picwidth}
        \includegraphics[width=\textwidth]{\figdirpockels 12sawtooth3}
        \caption{}
    \end{subfigure}
    \begin{subfigure}[b]{\picwidth}
        \includegraphics[width=\textwidth]{\figdirpockels 12sawtooth4}
        \caption{}
    \end{subfigure}
    \caption{These were the first measurements with the 
        oscillscope in order to calibrate the Analysator.
        The next measurements will show the result of the 
        final calibration. You can find the same configuration
        with different zoom from (b) to (d).}
    \label{fig:saw1}
\end{figure}
\flushleft
\begin{figure}
    \begin{subfigure}[b]{\picwidth}
        \includegraphics[width=\textwidth]{\figdirpockels 12sawtooth5}
        \caption{}
    \end{subfigure}\qquad
    \begin{subfigure}[b]{\picwidth}
        \includegraphics[width=\textwidth]{\figdirpockels 12sawtooth6}
        \caption{}
    \end{subfigure}
    \begin{subfigure}[b]{\picwidth}
        \includegraphics[width=\textwidth]{\figdirpockels 12sawtooth7}
        \caption{}
    \end{subfigure}
    \begin{subfigure}[b]{\picwidth}
        \includegraphics[width=\textwidth]{\figdirpockels 12sawtooth8}
        \caption{}
    \end{subfigure}
    \begin{subfigure}[b]{\picwidth}
        \includegraphics[width=\textwidth]{\figdirpockels 12sawtooth9}
        \caption{}
    \end{subfigure}
    \begin{subfigure}[b]{\picwidth}
        \includegraphics[width=\textwidth]{\figdirpockels 12sawtooth10}
        \caption{}
    \end{subfigure}

    \caption{These series of figures show the further 
        attempts to calibrate the analysator. As you can
        notice every figure shows a different degree of the 
        angle and hence the distribution of voltage changes. }
    \label{fig:saw2}
\end{figure}
\flushleft
After these we replaced the sawtooth signal by a sinus signal, but without direct current.
The Frequency generated was about $f=5.4$ Khz, but it was not stable
but oscillating within $0.1$ Khz. The manipulation of the trigger-
level increased the stability to a acceptable level. 
Then we applied the direct current and looked at the inluence of different parameters on
the received signal, especially the effect of changing the
voltage. 
\paragraph{Important note:} After the measurements the
signalfrequency of the electrical field has gone up to 
$f=6.129$ khz! We noticed this without us to regulate this 
behavior. After the measurements we have changed this frequency in
order to the into account the change of frequency by adjusting.
We furthermore note that the change of frequency
does not seem to have a huge influence on the noisy regime except
that by reducing the frequency to $f=3.0$kHz we see the increase
of structure, meaning less noise and a clearly recognizable
sinusshape in the former noisy shape while the noisy regime
was shifted to other lower frequencies. We will look more refined
into this behavior in the next chapter.
\begin{figure}
    \begin{subfigure}[b]{\picwidth}
        \includegraphics[width=\textwidth]{\figdirpockels 22sinus01}
        \caption{}
    \end{subfigure}\qquad
    \begin{subfigure}[b]{\picwidth}
        \includegraphics[width=\textwidth]{\figdirpockels 22sinus02}
        \caption{}
    \end{subfigure}
    \begin{subfigure}[b]{\picwidth}
        \includegraphics[width=\textwidth]{\figdirpockels 22sinus03}
        \caption{}
    \end{subfigure}
    \begin{subfigure}[b]{\picwidth}
        \includegraphics[width=\textwidth]{\figdirpockels 22sinus04}
        \caption{}
    \end{subfigure}
    \begin{subfigure}[b]{\picwidth}
        \includegraphics[width=\textwidth]{\figdirpockels 22sinus05}
        \caption{}
    \end{subfigure}
    \begin{subfigure}[b]{\picwidth}
        \includegraphics[width=\textwidth]{\figdirpockels 22sinus06}
        \caption{}
    \end{subfigure}

    \caption{These series of figures show the impact of
        an applied sinussignal and a direct current with
        varying voltage. In general we do not recognize huge 
        qualitative differences amongst those, but we will look
        later with a more refined analysis to it.}
    \label{fig:sinus1}
\end{figure}
\flushleft
\begin{figure}
    \begin{subfigure}[b]{\picwidth}
        \includegraphics[width=\textwidth]{\figdirpockels 22sinus07}
        \caption{}
    \end{subfigure}\qquad
    \begin{subfigure}[b]{\picwidth}
        \includegraphics[width=\textwidth]{\figdirpockels 22sinus08}
        \caption{}
    \end{subfigure}
    \begin{subfigure}[b]{\picwidth}
        \includegraphics[width=\textwidth]{\figdirpockels 22sinus09}
        \caption{}
    \end{subfigure}
    \begin{subfigure}[b]{\picwidth}
        \includegraphics[width=\textwidth]{\figdirpockels 22sinus10}
        \caption{}
    \end{subfigure}
    \begin{subfigure}[b]{\picwidth}
        \includegraphics[width=\textwidth]{\figdirpockels 22sinus11}
        \caption{}
    \end{subfigure}
    \begin{subfigure}[b]{\picwidth}
        \includegraphics[width=\textwidth]{\figdirpockels 22sinus12}
        \caption{}
    \end{subfigure}

    \caption{Now we enter the noisy regime in which we can 
        observe the effect of the state of the analysator. 
        Here we are indulged to find the noisiest curve in 
        order to check the range in which the $U_{\lambda / 2}$
        might be to find. As you will notice from the figures we
        suspect it to be in the range from $144.4V$ to $148.0V$
        but also see the next series of figures.}
    \label{fig:sinus2}
\end{figure}
\subsection{Refined frequency analysis}
The figures in this section are accomplished with $U_{DC}=145.8 V$.
In this analysis we investigate the effect of the frequency
on the regime in which we get a rather noisy signal. 

After
adjusting to a relatively low frequency $f=2.0$ kHz we changed
the frequencyrange on the oscillscope from 
$[1 - 25]$kHz to $[35 - 2500]$ Hz in order to stabilize the signal.
Now since we have found the frequency with which we have a chance
to find the already mentioned frequency-doubling.
\begin{figure}
    \includegraphics[width=15cm]{\figdirpockels 23sinus02}
    \caption{At this level the regime of the voltage is shifted
        again, but in this case to a higher value. Unfortunatelly
        at this high frequency it is not possible to distinguish
        clearly between the different regimes because the signal
        is very noisy at all voltages (due to shackling 
            and wiggling).}
\end{figure}
\clearpage
\subsection{Reanalysis at the lower frequency}
At the frequency $f=1$kHz we repeat now the former analysis and are
able to search after the so called \textit{frequency doubling}.
The signal is much more stable compared to the higher frequencies 
with regards to the noise and we will see in the analysis that
it is even possible to interpolate the regime we are searching after.
Therefore we inspect the curves from $U=135$V up to $U=142$V.
\begin{figure}
    \begin{subfigure}[b]{\picwidth}
        \includegraphics[width=\textwidth]{\figdirpockels 24sinus01}
        \caption{}
    \end{subfigure}
    \begin{subfigure}[b]{\picwidth}
        \includegraphics[width=\textwidth]{\figdirpockels 24sinus02}
        \caption{}
    \end{subfigure}
    \begin{subfigure}[b]{\picwidth}
        \includegraphics[width=\textwidth]{\figdirpockels 24sinus03}
        \caption{}
    \end{subfigure}
    \begin{subfigure}[b]{\picwidth}
        \includegraphics[width=\textwidth]{\figdirpockels 24sinus04}
        \caption{}
    \end{subfigure}
    \caption{Refined search for the noisy regime.}
    \label{fig:sinus7}
\end{figure}

\begin{figure}
    \begin{subfigure}[b]{\picwidth}
        \includegraphics[width=\textwidth]{\figdirpockels 24sinus05}
        \caption{}
    \end{subfigure}\qquad
    \begin{subfigure}[b]{\picwidth}
        \includegraphics[width=\textwidth]{\figdirpockels 24sinus06}
        \caption{}
    \end{subfigure}
    \begin{subfigure}[b]{\picwidth}
        \includegraphics[width=\textwidth]{\figdirpockels 24sinus07}
        \caption{}
    \end{subfigure}
    \begin{subfigure}[b]{\picwidth}
        \includegraphics[width=\textwidth]{\figdirpockels 24sinus08}
        \caption{}
    \end{subfigure}
    \begin{subfigure}[b]{\picwidth}
        \includegraphics[width=\textwidth]{\figdirpockels 24sinus09}
        \caption{}
    \end{subfigure}
    \begin{subfigure}[b]{\picwidth}
        \includegraphics[width=\textwidth]{\figdirpockels 24sinus10}
        \caption{}
    \end{subfigure}
    \caption{Refined search for the noisy regime.}
    \label{fig:sinus8}
\end{figure}
\subsection{Low frequency of laser}\label{sec:laser}
Now we removed everything except from the photodiode and noticed that
the laser exhibits the noise signal which we found in the beginning
(see figure~\ref{fig:laser_a}). After this we also removed the
laser and noticed that the light coming from the ceiling was 
measured by the diode (see figure~\ref{fig:laser_b}).
After we added the laser again, the polarizator and the analysator
(see figure~\ref{fig:laser_f} ).
We can see that the polarization 
state of the laser does not change the noise signal which we 
tracked down in this analysis. 
\begin{figure}
    \begin{subfigure}[b]{\picwidth}
        \includegraphics[width=\textwidth]{\figdirpockels 24sinus20}
        \caption{}
    \end{subfigure}
    \begin{subfigure}[b]{\picwidth}
        \includegraphics[width=\textwidth]{\figdirpockels 25_laser01}
        \caption{}
        \label{fig:laser_a}
    \end{subfigure}
    \begin{subfigure}[b]{\picwidth}
        \includegraphics[width=\textwidth]{\figdirpockels 25_no_laser01}
        \caption{}
        \label{fig:laser_b}
    \end{subfigure}
    \begin{subfigure}[b]{\picwidth}
        \includegraphics[width=\textwidth]{\figdirpockels 25_laser05}
        \caption{}
        \label{fig:laser_f}
    \end{subfigure}

    \caption{As mentioned in the beginning, an underlying
        frequency of about $f=23$Hz is very visible (a) since 
            we are nearly in the range of that frequency.
        From (b) to (d) we removed the Pockelcell and the applied current and observed
        the laser alone.
        (b) and (d) show the very low frequency which
        the laser inhibits as a nuisance where at (c) we changed the state of the analysator
        such that the intensity was minimal. We expect this signal coming from the electronic components
        the laser consists of. Figure (b) shows the frequency only originating from the room light.
        }
    \label{fig:laser}
\end{figure}
\clearpage

