\section{Theory behind the Pockels effect}

\subsection{Linear and circular polarized light}

\subsection{Birefringence and index ellipsoid}
If the refraction index $n$ of a material depends on the linear polarization of light, 
then a beam of light propagating in the material will be split up into 
two beams with perpendicular polarization and different propagation speed 
\beq
   v_i = \frac{c}{n_i}, 
\eeq
In order to explain this phenomenon, it is helpful to introduce quantities 
connected to the structure of the material, i.~e. it being anisotropic, 
and retrieve the way the light propagates by looking at the solutions 
of Maxwell's equations in matter, 
\bea
\nabla \cdot \D &=& \rho_\text{f} 
\label{eq:max1} \\ 
\nabla \cdot \B &=& 0
\label{eq:max2} \\ 
\nabla \times \E &=& -\frac{\partial \B} {\partial t}
\label{eq:max3} \\ 
    \nabla \times \mathbf{H} &=& \mathbf{J}_\text{f} + 
        \frac{\partial \D} {\partial t}, 
\label{eq:max4}
\eea
where 
\bit
\item $\D$ is the electric displacement field, related to the eletric field $\E$ 
    by the constitutive equation 
    \beq
        \D = \Eps \E 
    \label{eq:const1}
    \eeq
\item $\mathbf{H}$ is the magnetizing field, with the constitutive equation 
    \beq
        \mathbf{H} = \mu^{-1} \B
    \label{eq:const2}
    \eeq
\item $\mathbf{J}_\text{f}$ is the free current density, and
\item $\mathbf{\rho}_\text{f}$ is the free charge density.
\eit
Equations \eqref{eq:const1} and \eqref{eq:const2} are valid for materials without 
coupling between magnetic and eletric fields, which is the case in our 
crystal. We are, however, facing an anistropic material, so that 
$\Eps$ is a tensor, while we assume the permeability $\mu$ of the material 
to be just the vacuum permeability $\mu_0$, such that
$\B = \mu \mathbf{H} = \mu_0 \mathbf{H}$. 
If we assume $\mathbf{\rho}_\text{f} = 0$, $\mathbf{J}_\text{f} = 0$ 
(no free charge and currents), and consider plane waves, 
described by
\bea
    \E &=& \E_0 \exp \left[i(\mathbf{k \cdot r}-\omega t)\right] \\
    \mathbf{H} &=& \mathbf{H_0} \exp \left[i(\mathbf{k \cdot r}-\omega t)\right], 
\eea
then inserting into \eqref{eq:max3} and \eqref{eq:max4} yields:
\bea
    \mu_0 \omega \mathbf{H} &=& \K \times \E 
    \label{eq:plan_H} \\
    \omega \D &=& - \K \times \mathbf{H}.
    \label{eq:plan_D} 
\eea
We observe, that $\K, \D$, and $\Ha$ are mutually perpendicular. 
Looking at the energy flux
\beq
    \mathbf{S} = \E \times \Ha, 
\eeq
we further see, that the directions of $\mathbf{S}$ and $\K$ do not 
coincide if $\E \nparallel \Eps \E$. If we define the 
refractive index $\mathbf{n}$ by 
\beq
    \K = \frac{\omega}{c} \N \, ,
\eeq
we can rewrite equations \eqref{eq:plan_H} and \eqref{eq:plan_D} as 
\bea
    \mathbf{H} &=& \frac{1}{\mu_0 c} \N \times \E 
    \label{eq:plan_Hb} \\
    \D &=& - \frac{1}{c} \N \times \mathbf{H}.
    \label{eq:plan_Db} \, .
\eea
Inserting the latter one into the first and using 
the identity $c^2 = \frac{1}{\mu_0 \epsn}$, we get 
\bea
    \D  &=& \frac{1}{\mu_0 c^2} \N \times \left(\E \times \N\right) \nonumber \\
        &=& \epsn \left(n^2 \E - \left(\N \cdot \E\right) \N \right)
\eea
With the constitutive equation \eqref{eq:const1}, we obtain three 
equations linear in the components $E_k$:
\beq
    \left(n^2 \delta_{ik} - n_i n_k - \frac{\epsilon_{ik}}{\epsn}\right) E_k = 0
\eeq
This equation is solve if its determinant vanishes:
\beq
    \mathrm{det}\,\left|n^2 \delta_{ik} - n_i n_k - \frac{\epsilon_{ik}}{\epsn}\right| = 0
\eeq
By applying the principle axis theorem~\cite{strang2003introduction}, 
we can write $\Eps$ as the diagonal matrix with elements $\eps_x, \eps_y, \eps_z$. 
This yields \emph{Fresnel's equation}:
\bea
    \frac{n^2}{\epsn} \left(\eps_x n_x^2 + \eps_y n_y^2 + \eps_z n_z^2 \right) 
    - \left[
        \frac{n_x^2}{\epsn^2} \eps_x \left(\eps_y + \eps_z\right) + 
        \frac{n_y^2}{\epsn^2} \eps_y \left(\eps_z + \eps_x\right) + 
        \frac{n_z^2}{\epsn^2} \eps_z \left(\eps_x + \eps_y\right) 
    \right] + \nonumber \\
    + \quad \frac{\eps_x \eps_y \eps_z}{\epsn^3} = 0
    \label{eq:fresnel}
\eea
Being of second order in $n_i^2$, $i \in \{x, y, z\}$, there are up to two linearly 
independent solutions, corresponding to two possible directions of polarization. 
The case of a uniaxial crystal is especially easy to solve. If we take the 
$z$-axis to be that of rotational symmetry, we can rename the components of 
$\Eps$ with $\eps_x = \eps_y = \eps_\perp$ and $\eps_z = \eps_\parallel$. 
The $z$-axis is also called the \emph{optical axis}.
Fresnel's equation \eqref{eq:fresnel} can then be factored into
\beq
    \left(n^2 - \frac{\eps_\perp}{\epsn}\right) 
    \left[\frac{\eps_\parallel}{\epsn} n^2 + 
        \frac{\eps_\perp}{\epsn} \left(n_x^2 + n_y^2\right) + 
        \frac{\eps_\perp \eps_\parallel}{\epsn^2} 
    \right] = 0 \, .
\eeq
In other words, we have two quadratic equations
\bea
    n^2 &=& \frac{\eps_\perp}{\epsn} 
    \label{eq:sphere} \\
    \frac{n_z^2}{\eps_\perp} + \frac{n_x^2 + n_y^2}{\eps_\parallel} &=& 1
    \label{eq:ellipsoid}
\eea
Equation \eqref{eq:sphere} gives a sphere for the wave-vector surface. To the 
corresponding types of waves, the crystal behaves like an isotropic body, being 
characterized by the refractive index $n = \sqrt{\eps_\perp}$. These waves are 
called \emph{ordinary waves}. The second type, accordingly named 
\emph{extraordinary waves}, is confined by the ellipsoid \eqref{eq:ellipsoid}. 
It's magnitude depends on 
the angle $\theta$ to the optical axis by 
\beq
    \frac{1}{n^2} = \frac{\sin^2 \theta}{\eps_\parallel} - 
        \frac{\cos^2 \theta}{\eps_\perp} \, .
    label{eq:theta}
\eeq
\\


In order to relate these two surfaces to two directions of polarization and 
determine the direction of a ray, we introduced the \emph{ray vector} $\mathbf{s}$, 
which is defined by the direction of the group velocity
\beq
    \frac{\partial \omega}{\partial \K}
\eeq
and the magnitude fullfilling 
\beq
    \mathbf{s} \cdot \N = 1\,.
\eeq
The ray vector defines the \emph{ray surface} by the condition 
$\phi = \mathrm{const.}$ 
for all $\mathbf{s}$ on that surface, where $\phi$ is the phase of the wave. 

We can show that $\mathbf{s} \parallel \mathbf{S}$: Differentiating 
equations \eqref{eq:plan_Hb} and \eqref{eq:plan_Db}, we get 
\bea
    c \delta \D &=& \delta \Ha \times \N + \Ha \times \delta \N \\
    \mu_0 c \delta \Ha &=& \N \times \delta \E + \delta \N \times \E \, ,
\eea
which by multiplying with $\E$ and $\Ha$, respectively, and applying 
basic properties of vector and scalar products, yield
\bea
    c \E \cdot \delta \D &=& \E \cdot \left(\delta \Ha \times \N \right) + 
        \E \cdot \left(\Ha \times \delta \N \right) \nonumber \\
    &=& \delta \Ha \cdot \left(\N \times \E \right) + 
        \delta \N \cdot \left(\E \times \Ha \right) \nonumber \\
    &=& \mu_0 c \Ha \cdot \delta \Ha + \delta \N \cdot \left(\E \times \Ha \right) \\
    \mu_0 c \Ha \cdot \delta \Ha &=& c \D \cdot \delta \E + 
        \delta \N \cdot \left(\E \times \Ha \right) \, ,
\eea
using again \eqref{eq:plan_Hb} and \eqref{eq:plan_Db}. It follows directly, that 
\beq
    \delta \N \cdot \left( \E \times \Ha \right) = \delta \N \cdot \mathbf{S} = 0 \, .
\eeq
To show, that $\se$ and $\mathbf{S}$ have the same direction, we need to show, 
that $\se$ is orthogonal on the surface of the wave vector surface, or, since 
the infinitesimal displacement $\delta \N$ lies 
on that surface, that $\se \cdot \delta \N = 0$. 
The wave vector surface given by \eqref{eq:fresnel} can be described by 
$f(\omega, \K) = 0$, which implies 
\beq
    \frac{\partial \omega}{\partial \K} = 
    - \frac{\partial f / \partial \K}{\partial f / \partial \omega}
\eeq
for the group velocity and thus 
\beq
    \se \parallel \frac{\partial f }{\partial \K} \parallel \frac{\partial f}{\partial \N}
    \label{eq:s_dir}
\eeq
since the gradient along $\K$ is taken for constant $\omega$. $\partial f / \partial \N$, however, 
is normal to the surface $f = 0$, so we can conclude, that
\bea
    \se \cdot \delta \N &=& 0 \\
    \Rightarrow \qquad \se &\parallel& \mathbf{S} \, .
\eea
If follows, that 
\beq
    \se \cdot \Ha = 0 \qquad \se \cdot \E = 0 \, .
    \label{eq:coplanar}
\eeq
For uniaxial crystals, we can specify the ray surface in a form similar to 
\eqref{eq:ellipsoid} by doing analoge calculations with the folloing substitutions:
\beq
    \E \leftrightarrow c \D \, ; 
    \qquad \N \leftrightarrow \mu_0 c \se \, ; 
    \qquad \eps_{ik} \leftrightarrow {\eps^{-1}}_{ik} \, .
\eeq
The result is 
\bea
    s^2 &=& \frac{\epsn}{\eps_\perp}
    \label{eq:s_sphere} \\
    \eps_\perp {s_z}^2 + \eps_\parallel \left( {s_x}^2 + {s_y}^2\right)&=& 1 \, .
    \label{eq:s_ellipsoid}
\eea
Since $\eps_x = \eps_y$, $\N$ and $\se$ must be coplanar with the optical axes. This 
common plane is called \emph{principal axis} for a given $\N$. Let this plane be 
defined by the $xz$-plane. We can get the direction 
of $\se$ from the relation \eqref{eq:s_dir}, namely by taking the derivatives of 
\eqref{eq:ellipsoid} with respect to $n_x$ and $n_z$:
\beq
    \frac{s_x}{s_z} = \frac{\eps_\perp n_x}{\eps_\parallel n_z} \, .
\eeq
The angle $\theta'$ between optical axis and ray vector is given in terms of 
$\theta$, as defined in equation \eqref{eq:theta}:
\beq
    \tan \theta' = \frac{\eps_\perp}{\eps_\parallel} \tan \theta\, .
\eeq
We observe, that the directions of $\N$ and $\se$ aor only the same for 
$\theta = m \pi / 2$, $m \in mathbb{Z}$, 
thus for waves propagating parallel or perpendicular to the optical axis. 
\\

The polarization is analyzed taking yet another direction: 
Instead of describing the wave vector surface in principle axis 
coordinates of $\eps$, we can use coordinates corresponding to 
the fact, the $\D$ is transverse to $\N$. Taking one axis to be 
parallel to $\N$, we denote the other two directions with Greek letters. 
For the components of $\D$, we get 
\beq
    D_\alpha = n^2 E_\alpha
\eeq
and with the constitutive equation \eqref{eq:const1} 
\beq
    E_\alpha = {\eps^{-1}}_{\alpha \beta} D_\beta \, 
\eeq
we get the two dimensional eigenvalue problem
\beq
    {\eps^{-1}}_{\alpha \beta} D_\beta = n^{-2} D_\beta \, .
\eeq
In the case of no degeneracy, we obtain two orthogonal eigenvectors 
$\D_1$ and $\D_2$. Degeneracy is only present, if the components of 
$\eps^{-1}$ are all the same in its own principla axis coordinate system - 
this would just be the case of isotropic materials. 

We unite the results of the foregoing anaylsis with the following conclusion: 
$\N, \se, \E$ and $\D$ are always coplanar. For extraordinary waves, 
$\N$ and $\se$ are not parallel, but in the same principal section. 
The wave is thus polarized such that $\E$ and $\D$ lie in that principal 
section. As $\D$ for the ordinary waves of the same $\N$ as perpendicular 
to that of the extraordinary, their polarization is such that 
$\E$ and $\D$ are perpendicular to the principal section. 




\subsection{Deducing the Pockels effect in terms of nonlinear Maxwell equations}
\subsubsection{Primary and secondary electro-optical effect}
\subsubsection{Influence of electric field on the index ellipsiod}

\subsection{Structure of crystal lattice}

\subsection{Reduction of components of dielctric tensor for $\bar{4}2$m crystals}

\subsection{Discussion of the effect for the transverse Pockels Cell}

\subsection{Polarization filters}

\subsection{Methods of determining $U_\lambda$}
\subsubsection{Saw tooth}
\subsubsection{Modulated direct current}

\subsection{Experimental setup}
\subsubsection{Properties of He-Ne-Laser}
\subsubsection{Properties of used photodiode}
