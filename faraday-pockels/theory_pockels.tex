\section{Theory behind the Pockels effect}

\subsection{Linear and circular polarized light}

\subsection{Birefringence and index ellipsoid}
If the refraction index $n$ of a material depends on the linear polarization of light, 
then a beam of light propagating in the material will be split up into 
two beams with perpendicular polarization and different propagation speed 
\beq
   v_i = \frac{c}{n_i}, 
\eeq
In order to explain this phenomenon, it is helpful to introduce quantities 
connected to the structure of the material, i.~e. it being anisotropic, 
and retrieve the way the light propagates by looking at the solutions 
of Maxwell's equations in matter, 
\bea
\nabla \cdot \mathbf{D} &=& \rho_\text{f} 
\label{eq:max1} \\ 
\nabla \cdot \mathbf{B} &=& 0
\label{eq:max2} \\ 
\nabla \times \mathbf{E} &=& -\frac{\partial \mathbf{B}} {\partial t}
\label{eq:max3} \\ 
    \nabla \times \mathbf{H} &=& \mathbf{J}_\text{f} + 
        \frac{\partial \mathbf{D}} {\partial t}, 
\label{eq:max4} \\ 
\eea
where 
\bit
\item $\D$ is the electric displacement field, related to the eletric field $\E$ 
    by the constitutive equation 
    \beq
        \D = \epsilon \E 
    \label{eq:const1}
    \eeq
\item $\mathbf{H}$ is the magnetizing field, with the constitutive equation 
    \beq
        \mathbf{H} = \mu \B
    \label{eq:const2}
    \eeq
\item $\mathbf{J}_\text{f}$ is the free current density, and
\item $\mathbf{\rho}_\text{f}$ is the free charge density.
\eit
Equations \eqref{eq:const1} and \eqref{eq:const2} are valid for materials without 
coupling between magnetic and eletric fields, which is the case in our 
crystal. We are, however, facing an anistropic material, so that 
$\epsilon$ is a tensor, while we assume the permeability $\mu$ of the material 
to be just the vacuum permeability $\mu_0$, such that
$\B = \mu \mathbf{H} = \mu_0 \mathbf{H}$. 
If we assume $\mathbf{\rho}_\text{f}$, $\mathbf{\rho}_\text{f} = 0$ 
(no free charge and currents), then applying the curl 
operator to \eqref{eq:max3} yields:
\bea
    \nabla \times \nabla \times \mathbf{E} &=& 
    - \frac{\partial }{\partial t} \left(\nabla \times \mathbf{B}\right)  \nn
    &=& - \mu_0 \frac{\partial }{\partial t} \left(\nabla \times \mathbf{H}\right) \nn
    &=& - \mu_0 \frac{\partial }{\partial t} \left(\mathbf{J}_\text{f} + 
        \frac{\partial \D} {\partial t}\right) \nn
    &=& -\mu_0 \frac{\partial^2 \D }{\partial t^2}  \\
    \nabla \cdot \mathbf{D} &=& 0
\eea



biaxil - uniaxial 
optical axis 


\subsection{Deducing the Pockels effect in terms of nonlinear Maxwell equations}
\subsubsection{Primary and secondary electro-optical effect}
\subsubsection{Influence of electric field on the index ellipsiod}

\subsection{Structure of crystal lattice}

\subsection{Reduction of components of dielctric tensor for $\bar{4}2$m crystals}

\subsection{Discussion of the effect for the transverse Pockels Cell}

\subsection{Polarization filters}

\subsection{Methods of determining $U_\lambda$}
\subsubsection{Saw tooth}
\subsubsection{Modulated direct current}

\subsection{Experimental setup}
\subsubsection{Properties of He-Ne-Laser}
\subsubsection{Properties of used photodiode}
