\section{Theory behind the Pockels effect}

\subsection{Linear and circular polarized light}

\subsection{Birefringence and index ellipsoid}
If the refraction index $n$ of a material depends on the linear polarization of light, 
then a beam of light propagating in the material will be split up into 
two beams with perpendicular polarization and different propagation speed 
\beq
   v_i = \frac{c}{n_i}, 
\eeq
In order to explain this phenomenon, it is helpful to introduce quantities 
connected to the structure of the material, i.~e. it being anisotropic, 
and retrieve the way the light propagates by looking at the solutions 
of Maxwell's equations in matter, 
\bea
\nabla \cdot \D &=& \rho_\text{f} 
\label{eq:max1} \\ 
\nabla \cdot \B &=& 0
\label{eq:max2} \\ 
\nabla \times \E &=& -\frac{\partial \B} {\partial t}
\label{eq:max3} \\ 
    \nabla \times \mathbf{H} &=& \mathbf{J}_\text{f} + 
        \frac{\partial \D} {\partial t}, 
\label{eq:max4}
\eea
where 
\bit
\item $\D$ is the electric displacement field, related to the eletric field $\E$ 
    by the constitutive equation 
    \beq
        \D = \Eps \E 
    \label{eq:const1}
    \eeq
\item $\mathbf{H}$ is the magnetizing field, with the constitutive equation 
    \beq
        \mathbf{H} = \mu^{-1} \B
    \label{eq:const2}
    \eeq
\item $\mathbf{J}_\text{f}$ is the free current density, and
\item $\mathbf{\rho}_\text{f}$ is the free charge density.
\eit
Equations \eqref{eq:const1} and \eqref{eq:const2} are valid for materials without 
coupling between magnetic and eletric fields, which is the case in our 
crystal. We are, however, facing an anistropic material, so that 
$\Eps$ is a tensor, while we assume the permeability $\mu$ of the material 
to be just the vacuum permeability $\mu_0$, such that
$\B = \mu \mathbf{H} = \mu_0 \mathbf{H}$. 
If we assume $\mathbf{\rho}_\text{f} = 0$, $\mathbf{J}_\text{f} = 0$ 
(no free charge and currents), and consider plane waves, 
described by
\bea
    \E &=& \E_0 \exp \left[i(\mathbf{k \cdot r}-\omega t)\right] \\
    \mathbf{H} &=& \mathbf{H_0} \exp \left[i(\mathbf{k \cdot r}-\omega t)\right], 
\eea
then inserting into \eqref{eq:max3} and \eqref{eq:max4} yields:
\bea
    \mu_0 \omega \mathbf{H} &=& \K \times \E 
    \label{eq:plan_H} \\
    \omega \D &=& - \K \times \mathbf{H}.
    \label{eq:plan_D} 
\eea
We observe, that $\K, \D$, and $\Ha$ are mutually perpendicular. 
Looking at the energy flux
\beq
    \mathbf{S} = \E \times \Ha, 
\eeq
we further see, that the directions of $\mathbf{S}$ and $\K$ do not 
coincide if $\E \nparallel \Eps \E$. If we define the 
refractive index $\mathbf{n}$ by 
\beq
    \K = \frac{\omega}{c} \N \, ,
\eeq
we can rewrite equations \eqref{eq:plan_H} and \eqref{eq:plan_B} as 
\bea
    \mathbf{H} &=& \frac{1}{\mu_0 c} \N \times \E 
    \label{eq:plan_Hb} \\
    \D &=& - \frac{1}{c} \N \times \mathbf{H}.
    \label{eq:plan_Db} \, .
\eea
Inserting the latter one into the first and using 
the identity $c^2 = \frac{1}{\mu_0 \epsn}$, we get 
\bea
    \D  &=& \frac{1}{\mu_0 c^2} \N \times \left(\E \times \N\right) \nonumber \\
        &=& \epsn \left(n^2 \E - \left(\N \cdot \E\right) \N \right)
\eea
With the constitutive equation \eqref{eq:const1}, we obtain three 
equations linear in the components $E_k$:
\beq
    \left(n^2 \delta_{ik} - n_i n_k - \frac{\epsilon_{ik}}{\epsn}\right) E_k = 0
\eeq
This equation is solve if its determinant vanishes:
\beq
    \mathrm{det}\,\left|n^2 \delta_{ik} - n_i n_k - \frac{\epsilon_{ik}}{\epsn}\right| = 0
\eeq
By applying the principle axis theorem~\cite{strang2003introduction}, 
we can write $\Eps$ as the diagonal matrix with elements $\eps_x, \eps_y, \eps_z$. 
This yields \emph{Fresnel's equation}:
\bea
    \frac{n^2}{\epsn} \left(\eps_x n_x^2 + \eps_y n_y^2 + \eps_z n_z^2 \right) 
    - \left[
        \frac{n_x^2}{\epsn^2} \eps_x \left(\eps_y + \eps_z\right) + 
        \frac{n_y^2}{\epsn^2} \eps_y \left(\eps_z + \eps_x\right) + 
        \frac{n_z^2}{\epsn^2} \eps_z \left(\eps_x + \eps_y\right) 
    \right] + \nonumber \\
    + \quad \frac{\eps_x \eps_y \eps_z}{\epsn^3} = 0
    \label{eq:fresnel}
\eea
Being of second order in $n_i^2$, $i \in \{x, y, z\}$, there are up to two linearly 
independent solutions, corresponding to two possible directions of polarization. 
The case of a uniaxial crystal is especially easy to solve. If we take the 
$z$-axis to be that of rotational symmetry, we can rename the components of 
$\Eps$ with $\eps_x = \eps_y = \eps_\perp$ and $\eps_z = \eps_\parallel$. 
Fresnel's equation \eqref{eq:fresnel} can then be factored into
\beq
    \left(n^2 - \frac{\eps_\perp}{\epsn}\right) 
    \left[\frac{\eps_\parallel}{\epsn} n^2 + 
        \frac{\eps_\perp}{\epsn} \left(n_x^2 + n_y^2\right) + 
        \frac{\eps_\perp \eps_\parallel}{\epsn^2} 
    \right] = 0
\eeq



biaxil - uniaxial 
optical axis 


\subsection{Deducing the Pockels effect in terms of nonlinear Maxwell equations}
\subsubsection{Primary and secondary electro-optical effect}
\subsubsection{Influence of electric field on the index ellipsiod}

\subsection{Structure of crystal lattice}

\subsection{Reduction of components of dielctric tensor for $\bar{4}2$m crystals}

\subsection{Discussion of the effect for the transverse Pockels Cell}

\subsection{Polarization filters}

\subsection{Methods of determining $U_\lambda$}
\subsubsection{Saw tooth}
\subsubsection{Modulated direct current}

\subsection{Experimental setup}
\subsubsection{Properties of He-Ne-Laser}
\subsubsection{Properties of used photodiode}
