\section{Theory behind the Pockels effect}

\subsection{Light propagating in matter}
\paragraph{The basis of discussing} 
electro-optical effects are 
Maxwell's equations in matter~\cite{boyd2003nonlinear}:
\begin{subequations} 
\begin{align}
\nabla \cdot \D &= \rho_\text{f} 
\label{eq:max1} \\ 
\nabla \cdot \B &= 0
\label{eq:max2} \\ 
\nabla \times \E &= -\frac{\partial \B} {\partial t}
\label{eq:max3} \\ 
    \nabla \times \mathbf{H} &= \mathbf{J}_\text{f} + 
        \frac{\partial \D} {\partial t}, 
\label{eq:max4}
\end{align}
\label{eq:maxwell}
\end{subequations}
where 
\begin{itemize}
    \item $\D$ is the electric displacement field, related to the eletric field $\E$ 
    by the constitutive equation 
    \begin{equation}
        \D = \eps_0 \Eps \E 
    \label{eq:const1}
    \end{equation}
    where $\Eps$ is the dielectric tensor and $\eps_0$ is the permittivity of free 
    space; 
    \item $\mathbf{H}$ is the magnetizing field, with the constitutive equation 
    \begin{equation}
        \mathbf{H} = \mu^{-1} \, ;
    \label{eq:const2}
    \end{equation}
    \item $\mathbf{J}_\text{f}$ is the free current density, and
    \item $\mathbf{\rho}_\text{f}$ is the free charge density.
\end{itemize}
Equations \eqref{eq:const1} and \eqref{eq:const2} are valid for materials without 
coupling between magnetic and eletric fields, which is the case in our 
crystal. We are, however, facing an anistropic material, so that 
$\Eps$ is a tensor, while we assume the permeability $\mu$ of the material 
to be just the vacuum permeability $\mu_0$, such that
$\B = \mu \mathbf{H} = \mu_0 \mathbf{H}$. 
If we assume $\mathbf{\rho}_\text{f} = 0$, $\mathbf{J}_\text{f} = 0$ 
(no free charge and currents), we obtain the homogenious 
Maxwell equations. By applying the curl operator $\nabla \times$ 
to \eqref{eq:max3}, we get the wave equation 
\begin{equation}
    - \Delta \E + \frac{\eps}{\eps_0 c^2} \frac{\partial^2 \E}{\partial t^2}  = 0
    \label{eq:wave_eq}
\end{equation}
which possesses plane waves as solutions, 
described by
\begin{subequations}
\begin{align}
    \E &= \En \exp \left[i(\mathbf{k \cdot r}-\omega t)\right] \\
    \mathbf{H} &= \mathbf{H_0} \exp \left[i(\mathbf{k \cdot r}-\omega t)\right], 
\end{align}
\end{subequations}

Inserting into \eqref{eq:max3} and \eqref{eq:max4} yields:
\begin{subequations}
\begin{align}
    \mu_0 \omega \mathbf{H} &= \K \times \E 
    \label{eq:plan_H} \\
    \omega \D &= - \K \times \mathbf{H}.
    \label{eq:plan_D} 
\end{align}
\end{subequations}
We observe, that $\K, \D$, and $\Ha$ are mutually perpendicular. 
Looking at the energy flux
\begin{equation}
    \mathbf{S} = \E \times \Ha, 
\end{equation}
we further see, that the directions of $\mathbf{S}$ and $\K$ do not 
coincide if $\E \nparallel \Eps \E$. 

\paragraph{As a consequence}, 
plane waves can be polarized in any direction 
perpendicular to $\K$. Linear polarized waves conserve the 
direction of polarization in time and space. They are formed 
the superposing two waves with the same frequency and a 
relative phase $\Delta \phi = n \pi$, where $n \in \mathbb{Z}$. 
Any linearly polarized wave can be split up into two orthogonal 
components of equal amplitude, each forming an angle of $45^\circ$ 
with the original polarization.
A linear combination of two aves of the same frequency $\omega$, 
wavevector $\K$ but orthogonal polarization and a fixed phase 
difference $\Delta \phi \neq 0$ is generally polarized elliptically.
For the special case of equal amplitudes and $\Delta \phi = \pi /2$,
the polarization is circular. 

\paragraph{In order to asses} 
the differences in phase induces by the Pockels effect, we need to 
relate refraction to the phase. To do so, we look at two plane waves  
\begin{align}
\E_1(\mathbf{r}, t) = (\E_1)_0 \exp\left\{i(\K_1 \cdot \mathbf{r} - \omega t) \right\} \\ 
\E_2(\mathbf{r}, t) = (\E_2)_0 \exp\left\{i(\K_2 \cdot \mathbf{r} - \omega t) \right\} \\ 
\end{align}
with the same frequency $\omega$,
we can define $\Delta \phi$ at 
one time $t$ and position $\mathbf{r}$ by 
\begin{equation}
    \Delta \phi := \left(\K_2 - \K_1 \right) \cdot \mathbf{r}\, .
\end{equation}
By the definition of $\N$ \eqref{eq:def_n} and the identity 
\begin{equation}
    \omega = \frac{2 \pi c}{\lambda} \, ,
\end{equation}
we can write 
\begin{equation}
    \Delta \phi = \frac{2 \pi}{\lambda} 
    \left(\N_2 - \N_1 \right) \cdot \mathbf{r} \, .
\end{equation}
If we further assume $\N$ to be parallel to $\mathbf{r}$, 
and measure $\Delta \phi$ after the two waves passed 
through a crystal of lenght $l$, then we get
\begin{equation}
    \Delta \phi_l = \frac{2 \pi}{\lambda} 
    \left(n_2 - n_1 \right) l \, .
    \label{eq:dphi}
\end{equation}


\subsection{Birefringence}
\paragraph{If the refraction index}
 $n$ of a material depends on the linear polarization of light, 
then a beam of light propagating in the material will be split up into 
two beams with perpendicular polarization and different propagation speed 
\begin{equation}
   v_i = \frac{c}{n_i}, 
\end{equation}
In order to explain this phenomenon, it is helpful to examine the 
structure of the material and connected quantities, as well as 
to retrieve the way light propagates by looking at the solutions 
of Maxwell's equations in matter \eqref{eq:maxwell}.
If we define the 
refractive index $\mathbf{n}$ by 
\begin{equation}
    \K = \frac{\omega}{c} \N \, ,
    \label{eq:def_n}
\end{equation}
we can rewrite equations \eqref{eq:plan_H} and \eqref{eq:plan_D} as 
\begin{subequations}
\begin{align}
    \mathbf{H} &= \frac{1}{\mu_0 c} \N \times \E 
    \label{eq:plan_Hb} \\
    \D &= - \frac{1}{c} \N \times \mathbf{H}
    \label{eq:plan_Db} \, .
\end{align}
\end{subequations}
Inserting the latter one into the first and using 
the identity $c^2 = \frac{1}{\mu_0 \eps_0}$, we get 
\begin{equation}
    \begin{split}
    \D  &= \frac{1}{\mu_0 c^2} \N \times \left(\E \times \N\right) \\
        &= \eps_0 \left(n^2 \E - \left(\N \cdot \E\right) \N \right)
    \label{eq:D(E)}
    \end{split}
\end{equation}
With the constitutive equation \eqref{eq:const1}, we obtain three 
equations linear in the components $E_k$:
\begin{equation}
    \left(n^2 \delta_{ik} - n_i n_k - \epsilon_{ik}\right) E_k = 0
\end{equation}
This equation is solve if its determinant vanishes:
\begin{equation}
    \mathrm{det}\,\left|n^2 \delta_{ik} - n_i n_k - \epsilon_{ik}\right| = 0
\end{equation}
By applying the principle axis theorem~\cite{strang2003introduction}, 
we can write $\Eps$ as the diagonal matrix with elements $\eps_x, \eps_y, \eps_z$. 
This yields \emph{Fresnel's equation}:
\begin{align}
    n^2 \left(\eps_x n_x^2 + \eps_y n_y^2 + \eps_z n_z^2 \right) 
    - \left[
        n_x^2 \eps_x \left(\eps_y + \eps_z\right) + 
        n_y^2 \eps_y \left(\eps_z + \eps_x\right) + 
        n_z^2 \eps_z \left(\eps_x + \eps_y\right) 
    \right] + \nonumber \\
    + \quad \eps_x \eps_y \eps_z = 0
    \label{eq:fresnel}
\end{align}
Being of second order in $n_i^2$, $i \in \{x, y, z\}$, there are up to two linearly 
independent solutions, corresponding to two possible directions of polarization. 


\paragraph{The case of a uniaxial crystal} 
is especially easy to solve. 
If we take the $z$-axis to be that of rotational symmetry, 
we can rename the components of 
$\Eps$ with 
\begin{align}
\eps_x &= \eps_y = \eps_\perp 
\label{eq:eps_perp}\\
\eps_z &= \eps_\parallel \, .
\label{eq:eps_parallel}
\end{align}
The $z$-axis is also called the \emph{optical axis}.
Fresnel's equation \eqref{eq:fresnel} can then be factored into
\begin{equation}
    \left(n^2 - \eps_\perp\right) 
    \left[\eps_\parallel n_z^2 + 
        \eps_\perp \left(n_x^2 + n_y^2\right) -
        \eps_\perp \eps_\parallel 
    \right] = 0 \, .
\end{equation}
In other words, we have two quadratic equations
\begin{align}
    n^2 &= \eps_\perp
    \label{eq:sphere} \\
    n_z^2 \eps_\perp + \left(n_x^2 + n_y^2\right) \eps_\parallel &= 1
    \label{eq:ellipsoid}
\end{align}
Equation \eqref{eq:sphere} gives a sphere for the wave-vector surface. To the 
corresponding types of waves, the crystal behaves like an isotropic body, being 
characterized by the refractive index $n = \sqrt{\eps_\perp}$. These waves are 
called \emph{ordinary waves}. The second type, accordingly named 
\emph{extraordinary waves}, is confined by the ellipsoid \eqref{eq:ellipsoid}. 
It's magnitude depends on 
the angle $\theta$ to the optical axis by 
\begin{equation}
    \frac{1}{n^2} =  \frac{\sin^2 \theta}{\eps_\parallel} - 
        \frac{\cos^2 \theta}{\eps_\perp} \, .
    \label{eq:theta}
\end{equation}

\paragraph{In order to relate} 
these two surfaces to two directions of polarization and 
determine the direction of a ray, we introduced the \emph{ray vector} $\mathbf{s}$, 
which is defined by the direction of the group velocity
\begin{equation}
    \frac{\partial \omega}{\partial \K}
\end{equation}
and the magnitude fullfilling 
\begin{equation}
    \mathbf{s} \cdot \N = 1\,.
\end{equation}
The ray vector defines the \emph{ray surface} by the condition 
$\phi = \mathrm{const.}$ 
for all $\mathbf{s}$ on that surface, where $\phi$ is the phase of the wave. 

We can show that $\mathbf{s} \parallel \mathbf{S}$: Differentiating 
equations \eqref{eq:plan_Hb} and \eqref{eq:plan_Db}, we get 
\begin{subequations}
\begin{align}
    c \delta \D &= \delta \Ha \times \N + \Ha \times \delta \N \\
    \mu_0 c \delta \Ha &= \N \times \delta \E + \delta \N \times \E \, ,
\end{align}
\end{subequations}
which by multiplying with $\E$ and $\Ha$, respectively, and applying 
basic properties of vector and scalar products, yield
\begin{subequations}
\begin{align}
    \begin{split}
    c \E \cdot \delta \D 
    &= \E \cdot \left(\delta \Ha \times \N \right) + 
        \E \cdot \left(\Ha \times \delta \N \right) \\
    &= \delta \Ha \cdot \left(\N \times \E \right) + 
        \delta \N \cdot \left(\E \times \Ha \right)  \\
    &= \mu_0 c \Ha \cdot \delta \Ha + 
        \delta \N \cdot \left(\E \times \Ha \right)
    \end{split} \\
    \mu_0 c \Ha \cdot \delta \Ha &= c \D \cdot \delta \E + 
        \delta \N \cdot \left(\E \times \Ha \right) \, ,
\end{align}
\end{subequations}
using again \eqref{eq:plan_Hb} and \eqref{eq:plan_Db}. It follows directly, that 
\begin{equation}
    \delta \N \cdot \left( \E \times \Ha \right) = \delta \N \cdot \mathbf{S} = 0 \, .
\end{equation}
To show, that $\se$ and $\mathbf{S}$ have the same direction, we need to show, 
that $\se$ is orthogonal on the surface of the wave vector surface, or, since 
the infinitesimal displacement $\delta \N$ lies 
on that surface, that $\se \cdot \delta \N = 0$. 
The wave vector surface given by \eqref{eq:fresnel} can be described by 
$f(\omega, \K) = 0$, which implies 
\begin{equation}
    \frac{\partial \omega}{\partial \K} = 
    - \frac{\partial f / \partial \K}{\partial f / \partial \omega}
\end{equation}
for the group velocity and thus 
\begin{equation}
    \se \parallel \frac{\partial f }{\partial \K} \parallel \frac{\partial f}{\partial \N}
    \label{eq:s_dir}
\end{equation}
since the gradient along $\K$ is taken for constant $\omega$. $\partial f / \partial \N$, however, 
is normal to the surface $f = 0$, so we can conclude, that
\begin{align}
    \se \cdot \delta \N &= 0 \\
    \Rightarrow \qquad \se &\parallel \mathbf{S} \, .
\end{align}
If follows, that 
\begin{equation}
    \se \cdot \Ha = 0 \qquad \se \cdot \E = 0 \, .
    \label{eq:coplanar}
\end{equation}
For uniaxial crystals, we can specify the ray surface in a form similar to 
\eqref{eq:ellipsoid} by doing analoge calculations with the folloing substitutions:
\begin{equation}
    \E \leftrightarrow c \D \, ; 
    \qquad \N \leftrightarrow \mu_0 c \se \, ; 
    \qquad \eps_{ik} \leftrightarrow {\eps^{-1}}_{ik} \, .
\end{equation}
The result is 
\begin{align}
    s^2 &= \eps_\perp
    \label{eq:s_sphere} \\
    {s_z}^2 \eps_\perp + \left( {s_x}^2 + 
        {s_y}^2\right) \eps_\parallel &= 1 \, .
    \label{eq:s_ellipsoid}
\end{align}
Since $\eps_x = \eps_y$, $\N$ and $\se$ must be coplanar with the optical axes. This 
common plane is called \emph{principal axis} for a given $\N$. Let this plane be 
defined by the $xz$-plane. We can get the direction 
of $\se$ from the relation \eqref{eq:s_dir}, namely by taking the derivatives of 
\eqref{eq:ellipsoid} with respect to $n_x$ and $n_z$:
\begin{equation}
    \frac{s_x}{s_z} = \frac{\eps_\perp n_x}{\eps_\parallel n_z} \, .
\end{equation}
The angle $\theta'$ between optical axis and ray vector is given in terms of 
$\theta$, as defined in equation \eqref{eq:theta}:
\begin{equation}
    \tan \theta' = \frac{\eps_\perp}{\eps_\parallel} \tan \theta\, .
\end{equation}
We observe, that the directions of $\N$ and $\se$ aor only the same for 
$\theta = m \pi / 2$, $m \in \mathbb{Z}$, 
thus for waves propagating parallel or perpendicular to the optical axis. 


\paragraph{The polarization} is analyzed taking yet another direction: 
Instead of describing the wave vector surface in principle axis 
coordinates of $\eps$, we can use coordinates corresponding to 
the fact, the $\D$ is transverse to $\N$. Taking one axis to be 
parallel to $\N$, we denote the other two directions with Greek letters. 
For the components of $\D$, we get from equation \eqref{eq:D(E)}
\begin{equation}
    D_\alpha = \epsn n^2 E_\alpha
\end{equation}
and with the constitutive equation \eqref{eq:const1} 
\begin{equation}
    E_\alpha = \frac{1}{\epsn} {\eps^{-1}}_{\alpha \beta} D_\beta \, 
\end{equation}
we get the two dimensional eigenvalue problem
\begin{equation}
    {\eps^{-1}}_{\alpha \beta} D_\beta =  n^{-2} D_\beta \,
\end{equation}
for the two-dimensional symmetric tensor $\eps^{-1}$. 
In the case of no degeneracy, we obtain two orthogonal eigenvectors 
$\D_1$ and $\D_2$. Degeneracy is only present, if the components of 
$\eps^{-1}$ are all the same in its own principla axis coordinate system - 
this would just be the case of isotropic materials. 


\paragraph{We unite the results} of the foregoing anaylsis with the 
following conclusion: 
$\N, \se, \E$ and $\D$ are always coplanar. For extraordinary waves, 
$\N$ and $\se$ are not parallel, but in the same principal section. 
The wave is thus polarized such that $\E$ and $\D$ lie in that principal 
section. As $\D$ for the ordinary waves of the same $\N$ as perpendicular 
to that of the extraordinary, their polarization is such that 
$\E$ and $\D$ are perpendicular to the principal section. 

\paragraph{For biaxial crystals,}
the solutions are more complicated and 
treated for example in \cite{landau1984electrodynamics}. 
The fourth order surface defined by Fresnel's equation 
\eqref{eq:fresnel} cannot be separated into two 
simple geometric figures any more, but form a 
complex surface with exactly four points of selfintersection
~\cite{born1999principles}. 
There are no more ordinary rays, but only 
extraordinary ones. We can find two 
optical axes as point of self intersection of the 
surface, usually obtainted by looking at the intersections 
the the coordinate planes in the principla axis basis.
In order to do so, we set to zero one of the components 
in Fresnel's equation \eqref{eq:fresnel}. For the 
$xy$-plane, we set $n_z = 0$ and obtain
\begin{equation}
    \left(n^2 - \eps_z\right) 
    \left(\eps_x n_x^2 + 
        \eps_y n_x^2  -
        \eps_x \eps_y 
    \right) = 0 \, .
\end{equation}
with the solutions
\begin{align}
    n^2 &= \eps_z 
    \label{eq:sphere} \\
    \frac{{n_x}^2}{ \eps_y} + \frac{{n_y}^2}{ \eps_x} &= 1
    \label{eq:ellipsoid}
\end{align}
and the analogues for $n_y = 0$ and $n_x = 0$, obtained by 
interchanging $x, y$ and $z$. If we assume 
\begin{equation}
    \eps_x < \eps_y < \eps_z, 
\end{equation}
we see that the intersects appear only in the $xz$-plane, as shown in 
figure \ref{fig:biaxial_ellipsoid}.
\begin{figure}
\includegraphics[width=\pltw]{figures/biaxial_ellipsoid.pdf}
\caption{Surface defined by Fresnel's equation for biaxial crystals.
    The lines are shown only on the planes parallel to the principle 
    axis of the crystal with $\eps_x < \eps_y < \eps_z$.
    The point of intersection defines the optical axis, 
    shown as a dashed line. 
    Taken from \cite{landau1984electrodynamics}. }
\label{fig:biaxial_ellipsoid}
\end{figure}
The optical axes are now defined by the straight lines intersecting 
the self intersects of the surface and the origin. 
Wave vector $\K$ and ray vector $\se$ have the same direction, 
only if they are orientated along one of the principle axes, 
otherwise they split up into two rays. 
Further, if $\K$ lies in one of the coordinate planes, so does $\se$.
There is, however an important exception: If the wave vector 
coincides with one of the optical axes, there are 
infinitely many ray vectors associated with it:
be can observe the phenomenon of 
\emph{internal conical refraction}. 
\FloatBarrier

\subsection{The electro-optic effect}
\paragraph{The index ellipsoid} 
as one of the two solutions for Fresnel's equation 
\eqref{eq:fresnel} for uniaxial crystals can be defined by the 
index ellipsoid, 
\begin{equation}
    \eta_{ij}^{(0)} n_i n_j = 1 \, ,
\end{equation}
where $\eta^{(0)} = {\eps^{(0)}}^{-1}$ is the \emph{impermeability 
tensor} of the unpertubed crystal with dielectric tensor $\eps^{(0)}$ 
and we sum over equal indices. 
Applying an electric field $\E$ on the crystal, $\eta^{(0)}$ will be 
changed to $\eta$. For reasonably small fields, we can expand
$\eta$ in powers of $\E$ as 
\begin{equation}
    \eta_{ij} = \eta_{ij}^{(0)} + r_{ijk} E_k + s_{ijkl} E_k E_l + \ldots
\end{equation}
The linear electro-optic effect is thus described by the 
\emph{electro-optic tensor} $r$, while 
$s$ is associtated with the quadratic electro-optic effect, etc. 
In this experiment, we will only look at effects of first order, 
since the associated second order \emph{Kerr effect} is 
neglectible for the given conditions~\cite{staatsexamen}.
Some basic properties of $r$ can be deduced immediately by 
looking at those of $\eps$: Since $\eps$ is real and symmetric, the 
same has to apply to $\eta$. Thus, $r$ has to be symmetric in the 
first two indices:
\begin{equation}
    r_{ijk} = r_{jik}
\end{equation}
We introduce a different notation as a $6 \times 3$ matrix with the definitions 
\begin{align}
    &r_{11k} = r_{1k}\, ,&
    &r_{22k} = r_{2k}\, ,&
    &r_{33k} = r_{3k}\, ,& \nonumber \\
    &r_{23k} = r_{32k} = r_{4k}\, ,& 
    &r_{13k} = r_{31k} = r_{5k}\, ,&
    &r_{12k} = r_{21k} = r_{6k} &
    \label{eq:notation}
\end{align}
for $k = 1, 2, 3$. The coefficient has a order of about 
$10^{-10}$ to $10^{-12}$ m/V.~\cite{sauter1996nonlinear}

\paragraph{The electro-optic tensor}
$r$ can be splitted into two components:
The \emph{primary} electro-optic effect with coefficient $r_{ij}'$
for the case of zero strain, where the crystal is 
not able to deform. The \emph{secondary} effect corresponds to deformation 
due to photoelastic and piezoelectric effects, described by the 
respective coefficients $d_{jk}$ and $p_{ik}$.
We can write
\begin{equation}
    r_{ij} = r_{ij}' + p_{ik} d_{jk}.
\end{equation}
In order to separate the effects, we apply an external field $\E$ 
oscillating at a frequency such that the crystal is not able 
to follow. In this case, the secondary electro-optical effect is surpressed, 
so we expect to measure $r'$. 

\subsection{Structure of crystals}

\subsubsection{Crystal systems}
\paragraph{Crystals can be categorized} 
into groups by their symmetries. 
There are three different categorizations in use, being 
partly equivalent and, accordingly introducing some confusion. 
The seven \emph{crystal system} divides crystals into 
according to the \emph{point group} of highest symmetry, 
the \emph{lattice system} 
categorizes according to the \emph{Bravais lattices}. While 
five of the crystal systems can be identified with five 
lattice systems, the hexagonal and trigonal crystal systems 
are different from the hexagonal and rhombohedral lattice systems. 
The \emph{crystal family} as a third categorization unifies 
the latter two and is thus made up of six elements. 

\paragraph{The lattice system} 
is based on the restrictions on the axial 
system, defined by the three 
\emph{primitive vectors} $\mathbf{a, b, c}$, which 
by the lattice translation operator 
\begin{equation}
    \mathbf{R} = n_1 \mathbf{a} + n_2 \mathbf{b} + n_3 \mathbf{c}
\end{equation}
define the \emph{Bravais lattice}. Each endpoint of a vector 
can be associated to one atom in the lattice of the crystal. 
Restricting only lenghts $a, b, c$ and angles $\alpha, \beta, \gamma$ 
between the primitive vectors, 
one obtaines the six crystal families, which by separating the 
family for $a = b \neq c, \alpha = \beta = 90^\circ, \gamma = 120^\circ$ 
into either hexagonal and trigonal system 
or hexagonal and rhombohedral system yields the crystal or lattice 
system, respectively. And overview is given in table 
\ref{tab:crystal_systems}. The hexagonal system is characterized by 
a sixfold rotational or rotoinversional symmetry, while the 
trigonal has a threefold rotational symmetry. 

\renewcommand{\arraystretch}{1.5}
\begin{table}[htdp]
    \begin{tabular}{|p{0.37\textwidth}|p{0.61\textwidth}|}
        \hline
        \rowcolor{LightCyan}
        $\textbf{Crystal system}$ & 
        $\textbf{Restrictions on the axial system   }$ \\ 
        \hline
        Trigonal        & $a \neq b \neq c; \quad \alpha = \beta = \gamma$ \\ 
        Monoclinic      & $a \neq b \neq c; \quad \alpha = \gamma = 90\Deg, \beta > 90\Deg$ \\ 
        Orthorhombic    & $a \neq b \neq c; \quad \alpha = \beta = \gamma = 90\Deg$         \\
        Tetragonal      & $a    = b \neq c; \quad \alpha = \beta = \gamma = 90\Deg$         \\ 
        Trigonal        & $a    = b \neq c; \quad \alpha = \beta = 90\Deg, \gamma = 120\Deg$ \\ 
        Hexagonal       & $a    = b \neq c; \quad \alpha = \beta = 90\Deg, \gamma = 120\Deg$ \\ 
        Cubic           & $a    = b =    c; \quad \alpha = \beta = \gamma = 90\Deg$         \\ 
        \hline
    \end{tabular}
\caption{
    The seven crystal systems, characterized by the restrictions on the axial system. 
    The trigonal and hexagonal systems are characterized by threefold and sixfold rotational 
    symmetries. 
    Reference: \cite{borchardt1995crystallography}
    }
\label{tab:crystal_systems}
\end{table}

\paragraph{Another characterization of crystals} 
can be done by the point groups. 
If we discard all lattice translations are discarded, we get 
a remaining set of symmetry operations on one of the lattice points, 
the so-called \emph{point group}. The symmetry operations $n, \bar{n}$ and $m$  
go through that point, while inversion ($\bar{1}$) is defined at the point. 
There are 32 point groups, which can be further ordered hierarchically to obtain 
the crystal groups. 
The space groups of highest symmetry for each crystal system is identified 
with one point group. As higher symmetries 
imply lower ones (e.~g. $4/m 2/m 2/m$ implies $2/m$), the lower 
symmetries often apply to several systems. However, the identification 
with the highest symmetry remains well defined. In table \ref{tab:point_groups}, 
we give an overview of the crystals and their associated point groups as 
well as subgroups.
The point groups are often named by 
the \emph{international} or \emph{Hermann-Mauguin notation} 
This convention allows to comprise all information about the point group in 
three symbols.
The notation follows the following rules~\cite{sands1993introduction}:
\begin{itemize}
    \item
    Each component corresponds to a different direction. 
    \item
    A number $n \in \mathbb{N}$ corresponds to a $n$-fold rotational symmetry 
    around the given axis. For crystals, $n$ is restricted to $1, 2, 3, 4, 6$.
    \item
    A number with an overbar $\bar{n}$ corresponds 
    to a rotoinversion as one operation. The special case $\bar{1}$ thus 
    means a simple inversion without rotation, i.~e. the existence of 
    an inversion center. 
    \item
    $m$ indicates a mirror plane, its direction is th normal to this plane.
    Terms written as $n/m$ thus indicate rotational symmetry and a mirror plane 
    normal to the axis of rotation, and are interpreted as one component. 
    \item
    For orthorhombic systems, all directions are mutually perpendicular. 
    The symbols refer to the $x$, $y$, and $z$ axis, respectively.
    \item
    For tetragonal systems, we chose the first component to be the $z$-axis, 
    with the symbol $4$ or $\bar{4}$. The second component refers to the 
    mutually perpendicular $x$- and $y$-axes and the third to directions in the 
    $xy$-plane bisecting the angles between $x$ and $y$. 
    \item
    In trigonal and hexagonal systems, the second component refers to 
    equivalent directions ($120^\circ$ or $60^\circ$ apart) in the $xy$-plane, 
    which is normal to the $z$-axis with $3, \bar{3}, 6$ or $\bar{6}$ symmetry. 
    \item
    For hexagonal systems, a third component corresponds to directions bisecting 
    the angles between axes specified by the second component. 
    \item
    If the second term is a $3$, then the system is cubic. The $3$ refers to the 
    four body diagonals of the cube, the first symbol refers to the cube axes, and the 
    third component to the face diagonals.
    \item
    If two or more axes coincide, the higher symmetry (generating more points) is 
    shown. 
\end{itemize}

\begin{table}[htdp]
    \begin{tabular}{|p{0.37\textwidth}|p{0.23\textwidth}|p{0.37\textwidth}|}
        \hline
        \rowcolor{LightCyan}
        $\textbf{Crystal system}$ & 
        \multicolumn{2}{|l|}{$\textbf{Point groups   }$} \\ 
        \cline{2-3}
        \rowcolor{LightCyan}
        &$\textbf{Highest symmetry }$ & $\textbf{subgroups}$\\ 
        \hline
        Trigonal        & $ \bar{1} $   & $ 1 $  \\ 
        Monoclinic      & $\frac{2}{m}$ & $m, 2$     \\ 
        Orthorhombic    & $\frac{2}{m} \frac{2}{m} \frac{2}{m}$ & $mm2, 222$     \\
        Tetragonal      & $\frac{4}{m} \frac{2}{m} \frac{2}{m}$ & $\bar{4}2m, 4mm, 422, \frac{4}{m}, \bar{4}, 4$     \\ 
        Trigonal        & $\bar{3} \frac{2}{m}$         & $3m, 32, \bar{3}, 3$      \\ 
        Hexagonal       & $\frac{6}{m} \frac{2}{m} \frac{2}{m}$ & $\bar{6}m2, 6mm, 622, \frac{6}{m}, \bar{6}, 6$     \\ 
        Cubic           & $\frac{4}{m}\bar{3}\frac{2}{m}$ & $\bar{4}3m, 432, \frac{2}{m}\bar{3}, 23 $     \\ 
        \hline
    \end{tabular}
\caption{
    Point groups, their subgroups and associated crystal systems, 
    written in the \emph{international notation}.
    Reference: \cite{borchardt1995crystallography}
    }
\label{tab:point_groups}
\end{table}

\paragraph{The $\bar{4}2m$-symmetry}
under consideration in this experiment corresponds to 
\begin{itemize}
    \item
    a fourfold rotoinversional symmetry around the $x_3$-axis
    \item
    a twofold rotational symmetry around the $x_1$-axis
    \item
    and a mirror plane parallel to the $x_3$-axis and the bisection of the 
    angles between $x_1$- and $x_2$-axis, 
\end{itemize}
where we renamed the axes $x, y$ and $z$ into $x_1, x_2$ and $x_3$, respectively. 
The $\bar{4}$-axis includes another mirror 
plane perpendicular to the first one and a twofold rotational symmetry 
around the $x_3$-axis. 
The structure is usually called tetragonal scalenohedral. 

\paragraph{Point groups can further} 
be illustrated by their \emph{stereographic projections},
that is constructed as follows
~\cite{borchardt1995crystallography}: 
We take the plane going through the origin 
of the point group and being perpendicular to the axis of highest 
symmetry to be the plane of projection. We further construct a sphere 
of arbitry radius and draw straight lines from the origin through the 
centers of the crystal's faces. If we take all intersections of these lines 
with the surface of the upper hemisphere and draw lines to the south pole, 
the intersects with the projection plane give the desired stereographic projection. 
The process is illustrated in the figures \ref{fig:stereo}. 
The projection can then be further reduced to the \emph{assymetric face unit}, 
which is the smallest unit of surface of the sphere, which will generate 
the entire sphere by application of all symmetry operations
~\cite{borchardt1995crystallography}.
For the $\bar{4}2m$-crystals, both the stereogram and the asymmteric face unit are 
shown in figure \ref{fig:stereo_42m}.

\begin{figure}[h]
    \begin{subfigure}[b]{\picwidth}
    \includegraphics[width=\textwidth]{figures/stereo_1}
    \caption{}
    \label{fig:stereo_1}
    \end{subfigure}\qquad
    \begin{subfigure}[b]{\picwidth}
    \includegraphics[width=\textwidth]{figures/stereo_2}
    \caption{}
    \label{fig:stereo_2}
    \end{subfigure}
    \begin{subfigure}[b]{\picwidth}
    \includegraphics[width=\textwidth]{figures/stereo_3}
    \caption{}
    \label{fig:stereo_3}
    \end{subfigure}
    \caption{
        Illustration of the stereographic projection:\\
        \ref{fig:stereo_1}:
            Crystal at the center of the sphere with points of intersections, 
            here shown for galena (PbS).
        \ref{fig:stereo_2}:
            Construction of the projection with the 
            plane of projection and south pole.
        \ref{fig:stereo_3}:
            Result of stereographic projection for PbS. The lines 
            show the interconnecting great circles projected on the plane. 
            All figures taken from~\cite{borchardt1995crystallography}.}
    \label{fig:stereo}
\end{figure}

\begin{figure}
    \centering
    \begin{subfigure}[b]{0.36\textwidth}
    \includegraphics[width=\textwidth]{figures/stereogram_42m}
    \caption{}
    \label{fig:stereogram_42m}
    \end{subfigure}\qquad
    \begin{subfigure}[b]{0.45\textwidth}
    \includegraphics[width=\textwidth]{figures/asym_face_unit_42m}
    \caption{}
    \label{fig:asym_face_unit_42m}
    \end{subfigure}
    \caption{
        Stereogram (\ref{fig:stereogram_42m}) and asymmetric face unit
        \ref{fig:asym_face_unit_42m} of the $\bar{4}2m$ point group,
        to which the ADP crystal used in this experiment belongs. 
        The symbols are 
        defined as follows: The square with diagonal corresponds 
        to the fourfold rotoinversion axis $\bar{4}$ perpendicular 
        to the plane, the small oval to the twofold rotation axis 
        $2$, which lies in the plane. The solid line at $45\Deg$
        corresponds to the mirror plane $m$. 
        Application of all symmetry operations on the shaded area 
        yield the entire stereographic projection. 
        Taken from~\cite{borchardt1995crystallography}.
        }
    \label{fig:stereo_42m}
\end{figure}



\FloatBarrier

\subsubsection{Optical classification of crystals}
\label{sec:optical_axes}
Crystals can be classified into three groups according to their 
optical properties~\cite{born1999principles}. 
\begin{itemize}
    \item 
    Group I - \emph{Isotropic crystals}:\\
    Crystals of the cubic system, for which $\eps_{ij} = \eps \delta$, and 
    $\eps$ is a scalar, meaning that three crystallographically 
    equivalent and mutually perpendicular axes can be chosen freely. 
    Optically, these crystals are equivalent to isotropic bodies. 
    \item
    Group II - \emph{Uniaxial crystals}: \\
    Crystals not belonging to group I, but for which one can choose 
    two or more crystallographically equivalent axes in a plane. 
    These crystals are in the rhobohedral, tetragonal or hexagonal system. 
    The optical axes coincides with the threefold, 
    fourfold or sixfold axis of symmtry, respectively, while the 
    other two axes arbitrary
    \item
    Group III - \emph{Biaxial crystals}: \\
    Crystals in which no two crystallographically equivalent axes may 
    be chosen, belonging to the triclinic, monoclinic, or orthorhombic 
    system. For the triclinic crystals, principla dielectric axes are 
    unrelated to the crystallographic  axes and vary with frequency. 
    For monoclinic ones, only one principle dielectric axis is crystallographically 
    fixed. It coincides with the twofold axes of symmetry, 
    or is perpendicular to the plane of symmetry, while other two axes 
    vary with frequency. 
    In orthorhombic crystals, all three principly dielectric axes are fixed. 
    They coincide with three mutually perpendicular twofold axes of symmetry.
\end{itemize}
A survey of all possible case is given in table \ref{fig:crystal_ellipsoid}.
\begin{table}
\includegraphics[width=\pltw]{figures/crystal_ellipsoid.pdf}
\caption{
    Table summarizing the optical properties of crystals, grouping them 
    according to the principle axes of the dieletric tensor $\eps$. 
    Two vectors at a small angle are indicating frequency-dependency 
    of the correspoding principle axes (signifying positions for two different 
    frequencies/colours). Axes that can be rotated freely in a plane are represented 
    by dashed vectors. 
    From \cite{born1999principles}. }
\label{fig:crystal_ellipsoid}
\end{table}

\FloatBarrier

\subsection{Reduction of Dielctric tensor}
\label{sec:reduce}
\paragraph{As described in} 
the previous section, the ADP-crystal has the symmetry group 
$\bar{4}2m$. We can reduce the number of non-zero components of the 
electro-optic tensor $r_{ijk}$ by the following consideration:
If we apply a coordinate transform $R$ to our basis for which 
our crystal is symmetric, then $r_ijk$ has to have the same symmetry, 
i.~e. the repsresentation of $r_ijk$ in the new basis has to be the same.
We define $R$ with its action on the basis by 
\begin{equation}
    \mathbf{\hat{e}}_i = R_i^j \mathbf{e}_j \, ,
\end{equation}
where $\mathbf{\hat{e}}_i$ and $\mathbf{e}_i$ are 
the $i$-th components of the new and old basis, respectively. 
Since $r_{ij}$ is a tensor, the components $\hat{r}_{ijk}$ in the 
new basis can be calculated from those in the old one by 
\begin{equation}
    \hat{r}_{ijk} = 
        \left( R^{-1}\right)_i^l
        \left( R^{-1}\right)_j^m
        \left( R^{-1}\right)_k^n
        r_{lmn} \, .
\end{equation}


\paragraph{Before applying the symmetries}
of the crystal under consideration 
we make one general observation: Crystals with an inversion 
center are not subject to either the Pockels effect. 
This can be deduced as follows:
As a center of inversion means symmetry under the transformation 
\begin{equation}
    \begin{split}
    x_1 &\rightarrow -x_1 \\
    x_2 &\rightarrow -x_2 \\
    x_3 &\rightarrow -x_3  \, ,
    \end{split}
\end{equation}
all components in the new bases $\hat{r}_{ijk}$ will be defined 
by 
\begin{equation}
   \hat{r}_{ijk} = - r_{ijk}.
\end{equation}
But symmetry implies equality. Thus all components have to be zero. 
The same reasoning applies to other effects with a similar mathematical 
structure, such as the piezoelectric effect. 


\paragraph{Returning to the} $\bar{4}2m$ group, 
we eliminate the components of $r_{ijk}$ by applying sequentially 
the following transformations, for which the crystal is symmetric:
\begin{itemize}
    \item
    rotation by $180^\circ$ around the $x_3$-axis;
    \item
    rotation by $90^\circ$ around the $x_3$-axis and inversion;
    \item
    rotation by $180^\circ$ around the $x_1$-axis. 
\end{itemize}
For the first rotation, the coordinates transfer with 
\begin{equation}
    \begin{split}
    x_1 &\rightarrow -x_1 \\
    x_2 &\rightarrow -x_2 \\
    x_3 &\rightarrow x_3  \, .
    \end{split}
\end{equation}
The matrix representation is given by
\begin{equation}
    R^{-1} = 
    \begin{pmatrix}
    -1 &  0 &  0 \\
     0 & -1 &  0 \\
     0 &  0 &  1 
    \end{pmatrix} \, .
\end{equation}
We calculate the first component in an examplary manner, 
\begin{equation}
    \begin{split}
    \hat{r}_{111} &= 
        \left( R^{-1}\right)_1^l
        \left( R^{-1}\right)_1^m
        \left( R^{-1}\right)_1^n
        r_{lmn}  \\
        &= (-1)(-1)(-1)r_{111} \\
        &= -r_{111} \, .
    \end{split}
\end{equation}
One can see immediately that for all components an odd number 
of indices $i \in \{1, 2\}$ yields $\hat{r}_{ijk} = -r_{ijk}$, 
while an even number yields $\hat{r}_{ijk} = r_{ijk}$. 
Since the condition of symmtry implies $\hat{r}_{ijk} = r_{ijk}$, 
we can summarize all components which do not have to equal zero:
\begin{align}
    &{r}_{113}\, ,& 
    &{r}_{123} = {r}_{213}\, , &
    &{r}_{231} = {r}_{321}\, , \nonumber \\ 
    &{r}_{223}\, ,& 
    &{r}_{131} = {r}_{311}\, , & 
    &{r}_{232} = {r}_{322}\, , \nonumber \\ 
    &{r}_{333}\, ,& 
    &{r}_{132} = {r}_{312}
\end{align}
A rotation by $90^\circ$ around the  
$x_3$ axis and following 
inversion is given by the coordinate transformation 
\begin{equation}
    \begin{split}
    x_1 &\rightarrow -x_2  \\
    x_2 &\rightarrow x_1 \\
    x_3 &\rightarrow -x_3  \, .
    \end{split}
\end{equation}
with the matrix representation
\begin{equation}
    R^{-1} = 
    \begin{pmatrix}
    0 & -1 &  0 \\
    1 &  0 &  0 \\
    0 &  0 & -1 
    \end{pmatrix} \, .
\end{equation}
Applying ths transformation to the remaining components yields:
\begin{align*}
    \hat{r}_{123} &=  r_{213} \, ,&  
    \hat{r}_{213} &=  r_{123} \, , \\
    \hat{r}_{132} &=  r_{231} \, ,&    
    \hat{r}_{231} &=  r_{132} \, , \\
    \hat{r}_{312} &=  r_{321} \, ,&   
    \hat{r}_{321} &=  r_{312} \, , \\
    \hat{r}_{113} &= -r_{223} \, ,&  
    \hat{r}_{223} &= -r_{113} \, , \\
    \hat{r}_{131} &= -r_{232} \, ,&  
    \hat{r}_{232} &= -r_{131} \, , \\
    \hat{r}_{311} &= -r_{322} \, ,&  
    \hat{r}_{322} &= -r_{311} \, , \\
    \hat{r}_{333} &= -r_{333} \,         % !
\end{align*}
We see, that only $r_{333}$ has to vanish, while further 
equalities are introduced. 
To further reduce the number, we  apply the rotation by 
$180^\circ$ around the $x_1$-axis, given by 
\begin{equation}
    \begin{split}
    x_1 &\rightarrow x_1  \\
    x_2 &\rightarrow -x_2 \\
    x_3 &\rightarrow -x_3 \\
    \end{split} \ ; \qquad
    R^{-1} = 
    \begin{pmatrix}
    1 &  0 &  0 \\
    0 & -1 &  0 \\
    0 &  0 & -1 
    \end{pmatrix} \, .
\end{equation}
Since $R^{-1}$ is diagonal, the indices remain the same, thus 
components with changing sign have to vanish. This is the case 
only for an those triples of indices with an 
odd number of indices $i \in \{2, 3\}$. We conclude, that 
the remaining components with corresponding equalities are given by
\begin{align*}
    r_{123} &=  r_{213}\, , \\
    r_{132} &=  r_{312} = r_{231} = r_{321}\,
\end{align*}
These can be renamed in the way given by equations \eqref{eq:notation}, 
which leaves only the three remaining components
\begin{equation}
    r_{41} = r_{52} \qquad \text{ and } \qquad  r_{63}\, .
\end{equation}

\subsection{Application to Pockels Cell}
\label{sec:application_cell}
By applying the result of the foregoing discussion, we can significantly 
reduce the parameters in questions for the pertubed index ellipsoid
\eqref{eq:B_ij}. We resume the most important results:
\begin{itemize}
    \item
    Without the external field, the crystal is uniaxial 
    (refer to section \ref{sec:optical_axes}).
    \item
    We chose the principle dielectric axis system. The 
    components of $\eps$ are named $\eps_\perp$ and 
    $\eps_\parallel$ as defined in equations 
    \eqref{eq:eps_perp} and \eqref{eq:eps_parallel}.  
    \item
    Optical axis and $\bar{4}$-axis coincide. We define 
    this axis to be the $x_3$-axis. Thus, the 
    only non-zero components of the electro-optic tensor 
    $r_{ij}$ are $r_{41} = r_{52}$ and $r_{63}$ in the 
    chosen coordinates, as described in the previous section 
    \ref{sec:reduce}
\end{itemize}


Inserting these conditions in the index generalized index ellipsoid 
\eqref{eq:ellipsoid_pertubed}, 
we arrive at the defining equation
\begin{equation}
    \frac{1}{\eps_\perp}\left(n_1^2 + n_2^2\right) + 
    \frac{1}{\eps_\parallel} n_3^2 + 
    2 r_{41} \left(E_1 n_2 + E_2 n_1\right)n_3 + 
    2 r_{63} E_3 n_1 n_2  = 1\, ,
    \label{eq:ellipsoid_red}
\end{equation}
where, as before, the wave vectors $\K$ are related to 
$\N$ by \eqref{eq:def_n}. 
The Pockels Cell used in this experiment is a transverse one, 
thus the $\E$-field is applied in direction of $x_1$, e.~i. 
$E_2 = E_3 = 0$. This yields
\begin{equation}
    \frac{1}{\eps_\perp}\left(n_1^2 + n_2^2\right) + 
    \frac{1}{\eps_\parallel} n_3^2 + 
    2 r_{41} E_1 n_2 n_3  = 1 \, .
    \label{eq:ellipsoid_trans}
\end{equation}
The crystal is aligned with a $45^\circ$-Y-cut. In order to 
use the laboratory frame as coordinate system, we rotate our 
coordinate system by $45^\circ$ around the $x_1$-axis, 
after which the components of $\N$ read
\begin{align}
    n_1 &= n_1'  \\
    n_2 &= \frac{1}{\sqrt{2}}\left(n_2' + n_3'\right) \\
    n_3 &= \frac{1}{\sqrt{2}}\left(n_2' - n_3'\right) \, .\\
    \label{eq:trans_ycut}
\end{align}
Equation \eqref{eq:ellipsoid_trans} then becomes
\begin{align}
    1 &= 
    \frac{1}{\eps_\perp}
    \left[ {n_1'}^2 + 
    \frac{1}{2} \left({n_2'}^2 + {n_3'}^2 \right) + 
    {n_2'}{n_3'} \right] + 
    \frac{1}{\eps_\parallel}
    \left[\frac{1}{2} \left({n_2'}^2 + {n_3'}^2 \right) -  
    {n_2'}{n_3'} \right] + 
    r_{41} E_1 \left({n_2'}^2 - {n_3'}^2\right) 
    \nonumber \\
    &= 
    \frac{1}{\eps_\perp} {n_1'}^2 + 
    \frac{1}{2} \left(\frac{1}{\eps_\perp} + \frac{1}{\eps_\parallel}\right)
    \left({n_2'}^2 + {n_3'}^2 \right) + 
    \left( \frac{1}{\eps_\perp} - \frac{1}{\eps_\parallel}\right)
    {n_2'}{n_3'}  + 
    r_{41} E_1 \left({n_2'}^2 - {n_3'}^2\right) 
    \nonumber \\
    &= 
    \frac{1}{\eps_\perp}  {n_1'}^2 + 
    {n_2'}^2 \left(\frac{1}{\eps_x} + r_{41} E_1 \right) +
    {n_3'}^2 \left(\frac{1}{\eps_x} - r_{41} E_1 \right) +
    {n_2'}{n_3'} \left( \frac{1}{\eps_\perp} - \frac{1}{\eps_\parallel}\right) \, ,
    \label{eq:ell_calc}
\end{align}
where we defined 
\begin{equation}
    \frac{1}{\eps_x} := 
    \frac{1}{2} \left( \frac{1}{\eps_\perp} + \frac{1}{\eps_\parallel}\right) \, .
    \label{eq:def_eps_x}
\end{equation}
We observe, that the main axes of the ellipsoid are no longer parallel to the 
crystal's princial axes.
For the experiment, we rely on polarized beams. For a ray with 
$\K$ in the $x_2'$-direction, polarized in the area bisecting the 
$x_1'$- and $x_3'$-axis, the component $n_2'$ vanishes.
Equation \eqref{eq:ell_calc} then reduces to
\begin{equation}
    1 = \frac{1}{\eps_\perp}  {n_1'}^2 + 
    {n_3'}^2 \left(\frac{1}{\eps_x} - r_{41} E_1 \right) \, .
    \label{eq:ell_02}
\end{equation}
The wave splits up into two rays, one polarized in $x_1'$-, 
the other in $x_3'$-direction. Accordingly, they experience the refraction 
\begin{align}
    n_1' &= \sqrt{\eps_\perp} 
    \label{eq:n1p} \\
    n_3' &= \frac{\sqrt{\eps_x}}{\sqrt{1 - r_{41} E_1 \eps_x}}
    \label{eq:n3p} 
\end{align}
In order to facilitate the result, we do a scale analysis for the 
parameters of the given Pockels Cell, which are stated in the 
section describing the experimental setup \ref{sec:setup_pockels}.
With $n = \sqrt{\eps / \epsn}$, we get $\eps_x / \epsn \sim 1$, for 
a voltage of $U = 240$V, $E = U / d \sim 10^5$V/m, 
so we can approximate $r_{41} E_1 \eps_x / \epsn \sim 10^{-6}$. 
We can thus expand \eqref{eq:n3p} to the first order in $E_1$ and 
obtain
\begin{equation}
    n_3' \approx \sqrt{\eps_x} + 
        \frac{1}{2} r_{41} E_1 {\eps_x}^\frac{3}{2}
    \label{eq:n3paprrox} \, .
\end{equation}
For the phase difference of the two components after passing through a crystal 
of length $l$, we get with \eqref{eq:dphi}:
\begin{equation}
    \begin{split}
    \Delta \phi_1 
    &= \frac{2l \pi}{\lambda} \left(n_3' - n_1' \right) \\
    &= \frac{l \pi}{\lambda}\Bigg[
    \underbrace{r_{41} E_1 {\eps_x}^\frac{3}{2}}_\text{Pockels effect} + 
    \underbrace{\left(\sqrt{\eps_x} - \sqrt{\eps_\perp} 
    \right)}_\text{natural birefraction}
    \Bigg]
    \end{split}
    \label{eq:dphi_1}
\end{equation}
In order to reunited the beams that are split up by the Pockels effect, 
we let the beam pass another crystal and corresponding $\E$-field 
turned around together by $180^\circ$. 
Since the situation is symmetric for the phase difference, we get for the 
\begin{equation}
    \Delta \phi_2 = \Delta \phi_1
\end{equation}
resulting phase difference  
The second summand in \eqref{dphi_1} corresponds to the natural birefringence. 
It can be cancelled out by applying another pair of crystals turned by 
$90^\circ$ towards the first one. In this case, the beam polarized 
in the former $x_1'$-direction is subject to the refractive index 
$n_3'$ and vice versa. In order to not cancel the part depending on $\E$, 
the field has to be turned around with respect to the new coordinates. 
The phase difference is then given by 
\begin{equation}
    \begin{split}
    \Delta \phi_3 = \Delta \phi_4
    &= \frac{2l \pi}{\lambda} \left(n_1' - n_3' \right) \\
    &= -\frac{l \pi}{\lambda}\left[
    r_{41} (-E_1) {\eps_x}^\frac{3}{2} + 
    \left(\sqrt{\eps_x} - \sqrt{\eps_\perp}
    \right)
    \right]
   \end{split}
    \label{eq:dphi_34}
\end{equation}
Adding all phase differences yields
\begin{equation}
    \begin{split}
    \Delta \phi_\mathrm{total}  
    &= \Delta \phi_1 + \Delta \phi_2 + \Delta \phi_3 + \Delta \phi_4 \\
    &= \frac{4l \pi}{\lambda} r_{41} E_1 {\eps_x}^\frac{3}{2} 
    \end{split}
    \label{eq:dphi_34}
\end{equation}
At $\Delta \phi = \pi$, the Pockels Cell acts as a \emph{half-wave plate}. 
In this case, light will pass through orthogonally oriented polarizers 
with the cell in between entirely only for the correspoding $\E_{\lambda/2}$,
while for $\E = 0$, no light is transmitted. Using $E = U / d$, we can thus 
define $U_{\lambda / 2}$ and use the setup to measure the electro-optic 
coefficient 
\begin{equation}
    r_{41} = \frac{\lambda d}{4 l U_{\lambda / 2}} 
    \left(\frac{1}{2} \left(\frac{1}{\eps_\perp} + \frac{1}{\eps_\parallel}\right)\right)^{\frac{3}{2}}
    \label{eq:r_41_U}
\end{equation}

