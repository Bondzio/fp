\section{Theory behind the Pockels effect}

\subsection{Linear and circular polarized light}

\subsection{Birefringence and index ellipsoid}
If the refraction index $n$ of a material depends on the linear polarization of light, 
then a beam of light propagating in the material will be split up into 
two beams with perpendicular polarization and different propagation speed 
\beq
   v_i = \frac{c}{n_i}, 
\eeq
In order to explain this phenomenon, it is helpful to introduce quantities 
connected to the structure of the material, i.~e. it being anisotropic, 
and retrieve the way the light propagates by looking at the solutions 
of Maxwell's equations in matter, 
\bea
\nabla \cdot \D &=& \rho_\text{f} 
\label{eq:max1} \\ 
\nabla \cdot \B &=& 0
\label{eq:max2} \\ 
\nabla \times \E &=& -\frac{\partial \B} {\partial t}
\label{eq:max3} \\ 
    \nabla \times \mathbf{H} &=& \mathbf{J}_\text{f} + 
        \frac{\partial \D} {\partial t}, 
\label{eq:max4}
\eea
where 
\bit
\item $\D$ is the electric displacement field, related to the eletric field $\E$ 
    by the constitutive equation 
    \beq
        \D = \eps \E 
    \label{eq:const1}
    \eeq
\item $\mathbf{H}$ is the magnetizing field, with the constitutive equation 
    \beq
        \mathbf{H} = \mu \B
    \label{eq:const2}
    \eeq
\item $\mathbf{J}_\text{f}$ is the free current density, and
\item $\mathbf{\rho}_\text{f}$ is the free charge density.
\eit
Equations \eqref{eq:const1} and \eqref{eq:const2} are valid for materials without 
coupling between magnetic and eletric fields, which is the case in our 
crystal. We are, however, facing an anistropic material, so that 
$\eps$ is a tensor, while we assume the permeability $\mu$ of the material 
to be just the vacuum permeability $\mu_0$, such that
$\B = \mu \mathbf{H} = \mu_0 \mathbf{H}$. 
If we assume $\mathbf{\rho}_\text{f} = 0$, $\mathbf{J}_\text{f} = 0$ 
(no free charge and currents), then applying the curl 
operator to \eqref{eq:max3} and \eqref{eq:max1} yields:
\bea
    \nabla \times \nabla \times \E &=& 
    - \frac{\partial }{\partial t} \left(\nabla \times \B\right)  \nn
    &=& - \mu_0 \frac{\partial }{\partial t} \left(\nabla \times \mathbf{H}\right) \nn
    &=& - \mu_0 \frac{\partial }{\partial t} \left(\mathbf{J}_\text{f} + 
        \frac{\partial \D} {\partial t}\right) \nn
    &=& -\mu_0 \frac{\partial^2 \D }{\partial t^2}  \nn
    &=& -\mu_0 \frac{\partial^2}{\partial t^2} \eps \E
    \label{eq:max3b} \\
    \nabla \cdot \D &=& 0
    \label{eq:max1b}
\eea
For plane waves, described by
\beq
    \E_\K = \E_0 \exp \left[i(\mathbf{k \cdot r}-\omega t)\right] \, , 
\eeq
inserting in \eqref{eq:max3b} and \eqref{eq:max1b} gives
\bea
    |\mathbf{k}|^2\En-\mathbf{(k \cdot E_0) k} 
    &=& \mu_0 \omega^2 (\mathbf{\eps} \, \En)
    \label{eq:max3c} \\
    \K \cdot (\eps \, \En) &=& 0.
    \label{eq:max1c}
\eea
By applying the principle axis theorem~\cite{strang2003introduction}, 
we can write $\eps$ as the diagonal matrix 
\beq
\mathbf{\epsilon}= \epsilon_0 
\begin{bmatrix} 
n_x^2   & 0     & 0 \\ 
0       & n_y^2 & 0 \\ 
0       & 0     & n_z^2 
\end{bmatrix} \, ,
\eeq
with refractive indeces $n_x, n_y, n_z$ along the principle axis. 
whereby \eqref{eq:max3c} and \eqref{eq:max1c} yield the solutions
\bea
\left(-k_y^2-k_z^2+\frac{\omega^2n_x^2}{c^2}\right)E_x + k_xk_yE_y + k_xk_zE_z &=& 0 \\
k_xk_yE_x + \left(-k_x^2-k_z^2+\frac{\omega^2n_y^2}{c^2}\right)E_y + k_yk_zE_z &=& 0 \\
k_xk_zE_x + k_yk_zE_y + \left(-k_x^2-k_y^2+\frac{\omega^2n_z^2}{c^2}\right)E_z &=& 0
\eea
with components $E_i$ and $k_i$ of $\E$ and $\K$ along the $i$-axis, $i \in \left\{x, y, z\right\}$.
The solutions of this set of equation linear in $E_i$ is solved if the determinant vanishes:
\beq
\begin{vmatrix} 
\left(-k_y^2-k_z^2+\frac{\omega^2n_x^2}{c^2}\right) & k_xk_y & k_xk_z \\ 
k_xk_y & \left(-k_x^2-k_z^2+\frac{\omega^2n_y^2}{c^2}\right) & k_yk_z \\ 
k_xk_z & k_yk_z & \left(-k_x^2-k_y^2+\frac{\omega^2n_z^2}{c^2}\right) 
\end{vmatrix} =0\, 
\eeq
Rearraging the solution of this equation yields
\beq
\frac{\omega^4}{c^4} - 
\frac{\omega^2}{c^2}
\left(\frac{k_x^2+k_y^2}{n_z^2} + \frac{k_x^2+k_z^2}{n_y^2} + \frac{k_y^2+k_z^2}{n_x^2}\right) + 
\K^2\left(\frac{k_x^2}{n_y^2n_z^2} + \frac{k_y^2}{n_x^2n_z^2} + \frac{k_z^2}{n_x^2n_y^2}\right) 
= 0 
\label{eq:sol_general}
\eeq

The effect of birefraction can be explained for biaxial crystals, first, being defined 
by roational symmetry around one axis. If we chose the $z$-axis, we can rename of 
refraction indices into $n_x = n_y =: n_o$ and $n_z =: n_e$, where $o$ and $e$ stand for 
ordinary and extraordinary, respectivly, as will be explained below. In this case, 
we can factor equation \eqref{eq:sol_general} into 
\beq
\left(\frac{k_x^2}{n_o^2}+\frac{k_y^2}{n_o^2}+\frac{k_z^2}{n_o^2} -\frac{\omega^2}{c^2}\right)
\left(\frac{k_x^2}{n_e^2}+\frac{k_y^2}{n_e^2}+\frac{k_z^2}{n_o^2} -\frac{\omega^2}{c^2}\right)
=0 \, . 
\label{eq:sol_biax}
\eeq
We get two sets of legal $\K$-vectors by setting either of the two factors to zero. 
The first set corresponds to the surface of a sphere with radius $n_o$, 
the second to a spheroid with rotational symmetry about the $z$-axis. 



biaxil - uniaxial 
optical axis 


\subsection{Deducing the Pockels effect in terms of nonlinear Maxwell equations}
\subsubsection{Primary and secondary electro-optical effect}
\subsubsection{Influence of electric field on the index ellipsiod}

\subsection{Structure of crystal lattice}

\subsection{Reduction of components of dielctric tensor for $\bar{4}2$m crystals}

\subsection{Discussion of the effect for the transverse Pockels Cell}

\subsection{Polarization filters}

\subsection{Methods of determining $U_\lambda$}
\subsubsection{Saw tooth}
\subsubsection{Modulated direct current}

\subsection{Experimental setup}
\subsubsection{Properties of He-Ne-Laser}
\subsubsection{Properties of used photodiode}
