\section{Conclusion}
\subsection{Conclusion Faraday experiment}
The angle between the Polarizationsurfaces of the two Nicols was calculated as:\\
\begin{equation*}
2\epsilon = (162\pm 6) ^{\circ} 
\end{equation*}
Since we have no reference to this value we cannot judge about whether this result is promising
or not, however we notice that the precission of the measurements were rather low.
Furthermore were able to exploit the
faraday phenomena to estimate the Verdet constant of flintglass.
Our result is:
\begin{align*}
    V &= \frac{\alpha}{I\mu}   = p_2 / \mu \\
      &= \left [ 1.007 \pm 0.008 \right ]\cdot 10^{-3} \quad \mathrm{Deg/A} \\
      &= \left [ 0.04868 \pm 0.00022 \right ] \cdot 10^{-3} \quad \mathrm{Min/Oe\cdot cm}
\end{align*}
Where we use $1\ \mathrm{Oe} = \frac{1000}{4\pi}\ \mathrm{A/m}$, $\quad1$ Deg $=60$ min. and $\mu = 2587.5\pm16.4$.
The producer gave the value $V = 0.05 \quad \mathrm{Min/Oe \cdot cm}$ but without uncertainty, so
we have to conclude that our value might be correct within the given experimental setup. 
