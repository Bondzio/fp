\section{Conclusion}
\subsection{Conclusion Pockels experiment}
The result for the sawtooth method seems to be reasonable with regards to the uncertainty
due to the experimental setup:
\begin{equation}
    \text{mean: } r_{41}= (26.7 \pm 2.9) pm
\end{equation}
Where the reference value is given by $r_{41} = 23.4$pm at $21^{\circ}$C. Our error is too low,
but one has to consider that we did not include thermal effects or optical effects in the 
estimation of the error.
For the DC current method we arrived at value far from the reference value:
\begin{equation}
    r_{41} = (41.7 \pm 2.3) \textrm{pm} 
\end{equation}
As we have mentioned before there are various possibilites for the disagreement with
the reference value: Uncertainies in the setup and electronics,
wrongly setup of the polarization filters, 
which might may not have been aligned properly with an 
angle of $45\Deg$.  \\
Furthermore we want to emphasize the fact that our experimental setup
was rather sensitive to the frequency, which is a huge disagreement to the
approximative theory building our calculations on. But since this is in
an order of magnitude of $20$V this would still not lead to an explanation,
how we achieve such a high value, since the difference to the expected 
voltage was about $100$V. We can assume that natural birefraction or 
secondary electro-optical effects this big difference, but more probable is 
a error in our experimental setup or the measurements. Since we could observe
the frequency doubling and took $U_{\lambda/2}$ from this regime, we 
conclude that the experiment was quite sucessfull in this category.



\subsection{Conclusion Faraday experiment}
The angle between the Polarizationsurfaces of the two Nicols was calculated as:\\
\begin{equation*}
2\epsilon = (162\pm 6) ^{\circ} 
\end{equation*}
Since we have no reference to this value we cannot judge about whether this result is promising
or not, however we notice that the precission of the measurements were rather low.
Furthermore were able to exploit the
faraday phenomena to estimate the Verdet constant of flintglass.
Our result is:
\begin{align*}
    V &= \frac{\alpha}{I\mu}   = p_2 / \mu \\
      &= \left [ 1.007 \pm 0.008 \right ]\cdot 10^{-3} \quad \mathrm{Deg/A} \\
      &= \left [ 0.04868 \pm 0.00022 \right ] \cdot 10^{-3} \quad \mathrm{Min/Oe\cdot cm}
\end{align*}
Where we use $1\ \mathrm{Oe} = \frac{1000}{4\pi}\ \mathrm{A/m}$, $\quad1$ Deg $=60$ min. and $\mu = 2587.5\pm16.4$.
The producer gave the value $V = 0.05 \quad \mathrm{Min/Oe \cdot cm}$ but without uncertainty, so
we have to conclude that our value might be correct within the given experimental setup. 
