\section{Evaluation for Pockels effects}
\subsection{Fourier analysis of frequencies}
As described before \ref{sec:pdf}, we use Fourier analysis to determine 
the input voltage $U_\mathrm{DC}$ at which frequency doubling occurs. 
To illustrate the procedure, we show an example at $U_\mathrm{DC} = 137.0$ V.
The cut-off is taken at $t = 1.1$ ms and $t = 9.05$ ms, as shown in figure 
\ref{fig:cut_off_example}. The result of the FFT is then shown in figure 
\ref{fig:fft_example} for a larger range of frequencies and those of interest, 
namly $\nu < 2$ kHz. One clearly observes the two peaks close to 
0.5 kHz and 1.0 kHz, being of the same scale. This procedure is 
done with all datasets corresponding to the signals shown in 
the figures \ref{fig:}



\subsection{Calculation of the electro-optic coefficient}
Now we can use the former calculation to calculate the electro-optic coefficient.
\begin{equation}
    r_{41} = \frac{\lambda d}{4 l U_{\lambda / 2}} 
    \left(\frac{1}{2} \left(\frac{1}{\eps_\perp} + \frac{1}{\eps_\parallel}\right)\right)^{\frac{3}{2}}
    \label{eq:r_41_U}
\end{equation}
Now we use the following constraints (these were given before):
\begin{itemize}
\setlength\itemsep{0em}
\item[] $\lambda = (632.8\pm 0.1)$nm
\item[] $n_1     = 1.522\pm 0.001$
\item[] $n_3     = 1.477\pm 0.001$
\item[] $l       = ( 20\pm 0.1)$mm
\item[] $d       = (2.4\pm 0.1)$mm
\end{itemize}
Now we can calculate $r_{41}$ with respect to $U_{\lambda/2}$ with two different methods.
\subsubsection{Saw tooth method}
\label{ssub:Saw tooth method}

\subsubsection{DC current method}
\label{ssub:DC current method}
We estimated $U_{\lambda/2}    = (135  \pm 15)$.
% r !python analysis_pockels/r41_calc.py
Our result is the following: 
\begin{equation*}
r_{41} = \left (20.9 \pm 2.5 \right )  \mathrm{pm}
\end{equation*}
