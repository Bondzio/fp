\section{Introduction}
In this experiment, we examine two effects of birefringence induced by 
external fields: the electro-optical Pockels effect and the magneto-optical 
Faraday effect. Both effects change the susceptibility of a material 
in the first order. 

\paragraph{The Pockels effect} describes the birefringence induced by
an external electric field, resulting from a change in the refractive index 
of a crystal depending on the direction of linear polarization. The change 
in refraction index if proportional, distinguishing it from the quadratic order 
Kerr effect.

\paragraph{The Faraday effect} on the other hand produces a rotation 
of the plane of polarization of a linearly polarized beam proportional 
to the component of a magnetic field in the direction of the beam. This 
is also referred to as circular birefringence, as the effect is based 
on a dependency of propagation speeds on the direction of circular 
polarization. Since linear polarized light can be described as a 
superposition of two circularly polarized components of equal amplitude 
but opposite handedness, the induced phase difference results in a 
rotation of plane of polarization. 
