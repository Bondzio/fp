\section{Introduction}
\paragraph{In this experiment,} we examine two effects of 
birefringence induced by 
external fields: the electro-optical \emph{Pockels effect} and 
the magneto-optical \emph{Faraday effect}. 
Both effects have in common their linear dependence on the 
field applied. 

\paragraph{The progagation speed} $v$ of light in a material is determined 
by the \emph{index of refraction} or 
\emph{refraction index} $n$ according to 
\beqn
   v = \frac{c}{n}, 
\eeqn
where $c$ is the speed of light in vacuum. 
In many materials, this parameter depends on several other quantities, 
such as wavelenght $\lambda$ of the light, leading to \emph{dispersion}, 
or the intensity of the incoming beam, leading to the nonlinear 
\emph{Kerr effect}. Often, it depends on the direction of polarization 
of the light. Dependence on linear polarization leads to \emph{birefringence}, 
where a beam is split up into two linearly polarized 
components following two different paths.  
Dependence on circular polarization, on the other hand, leads to 
\emph{circular birefringence}, 
or a difference in phase between two circularly polarized components, 
resulting in the rotation of the plane of polarization of linearly 
polarized light.
While an example of the latter one is induced by a magnetic field in the 
Faraday effect, 
the linear birefringence is mostly associated with the anisotropic structure 
of a material, observed for example in asymmetric crystals or in plastics 
when isotropy is lost due to mechanical stress. It can, however, be induced 
by external eletric fields as well, leading to the 
Pockels effect. Due to its dependency on the dielectric tensor, it 
is observed only in crystals without inversion symmetry about a point. 
To observe the effect, an alignment of cristals, called a 
\emph{Pockels Cell} is used, cancelling out the effects of birefringence 
without external field. 

