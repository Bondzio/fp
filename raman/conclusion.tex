\section{Conclusion}
We conclude this report with a review on the experiment and the most important results. During the setup we faced the
problem of choosing an order in which to experiment. We decided to start measuring the spectra with the much faster CCD
and do more exact measurements with the monochromator thereafter. This approach failed in so far as we were not able to 
obtain any good results with the monochromator. We thus turned back to use the CCD and did in the end get results 
which allowed us to complete all central tasks. 

In the analysis of the data, we started with the monochromator. Testing the calibration, we found no need to recalibrate. 
We quantified the detection probability of the device and took a closer look on the effect of the slit width. Using this 
analysis, we decided to work with an opening of 100 $\mu$m. We further characterized the upper boundary of detection in 
terms of wavelength: above 590 nm, the ability to detect radiation rapidly decreases. The recorded spectra of carbon 
disulfide, chloroform and carbon tetrachloride are displayed and visible peaks located. Due to the low resolution of the 
data, no fitting was possible. However, the simple location of maxima yielded results mostly comparable to literature 
values (with one exception in the spectrum of CHCl$_3$). In general, Stokes peaks were much easier to identify then 
the corresponding Anti-Stokes peaks. 

The second part of the analysis was based on data taken with the CCD spectrometer. The test of calibration was done more 
thoroughly with a linear regressions on the peaks of the Hg-lamp. Testing the detection probability for white light 
vertically and horizontally polarized, we derived a correction factor for the following analysis. We further investigated
the parameters of the laser, measuring a wavelength of $(532.1 \pm 0.3)$ nm in good agreement with the manufacturer's
declaration. A qualitative analysis of the notch filter was later on used to distinguish Raman peaks from artifacts of 
the filter. 

The most substantial part of this report was the examination of spectra of a number of samples. The three samples quoted 
before yielded similar but more detailed results: We found both the Stokes and the Anti-Stokes peak of CS$_2$, four 
Stokes and three Anti-Stokes peaks for CHCl$_3$ as well as three Stokes and corresponding Anti-Stokes peaks in the 
spectrum of CCl$_4$. Two further peaks could not be identified and are interpreted as artifacts of the setup. The only 
notable drawback compared to the monochromator is the limited range due to the notch filter (or the overflow when not 
using it). The vibrational modes with $\Delta \nu = 260 \text{ cm}^{-1}$ and $\Delta \nu = 217 \text{ cm}^{-1}$ in the 
spectra of CHCl$_3$ and CCl$_4$ are blocked by the filter but visible with the monochromator. 

The measurements with CCl$_4$ were further used to calculate the depolarization ratio $\rho_s $of three of its vibrational
modes. Our results are unanimously too large, indicating a considerable depolarization within the setup. Two of the 
scrutinized vibrational modes are not symmetric and thus expected to lead to depolarization. However, our results for the 
ratio are 20\% to 25\% higher then the values stated in literature. For the third, symmetric mode, we measured a value
of $\rho_s(\Delta \nu = 459 \text{ cm}^{-1}) = 0.287 \pm 0.003$ (for the Anti-Stokes peak) while the theory expects no
depolarization at all and the literature value is $\rho_\text{lit} = 0.026$. 

We continued by determining the concentration of ethanol in an unknown mixture with water. The difficulty of dealing with 
the underlying distribution due to fluorescence was solved by fitting a polynomial and removing the corresponding part. 
The intensity of five Stokes peaks were used in a linear regression over the known concentrations of our prepared 
mixtures. For each of these peaks, we intersected the measured intensity of the original sample with the according 
straight line. Averaging the four intersects yielded a concentration of $(76 \pm 4)$ \%.

Finally, we tried to calculate the temperature of the probe using the Stokes and Anti-Stokes peak's intensity ratio. 
Although this is expected to be constant for constant temperature, our measured values did fluctuate quiet remarkably. The 
resulting estimate is $T = (29 \pm 8) ^\circ C$, higher then the room temperature of 
$T_\text{room} = (20.3 \pm 0.3) ^\circ C$. The result is still much closer to the actual value than expected after reading
of several reported failures of this attempt.

In summary, the experiment can be considered successful in terms of learning and applying numerous concepts connected to 
the Raman effect. Despite the described experimental restrictions, our results turned out to be mostly satisfying or even
surprisingly exact. 
