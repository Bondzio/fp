\section{Theoretical introduction}
Light interacting with matter is still one of the most important aspects of physics, including
absorption, emission, but also scattering. In this report we will guide our attention to one
particular form of light interaction: \textit{Raman scattering}, which is widely used in spectroscopy
in many different areas, from applied chemistry to aerospace engineering. We will illustrate the 
basic functionalism through a series of simple experiments.

\subsection{Classical description}
\label{sub:classical_description}

\subsubsection{Oscillation of molecules}
\label{ssub:Oscillation_of_molecules}
Let us first investigate~\cite{landau1997mechanik} the classical oscillation of molecules, their normal modes and how a classical 
electric field interacts with them. When talking about a system of particles interacting with each other, not all degrees
of freedom will be oscillatory behavior. In general for a n-atomic molecule 
the dimension of the oscillation will be $3n - 6$ and $3n -5$ for a linear molecule (lying on a line), since three for
translation and three rotation of the whole molecule are subtracted. Now let us remove these 6 (5) dimensions from the
equation of motion and transform into the rotation and move less center of mass system:
\begin{equation}
    \sum_a m_a r_a = \mathrm{const} \equiv \sum_a m_a r_{a,0}.
\end{equation}
Using new coordinates, we express this condition as
\begin{equation}
    \sum_a m_a x_a = 0.
\end{equation}
This conditions out coordinates with respect to translation, now we need to find a constraint
to remove the rotations. Let us look at the angular momentum $M$: 
\begin{equation}
    M = \sum_a m_a r_a \times v_a \approx \sum_a m_a r_{a 0} \times \dot{x}_a = \frac{\partial}{\partial t}
    \sum_a m_a r_{a 0} \times u_a
\end{equation}
where we only look at small deviations $r_a = r_{a0} + x_a$. We can let it vanish if we demand
\begin{equation}
    \sum_a m_a r_{a0} \times x_a = 0.
\end{equation}
Remember that $u$ characterizes the inherent oscillations out of center of mass. Let us now
look at the energy of the molecule. Let a potential energy $U$ lead to the oscillation around zero,
then we can approximate it as a quadratic form with coefficients $k_{ik}$ (a generalized, coupled
harmonic oscillator):
\begin{equation}
    U = \frac{1}{2} \sum_{i,k} k_{ik} x_i x_k
\end{equation}
Hence the total energy becomes:
\begin{equation}
    E = \frac{1}{2} \sum_{i,k} m_{ik} \dot{x}_i\dot{x}_k + \frac{1}{2} \sum_{i,k} k_{ik} x_i x_k   
\end{equation}
with a linear transformation we can go into the uncoupled \textit{normal coordinates} $q$:
\begin{equation}
    E = \frac{1}{2} \sum_{i,a} \dot{q}^2_{ai} + \frac{1}{2} \sum_a \omega_a^2 \sum_i q^2_{a,i} 
\end{equation}
Where as before $a$ goes over the different modes and $i$ over the spatial dimensions.
Let us go back to the coordinates $x$ and look at the solution of the classical equation
of motion. The Lagrange function reads
\begin{equation}
    L = \frac{1}{2} \sum_{i,k} (m_{ik} \dot{x}_i\dot{x}_k -  k_{ik} x_i x_k   )
\end{equation}
The partial derivatives become 
\begin{equation}
    \frac{\partial L}{\partial \dot{x}_i} = \sum_k m_{ik} \dot{x}_k , \quad 
    \frac{\partial L}{\partial x_i} = - \sum_k k_{ik} x_k.
\end{equation}
The \textit{Euler Lagrange} equation is then 
\begin{equation}
    \label{eq:el}
    \sum_k (m_{ik} \ddot{x_k} + k_{ik} x_k) = 0 
\end{equation}
This is a linear matrix differential equation which can be solved in terms of exponentials
\begin{equation}
    x_k = A_k \exp(i\omega t).
\end{equation}
so~\eqref{eq:el} becomes 
\begin{equation}
    \sum_k \left( -\omega^2 m_{ik} + k_{ik} \right)A_k = 0
\end{equation}
which finally give rise to the eigen frequencies $\omega$
\begin{equation}
    |k_{ik} - \omega^2 m_{ik}| = 0. 
\end{equation}
Fortunately, it is not necessary to calculate those in every case, since
molecules are often governed by easy symmetries. The task simplifies to
find the respective symmetry group and its irreducible representations~\cite{landau1977quantum}. 


\subsubsection{Electric fields and dipoles}
\label{ssub:Electricfieldsanddipoles}
Let us examine classical electromagnetic waves. The solution to the homogeneous
wave equation in cylindrical coordinates reads
\begin{equation}
    E(x,t) = |E| \textrm{Re} \left[ \begin{pmatrix}
            \cos(\theta)\exp(i \alpha_x)\\ 
            \sin(\theta)\exp(i \alpha_y)
    \end{pmatrix}\exp(i(kz - \omega t))  \right].
\end{equation}
The phases $\alpha_x$ and $\alpha_y$ characterize the polarization. If the polarization
is linear, the condition is
\begin{equation}
    \alpha_x = \alpha_y = \alpha.
\end{equation}
The radiation field $E$ of a dipole
can be described with the dipole moment $M$ and in terms of the electric potential $\phi$
\begin{equation}
\phi \approx \frac{\textbf{M}\cdot \mathbf{R}}{4\pi\epsilon_0 R^3} \Rightarrow E = - \nabla \phi 
= \frac{3 \mathbf{(M \cdot n) n - M}}{4 \pi \epsilon_9 R^3}.
\end{equation}
with the unit vector $\textbf{n} = \frac{\textbf{R}}{R}$. We can use this knowledge now for 
generalization.
Beginning with a Hertzian dipole~\cite{koningstein1972introduction}, the total intensity
per second can be expressed in terms of the radiation field of the dipole 
\begin{equation}
    \label{eq:hertz}
   I = \frac{2}{3 c^2} \left \langle \frac{\mathrm{d}^2}{\mathrm{d} t^2}  M \right \rangle_t.
\end{equation}
Where the brackets denote to the time average and $M$ is the electric dipole moment. Let us assume,
we induce a dipole by an incoming electric field (for instance a vibrating molecule)
Then the dipole can in general be written in terms of the electric field $E$ with
\begin{equation}
    M_i = \sum^{3}_{j=1} \alpha_{ij} E_j(t) \equiv {(\alpha E )}_i
\end{equation}
with the polarization tensor $\alpha$.
Let us assume now a incoming plain wave 
\begin{equation}
    E_j = E^0_j \cos(\omega_L t).
\end{equation}
For an isotropic molecule this is an scalar and we get
for the intensity by~\eqref{eq:hertz}:
\begin{equation}
    I = \frac{16 \pi^4 c}{3 \lambda_0^4} \alpha ^3 E_0 ^2
\end{equation}
with $\lambda_0 = c/ 2\pi \omega_0$. Transforming into normal coordinates $q(t)$ we can write down the polarization tensor
in a Taylor expansion around zero\cite{ver}:
\begin{equation}
    \alpha = \alpha_0 + \frac{\partial\alpha}{\partial q}q(t) + \cdots \approx \alpha_0 + \alpha' q_0 \cos(\omega_M t).
\end{equation}
Where we approximated the vibration of the isotropic molecule by a frequency $\omega_M$.
The dipole moment then becomes
\begin{align}
    M &= \alpha_0 E^0 \cos(\omega_L t) + \alpha' E^0 \cos(\omega_L t) \cos(\omega_M t) \\
      &= \alpha_0 E^0 \cos(\omega_L t) + \alpha ' ((\omega_L + \omega_M)t ) + \alpha ' E^0 ((\omega_M - \omega_L)t) 
\end{align}



\subsection{Depolarization}

siehe Raman buch s.129
