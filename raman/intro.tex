\section{Introduction}
Atomic and molecular spectroscopy continues to provide new astonishing results, but already the early results are fascinating
and instructive for understanding the fundamental properties of matter and their interaction with light: In particular, in this
report we will give an compact introduction to Raman spectroscopy,
e.g.\ the inelastic scattering of photons on molecules, both from theoretical and experimental aspects. Predicted theoretically 
by Adolf Smekal in 1923, it was discovered in liquids by C. V. Raman and K. S. Krishnan in 1928\cite{venkataraman}. 
Today it has numerous applications in research as well as in industrial contexts, where it is used to identify materials. 
Several of these aspects are highlighted by the experiment. Equipped with two spectrometers and a Nd:Yag laser as a source
of photons, we will analyze the spectra of a number of samples for Raman peaks and use this information to determine the 
concentration of ethanol in a mixture with water, to estimate the temperature of a sample and to infer properties of 
molecular vibrational modes by calculating their depolarization ratios. 

